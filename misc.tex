\documentclass[11pt]{article}
\usepackage{array,multirow,amsthm,amsmath,amssymb,url}
% \usepackage{fullpage}
\newcommand{\RA}{\vdash}
\usepackage{mVersion}
\usepackage{natbib}
\usepackage{bussproofs}
\usepackage{mdframed}
\usepackage{tikz}
\usepackage{tikz-cd}
\title{Metatheory of Propositional Logic I}
\author{Hans Halvorson}
\date{\today}
% \setlength{\parindent}{0em}
% \setlength{\parskip}{1em}
\swapnumbers
\newtheorem{prop}{Proposition}
\newtheorem{thm}[prop]{Theorem}
\newtheorem{cor}[prop]{Corollary}
\newtheorem{lemma}[prop]{Lemma}
\newtheorem{claim}[prop]{Claim}
\newtheorem{fact}[prop]{Fact}
\newtheorem*{subthm}{Substitution Theorem}
\newtheorem{conj}[prop]{Conjecture}
\theoremstyle{definition}
\newtheorem*{defn}{Definition}
\newtheorem*{example}{Example}
\theoremstyle{remark}
\newtheorem{note}[prop]{Note}
\newtheorem*{disc}{Discussion}
\swapnumbers
\newtheorem{exercise}{Exercise}
% \usepackage{fbb}
\newcommand{\df}[1]{\textbf{#1}}
\begin{document}

\section{Extensions of Theories}

We'll now discuss several ways in which a theory $T$ can be extended
to a new theory $T'$.  Recall that in our terminology, $T'$ is an
\textbf{extension} of $T$ just in case $T\vDash \phi$ implies
$T'\vDash \phi$ for all $\phi$ in the original language of $T$, or
equivalently, $Cn(T)\subseteq Cn(T')$.  Here's a rough classification
of types of extensions:
\begin{enumerate}
\item Adding new axioms.
\item Adding new vocabulary. 
\item Some combination of the above two, e.g.\ adding new vocabulary
  and linking it to old vocabulary by means of axioms.
\end{enumerate}
We'll go through these case by case.

\subsection{Adding new axioms}

Let $\Sigma$ be a fixed signature, and let $T$ and $T'$ be theories in
$\Sigma$ such that $T\subseteq T'$.  By definition, $T'$ is an
extension of $T$.  Thus, the identity reconstrual $I:\Sigma\to\Sigma$
gives rise to a translation $I:T\to T'$, which is essentially
surjective. However, $I$ will not be conservative unless $T'$ and $T$
have the same consequences, i.e.\ unless $T'$ is not a proper
extension of $T$.

Let's consider a couple of examples.  First, let $\Sigma = \{ p,q\}$,
let $T$ be the empty theory, and let $T'$ be the theory with the
single axiom $\vdash p\leftrightarrow q$.  Then $I:T\to T'$ is a
translation, but it's clearly not conservative.  Consider also the
dual map $I^*:\mathrm{Mod}(T')\to \mathrm{Mod}(T)$.  Clearly,
$\mathrm{Mod}(T)$ has exactly four points, corresponding to all
possible truth-assignments to $p$ and $q$; and $\mathrm{Mod}(T')$ has
exactly two valuations, the one that assigns $1$ to both $p$ and $q$,
and the one that assigns $0$ to both $p$ and $q$.  In other words,
$\mathrm{Mod}(T')$ is the set of sub-models of $\mathrm{Mod}(T)$ that
satisfy the additional axiom of $T'$, and $I^*:\mathrm{Mod}(T')\to
\mathrm{Mod}(T)$ is the canonical inclusion.

Second, let $\Sigma = \{ p_0,p_1,\dots \}$, let $T$ be the empty
theory in $\Sigma$, and let $T'$ be the theory with axioms $\{
p_0\vdash p_i :i=0,1,\dots \}$.  Then $T\subseteq T'$, and $I:T\to T'$
is a translation.

In summary, extending $T$ by adding axioms is tantamount to
restricting $\mathrm{Mod}(T)$ to one of its proper subsets, i.e.\ to
eliminating certain models from $\mathrm{Mod}(T)$.

[[TO DO: Add picture]]

These considerations raise two further questions.  First, given a
subset $A$ of $\mathrm{Mod}(T)$, is there an extension $T'$ of $T$
such that $A=\mathrm{Mod}(T')$?  This question is relevant to debates
about the semantic versus syntactic view of theories.  One virtue of
the semantic approach is supposed to be the greater flexibility
provided by identifying theories with classes of models.  In that
case, an arbitrary subset $A$ of $\mathrm{Mod}(T)$ can be thought of
as corresponding to an extension of the theory $T$, even though $A\neq
\mathrm{Mod}(T')$ for any theory $T'$.  Second, given an essentially
surjective translation $F:T\to T'$, where $T'$ might be formulated in
a different signature from $T$, could we still think of $T'$ as
resulting from adding new axioms to $T$?  Let's consider these two
questions in turn.

1. For $A\subseteq \mathrm{Mod}(T)$, under what conditions is
$A=\mathrm{Mod}(T')$, where $T'$ is an extension of $T$?  

\begin{prop} Let $A\subseteq X_T$.  Then $A=X_{T'}$ for some extension
  $T'$ of $T$ iff $A$ is a closed subset of $X_T$. \end{prop}

\begin{proof} Suppose that $A$ is closed in $X_T$.  Thus, $X_T-A$ is
  open, which means that $X_T-A=\bigcup _{i\in I}[\phi _i]$, and
  $A=\bigcap _{i\in I}[\neg \phi _i]$.  Let $T'=\{ \neg \phi _i:i\in
  I\}$.  Then $v\in X_{T'}$ iff $v(\neg \phi _i)=1$ for all $i\in I$,
  iff $v\in [\neg \phi _i]$ for all $i\in A$, iff $v\in A$.
  Therefore, $X_{T'}=A$.
\end{proof}

TO DO: Relation to Keisler characterization result.



%% TO DO: what I want to say now is that A closed means precisely that
%% A contains limits of sequences.  

Recall that when $T$ is formulated in a countable signature, the Stone
space $X$ of $T$ is first-countable --- and so closed sets are
characterized by their containing the limits of convergent sequences.
That is, a subset $A\subseteq X$ is closed iff for any sequence
$(x_n)\subseteq A$, if $x_n$ converges to some $x$ in $X$, then $x\in
A$.  Thus, we can rephrase the above proposition as follows.

\begin{prop} Let $A\subseteq X_T$.  Then $A=X_{T'}$ for some extension
  $T'$ of $T$ iff $A$ contains the limit points of all convergent
  sequences in $A$.  \end{prop}


\begin{disc} From the point of view of propositional logic, there is
  little to be gained by allowing theories to correspond to
  \emph{arbitrary} sets of models.  First, if $A$ is an arbitrary set
  of models, then the closure $A^{-}$ of $A$ is characterized by
  axioms, i.e.\ $A^{-}=\mathrm{Mod}(T)$ for some theory $T$.  Suppose,
  however, that someone insists that some valuation $v\in
  A^{-}\backslash A$ should not be taken to be a model.  What reason
  could there be for banishing $v$?  For any finite collection $\phi
  _1,\dots ,\phi _n$ of sentences, there is a valuation $w$ in $A$
  such that $v$ and $w$ agree on $\phi _1,\dots ,\phi _n$.  Put even
  more strongly, any property possessed by $v$ is possessed by
  infinitely many of the elements in $A$.
\end{disc}

\begin{disc} Philosophers often say that, ``a proposition is a subset
  of possible worlds.''  But if the suggestion is that every subset of
  possible worlds corresponds to a proposition, then ... 

  The Stone space $X$ of a theory is precisely the set of possible
  worlds according to that theory.  But not every subset of $X$
  corresponds to a proposition.  Indeed, let $S\subseteq X$ be an
  arbitrary subset of $X$.  Then the following are equivalent:
\begin{enumerate}
\item There is a proposition $\phi$ of $\Sigma$ such that $S$ consists
  of all and only those worlds in which $\phi$ is true.
\item $S=[\phi ]$, where $[\phi ]=\{ v\in X:v(\phi )=1 \}$.
\item $S$ is a basic open subset of $X$. 
\end{enumerate}

Of course, there are theories $T$ such that the Stone space $X_T$ is
discrete --- i.e.\ where every subset $S$ of $X_T$ is open.  For
example, if $\Sigma$ is finite, then for any theory $T$ in $\Sigma$,
the Stone space $X_T$ is discrete.  But what about the case where
$\Sigma$ is infinite?  Let's first note that there are Stone spaces
that are not discrete.  For example, let $\Sigma = \{ p_0,p_1,\dots
\}$, and let $T$ be the empty theory in $\Sigma$.  We already saw that
$T$ has no atoms; and since $T$ has no atoms, no singleton subsets of
$X_T$ are open.  Therefore, $X_T$ is not discrete.  

This result is not surprising.  Consider a possible world $v$ in
$X_T$, and let's think about how we could use a proposition $\phi$ to
uniquely characterize $v$.  Let $P_v$ be the list of literals that are
true in $v$.  Obviously, $P_v$ contains infinitely many logically
independent sentences.  But the language $\Sigma$ does not permit
infinite conjunctions, and so there is no single proposition $\phi$
that entails all elements of $P_v$. 

These considerations raise a question: is there a Stone space $X$ that
is both infinite and discrete?  The answer is No, for the following
reason.  Stone spaces are compact Hausdorff spaces.  And if a compact
topological space is discrete, then it's finite.  (Recall that a space
$X$ is compact just in case every open cover has a finite subcover.
If for every $x\in X$, the set $\{ x\}$ is open, then $X$ is finite.)

But couldn't a philosopher now say that this result --- that not every
subset of possible worlds corresponds to a proposition --- is an
artifact of the formalism?  In fact, couldn't we just start with the
space $X$ of possible worlds, and declare by fiat that every subset of
$X$ is a proposition?  After all, we can take intersection of subsets
as corresponding to conjunction, union of subsets as corresponding to
disjunction, etc..  

Supposing that $X$ is infinite, it's easy to see that the resulting
logic would not be compact.  In particular, for each $x\in X$, there
is a descending sequence $P_i$ of infinite subsets of $X$ such that
$\bigcap _iP_i=\{ x\}$.  If we let $P_\infty = \{ x\}$, then the
infinite family $P_1,P_2,\dots $ implies $P_\infty$.  But no finite
subfamily of $P_1,P_2,\dots $ implies $P_\infty$.
\end{disc}



\begin{disc}[Limits of theories] It's not uncommon to hear talk about
  ``limiting relations'' between theories.  For example, Newtonian
  physics is sometimes said to be the limit of Einstein's theory of
  relativity when the speed of light $c$ is allowed to go to infinity.
  Similarly, Newtonian physics is sometimes said to be the limit of
  quantum mechanics when Planck's constant $\hbar$ is allowed to go to
  zero.

  One way we can think of these claims is as saying that there is a
  sequence of theories $T_c$, where $c$ is some numerical parameter,
  and that in some sense there is a theory $T_\infty$ such that
  $T_\infty=\lim _{c\to\infty}T_c$.  But how in the world are we
  supposed to make sense of the $\lim$ symbol on the right hand side?
  Is there some sort of topology on the space of theories?

  There's another way to think about limiting relations between
  theories.  Let me explain by means of an example.  For $n=1,2,\dots
  $, let $\Sigma _n=\{ p_1,p_2,\dots ,p_n \}$, and let $T_n$ be the
  empty theory in $\Sigma _n$.  Let $\Sigma _\infty =\{ p_1,p_2,\dots
  \}$, and let $T_\infty$ be the empty theory in $\Sigma _\infty$.
  When $i<j$, there is a conservative translation $F_{ij}:T_i\to T_j$,
  and $F_{ik}=F_{jk}\circ F_{ij}$.  Moreover, while none of the
  $F_{i\infty}$ are essentially surjective, the entire collection $\{
  F_{i\infty }:i<\infty \}$ is essentially surjective in the sense
  that for any sentence $\phi$ of $\Sigma _{\infty}$, there is a $k\in
  \mathbb{N}$ and a sentence $\psi$ of $\Sigma _k$ such that
  $F_{k\infty}(\psi )=\phi$. 

  When we get to the chapter on category theory, we will see that this
  example is a special case of a general construction called a
  \textbf{colimit}.  That is, the infinite theory $T_\infty$ is a
  colimit of the ascending sequence $T_1,T_2,\dots $ of finite
  theories.  In more intuitive language, the sequence $T_1,T_2,\dots $
  successively approximates the infinite theory $T_\infty$.

  It's also illuminating to consider the dual construction on the
  Stone spaces of the theories $T_1,T_2,\dots T_\infty$.  For each
  $i$, let $X_i$ be the Stone space of $T_i$.  Let $f_{ji}:X_j\to X_i$
  be the dual of $F_{ij}:T_i\to T_j$.  Then we have $f_{ki}=\circ
  f_{ji}\circ f_{kj}$.  Moreoever, while none of the individual maps
  $f_{\infty i}$ is injective, the collection $\{ f_{\infty
    i}:i<\infty \}$ is jointly injective in the sense that for any
  $x,y\in X_\infty$, if $x\neq y$ then there is a $k$ such that
  $f_{\infty k}(x)\neq f_{\infty k}(y)$.  In this case, the Stone
  space $X_{\infty}$ is said to be the \textbf{limit}, or
  \textbf{projective limit}, of the sequence $X_1,X_2,\dots $.

  What we see here is that $X_\infty$ is a rather special topological
  space: it is a limit of finite spaces.  Such spaces are called
  \textbf{profinite}.  In fact, for every propositional theory $T$,
  the Stone space $X_T$ is profinite.  In other words, and not
  surprisingly, every propositional theory $T$ can be approximated by
  finite propositional theories.  But we won't prove that fact here.
\end{disc}

\begin{prop} Let $X$ be a $T_1$ space.  If $X$ is finite, then $X$ is
  discrete. \label{finite-discrete} \end{prop}

\begin{proof} Since $X$ is finite we may write $X=\{ x_0,\dots
  ,x_n\}$.  It will suffice to show that that $\{ x_0\}$ is open.
  Since $X$ is $T_1$, for each $i>0$, there is a neighborhood $U_i$ of
  $x_0$ such that $x_i\not\in U_i$.  But then $U_1\cap \cdots \cap
  U_n$ is an neighborhood of $x_0$ that does not contain any of
  $x_1,\dots ,x_n$.  That is, $U_1\cap \cdots\cap U_n=\{ x_0\}$, and
  $X$ is discrete.
\end{proof}


\begin{disc}[Finite propositional theories] The results we've
  established provide a complete classification of propositional
  theories with a finite number of models.  Indeed, for each natural
  number $n$, there is only one compact Hausdorff space with $n$
  eleements, namely the discrete space (see Proposition
  \ref{finite-discrete}).  Since the topological space of models
  completely determines the theory $T$ (up to equivalence), it follows
  that there is only on theory $T$ (up to equivalence) with $n$
  models.  In certain cases, we can be even more concrete.  For
  example, when $n=2^m$ for some $m$, then $T$ is equivalent to the
  empty theory in a signature $\Sigma _m =\{ p_1,\dots ,p_m\}$.  And
  it's easy to see how to add axioms to these empty theories in order
  to decrease the number of models.  For example, let $T'=\{ p_1\}$ in
  the signature $\Sigma _m$.  Then $T'$ has $2^m-1$ models.
  Similarly, $T''=\{ p_1,p_2 \}$ has $2^m-2$ models.

  It's also easy to see now that there are many embeddings between
  finite propositional theories.  For example, suppose that $T$ is a
  theory with $m$ models, and $T'$ is a theory with $n$ models where
  $m\leq n$.  Obviously there is a surjection from $X_{T'}$ onto
  $X_{T}$, and this surjection induces a conservative translation
  $F:T\to T'$.  In particular, for any propositional theory $T$ with a
  finite number $m$ of models, there is an $n$ such that $m<2^n$, and
  thus there is a conservative translation $F:T\to T'$, where $T'$ is
  the empty theory in a signature with $n$ elements.  In other words,
  every theory with a finite number of models can be embedded in a
  theory that has no axioms.  In simple cases, we can illustrate this
  point more concretely.  Let $\Sigma =\{ p_1,p_2 \}$, and let $T = \{
  p_1\to p_2\}$.  Let $\Sigma '=\{ q_1,q_2 \}$, and let $T'$ be the
  empty theory in $\Sigma '$.  Consider the translation $F:T\to T'$
  that takes $p_1$ to $q_1\wedge q_2$, and that takes $p_2$ to $q_2$.
  [[TO DO: Verify that it's conservative.]]

  This stratagem --- of reducing a theory with non-logical axioms to
  another one without non-logical axioms --- was employed by Nelson
  Goodman and W.v.O. Quine in a paper, ``Elimination of extra-logical
  postulates'' (1940), and then again by Quine in, ``Implicit
  definition sustained'' (1964).  They claim that the technical fact
  has some important philosophical implications.  We will discuss this
  issue at greater length in Chapter ??.

  We'll see later that things become more complicated with predicate
  logic theories.  It's certainly possible to have distinct (i.e.\
  inequivalent) predicate logic theories with the same finite number
  of models.  In fact, there are distinct predicate logic theories
  that both have a single model.  For example, the theory that says
  there is exactly one thing has one model, up to isomorphism.  The
  theory that says that are exactly two things has one model, up to
  isomorphism.  But those two theories are definitely not equivalent,
  even by very liberal standards.  In particular, the models of the
  two theories have different (i.e.\ non-isomorphic) automorphism
  groups.
\end{disc}


%% NB. countable atomless also representable as the interval algebra
%% of Q+

%% claim: closed <==> compact <==> sequentially closed



\subsection{Adding new vocabulary}

It's a common mistake, when thinking about future science, to imagine
that future theories will reveal new facts, but these facts will be
couched in the language of our current theories.  But if the past is
any guide, we should expect something quite different.  We should
expect that future science will involve concepts which we have not yet
imagined.

The same goes for our idealized models of scientific theories.  It's
tempting to hold the signature $\Sigma$ fixed, and to consider
relations between theories which are all formulated in $\Sigma$.  But
we need to consider changes of signature as well --- or as Quine would
put it, changes in ideology.

One of the most obvious theoretical changes is simply adding new
vocabulary, without making any other changes.  In other words, we add
new vocabulary, but don't assert anything (other than tautologies) in
this new vocabulary.  Of course, that kind of move never happens in
real-world science.  After all, it's just an idealization!

Represented in terms of symbolic logic, this kind of theoretical
change involves passing from the original signature $\Sigma$ to an
expanded signature $\Sigma '$ that contains $\Sigma$.  And what is the
new theory?  The new theory consists of literally the same sentences
as the original theory.  However, despite the fact that $T=T'$, they
really aren't the same theory --- because the language of $T$ is
$\Sigma$, and the language of $T'$ is $\Sigma '$.  (Recall that
theories should really be considered as pairs consisting of signature
and sentences in that signature.  Thus, what we really have here are
the theories $\langle \Sigma ,T\rangle$ and $\langle \Sigma
',T\rangle$.)

Notice, in particular, that the original theory $\langle \Sigma
,T\rangle$, and the new theory $\langle \Sigma ',T\rangle$ might have
quite different properties.  For example, let $\Sigma = \{ p \}$, and
let $T$ be the theory that says ``$p$''.  Let $\Sigma '=\{ p,q\}$.
Then the original theory $\langle \Sigma ,T\rangle$ is complete.  But
the new theory $\langle \Sigma ',T\rangle$ is not complete.

In the sequel, we will often suppress the signature, and write $T'$
for $\langle \Sigma ',T\rangle$.  But please keep clearly in mind that
for $T$ and $T'$ there are two notions of equality.  There is the
weaker notion, namely that they consist of the same sentences.  And
there is stronger notion, namely that they consists of the same
sentences in the same signature.

In general, when $T'$ extends $T$ by the addition of vocabulary alone,
then there is an obvious embedding $I:\Sigma\to\Sigma '$.  Since $T'$
contains no new truths beyond what $T$ contains, $I:T\to T'$ is a
translation of $T$ into $T'$.  Obviously, $I:T\to T'$ is
\textbf{conservative} in the sense that if $T'\vdash I(\phi )$ then
$T\vdash \phi$.  We also claim that the dual map
$I^*:\mathrm{Mod}(T')\to \mathrm{Mod}(T)$ is surjective, i.e.\ for
each model $v$ of $T$, there is a model $w$ of $T'$ such that
$I^*(w)=v$.  In fact, $I^*$ is simply the restriction mapping: it
takes a valuation $w$ of $\Sigma '$, and returns the restricted
valuation $w|_{\Sigma}$.  Thus, to say that $I^*$ is surjective is
tantamount to saying that every model $v$ of $T$ can be extended to a
model $v'$ of $T'$.  Now, the truth of that may not be completely
obvious.  However, for the case we are dealing with, it amounts to no
more than the claim that assignments of values to elements of $\Sigma
'\backslash \Sigma$ are independent of assignments of values to
elements of $\Sigma$.  And that is obviously true.

These considerations raise a question.  Suppose that $F:T\to T'$ is an
arbitrary conservative translation (where we are no longer assuming
that $\Sigma \subseteq\Sigma '$).  Does it then follow that $T'$ is
equivalent to some extension of $T$ by means of adding new vocabulary?
The answer is: Not Necessarily.  Consider the following simple
example.

\begin{example} Let $\Sigma _0= \{ p \}$, and let $T _0$ be the empty
  theory in $\Sigma$.  Let $\Sigma =\{ p,q\}$, and let $T$ be the
  theory with axiom $p\vee q$.  We claim that the inclusion $\Sigma
  _0\subseteq\Sigma$ gives rise to a conservative translation
  $F:T_0\to T$.  Indeed, suppose that $T\vDash \phi$, where $\phi$
  contains only $p$.  That is, $p\vee q\vDash \phi$.  But then $\phi$
  is a tautology, which means that $T_0\vDash \phi$.  Therefore
  $F:T_0\to T$ is conservative, and $T_0$ is a sub-theory of $T$.

  However, $T$ has $3$ models, and any extension of $T_0$ by new
  vocabulary has $2^n$ models, for some $n$.  Therefore, $T$ is not
  equivalent to an extension of $T_0$ by new vocabulary.
\end{example}


TO DO: Theoretical terms.  Question: should the theoretical stuff be
conservative over the observational stuff?

\begin{disc}[Sub-theories] Let $\Sigma$ be a signature, and let $T$
  and $T_0$ be theories in $\Sigma$.  We say that $T_0$ is a
  \textbf{sub-theory} of $T$ just in case there is a conservative
  translation $F:T_0\to T$.  Question: $T_0$ is a sub-theory of $T$,
  then does it follow that there is a subset $\Sigma _0$ of $\Sigma$
  such that $F$ is an equivalence between $T_0$ and $T|_{\Sigma _0}$?
  (In other words, not every sub-theory can be obtained by a
  restriction of vocabulary.  I need better examples of this!!)

  I don't think so.  Let $\Sigma = \{ p_1,p_2\}$, and let $T$ be the
  empty theory in $\Sigma$.  Let $T_0$ be the empty theory in $\Sigma
  _0=\{ q\}$.  Now define $F(q)=p_1\vee p_2$, in which case $F(\neg
  q)=\neg p_1\wedge \neg p_2$.

  Proof: Let $\Sigma _0$ be the elements of the signature



  Claim: if $T$ is conservative over $T_0$, then $T_0$ is logically
  equivalent to $T|_{\Sigma _0}$.  Suppose first that $T_0\vDash
  \phi$.  Since $T$ extends $\phi$, it follows that $T\vDash \phi$.
  Suppose now that $T\vDash \phi$.  Since $T$ is conservative over
  $T_0$, it follows that $T_0\vDash\phi$.






\end{disc}



\section{Putnam's paradox}


%% claim: If $\Sigma$ is infinite, then $Th(w)$ is not finitely
%% axiomatizable.

%% claim: Stone space of countable non-atomic is homogeneous.

Putnam's paradox is usually presented as some sort of consequence of
model theory for predicate logic, and in particular the
Lowenheim-Skolem theorem.  However, Putnam's claims really only depend
on the fact that the languages of first-order logic are uninterpreted.
In fact, the moves that Putnam makes can also be made in propositional
logic.

Suppose that Ted's theory consists of a consistent set $T$ of
sentences.  Suppose moreoever, that the set $T$ can be represented as
sentences of a propositional logic signature $\Sigma$.  Since $T$ is
consistent, there is a model $M$ of $T$.  In a somewhat atypical
fashion --- but nonetheless techically correct --- let's think of $M$
as map that assigns extensions to relation symbols.  The propositional
constants in $\Sigma$ are $0$-ary, an so for each $p\in \Sigma$,
$M(p)$ is the bottom subobject of $1$, namely the empty set $0$, or
$M(p)$ is the top subobject of $1$, namely $1$ itself.

Now, the WORLD is a set $W$ of objects.  Define a function $f:1\to W$
by letting $f(*)$ be any element of $W$.  

\begin{example}[complete theories] Let $\Sigma _1=\{ p\}$, and let
  $T=\{ p\}$.  Let $\Sigma _2=\{ p,q\}$, and let $T_2=\{ p,q\}$.
  Question: are $T_1$ and $T_2$ equivalent theories?  Define $F:\Sigma
  _1\to \Sigma _2$ by $F(p)=p$.  I claim that $F$ is essentially
  surjective.


  ... if $T$ is complete, then $S(T)$ is the single point space.
  Therefore any two complete theories are definitionally equivalent.

  If $T$ is consistent, and $T'$ is complete, then there is a monic
  $m:T\to T'$ and an epic $e:T\to T'$.  

  Epic: $S(T')$ has one point.  $S(T)$ has at least one point.  So
  there is an injection $f:S(T')\to S(T)$.  Thus, $f^*:T\to T'$ is
  epic.

  Monic: Let $g:S(T)\to S(T')$ be the map that takes everything to a
  single point.  Then $g^*:T'\to T$ is monic.

  Things look even more simple from the point of view of
  $\mathbf{Bool}$.  Obviously $2$ can be embedded in any non-trivial
  $B$.  And by the BPI, there is a surjection $p:B\to 2$.
\end{example}


\begin{defn} A category $\mathbf{C}$ is said to be \emph{balanced}
  just in case any monic epic morphism is an isomorphism. \end{defn}

\begin{example} The category $\mathbf{Top}$ is not
  balanced. \end{example}


\begin{prop} The category $\mathbf{Bool}$ is balanced. \end{prop}



\begin{prop} $Th(v)\cong Th(w)$ if and only if $v$ and $w$ are
  topologically indistinguishable. \end{prop}

\begin{disc}
  Any consistent theory $T$ can be completed to a theory $T'$.
  (Proof: $T$ consistent means that $v\in [T]$.  Hence $T\subseteq
  Th(v)$, and $Th(v)$ is a conservative extension of $T$.  Now let $w$
  be an arbitrary valuation.  We claim that $Th(w)\cong Th(v)$.  

\end{disc}


\section{Metatheory for propositional logic}

Exercises:
\begin{enumerate}
\item Show that if $A\subseteq B$ then $Cn(A)\subseteq Cn(B)$.
\item Show that $Cn(Cn(A))=Cn(A)$.
\item Show that $Cn(A\cap B)\subseteq Cn(A)\cap Cn(B)$. [hint: use 1]  
\item Show that $Cn(A)\cup Cn(B)\subseteq Cn(A\cup B)$. [hint: use 1]
\end{enumerate}


\section{Category Theory}

Show that $d_{X\otimes Y}= (1_{X\otimes X}\otimes d_Y) \circ (d_X\otimes 1_Y)$

\[ d_{X\times Y}=d_X\times d_Y \]


Suppose that $\mathbf{C}$ is a category in which all finite limits
exist.  Let $f,g:X\to Y$.  Then the equalizer of $f$ and $g$ is
the pullback of $d_Y$ along $\langle f,g\rangle$.



\section{Set theory}

Exercise: Let $f:X\to Y$.  Then $f$ is monic if and only if $f(\neg
E)\subseteq \neg f(E)$ for all subsets $E$ of $X$.

Let $g:X\to Y$.  Then $g$ is epi if and only if $\neg g(B)\subseteq
g(\neg B)$ for all subsets $B$ of $Y$.



Exercise: Show that $f(f^{-1}(U))\subseteq U$.  Give an example where
$f(f^{-1}(U))\neq U$.  

Proof: if $y\in f(f^{-1}(U))$ then $y=f(x)$ with $x\in f^{-1}(U)$.
Thus, $y=f(x)\in U$.  Now let $f:1\to 2$, and let $U=2$.  Then
$f^{-1}(U)=1$, and $f(f^{-1}(2))=f(1)=1$. 





\end{document}