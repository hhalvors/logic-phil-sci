%% http://math.stackexchange.com/questions/359693/overview-of-basic-results-about-images-and-preimages

% \documentclass[11pt]{article}
% \title{The Category of Sets}
% \usepackage{array,multirow,amsthm,amsmath,amssymb,url,eucal}
% \usepackage{bussproofs}
% \usepackage{tikz-cd}
% \usepackage{natbib}
% \newtheorem{prop}{Proposition}[section]
% \newtheorem{lemma}[prop]{Lemma}
% \newtheorem{thm}[prop]{Theorem}
% \newtheorem{cor}[prop]{Corollary}
% \theoremstyle{definition}
% \newtheorem*{defn}{Definition}
% \newtheorem{axiom}{Axiom}
% \newtheorem{exercise}[prop]{Exercise}
% \newtheorem*{exercises}{Exercises}
% \newtheorem*{example}{Example}
% \newtheorem{facts}[prop]{Facts}
% \newtheorem{fact}[prop]{Fact}
% \theoremstyle{remark}
% \newtheorem*{disc}{Discussion}
% \newtheorem*{note}{Note}
% \usepackage{mathrsfs}
% % \usepackage{showlabels}
% \usepackage[framemethod=TikZ]{mdframed}
% \author{Hans Halvorson}
% \date{\today}

% \begin{document}
% \newcommand{\RA}{\vdash}
% \newcommand{\7}{\mathbb}
% \newcommand{\2}{\mathscr}
% \newcommand{\cl}{\overline}
% \renewcommand{\emph}{\textbf}


% \maketitle

%%% TO DO 10/16
%% -- reintroduce Boolean structure Omega
%% -- regular categories -- graph of function -> function AxE!y
%% -- pullback of subobjects (instead of equalizer), comprehension
%% -- Cartesian closed. X^0 = 1 ?

%%% TO DO on prop metatheory
%% -- show f conservative => f^* surjective
%% -- show f equivalence <=> f conservative & f eso
%% -- show f equivalence <=> CDE

%%% TO DO on Boolean algebras
%% -- example: finite-cofinite algebra

% TO DO: 
% \begin{itemize}
% \item complement of a subobject
% \item Boolean operations on $\Omega$.
% \item $\mathrm{Sub}(1)\cong \Omega$
% \item $\Omega$ has four subobjects.
% \item right adjoint to pullback (universal quantifier)
% \item Show that $X\times 2\cong X\times (1+1)\cong X+X$. 
% \item Lattice of subobjects
% \item union of subobjects
% \item Power sets.  Boolean operations on power sets!
% \item Exponential objects
% \item distributive law: Show $(A_0\cup A_1)\cap B=(A_0\cap B)\cup (A_1\cap B)$.
%   Relation to the fact that images exist?  
% \item Sketch construction of rational numbers, real numbers
% \end{itemize}

%% TO DO: define union of subsets

%% TO DO: subsets form a (complete) Boolean algebra

\chapter{The Category of Sets} \label{cat-set}


\section{Introduction}

In the previous chapter we started to reason about theories (in
propositional logic) without explicitly saying anything about the
rules of reasoning that we would be permitted to use.  Now we need to
talk more explicitly about the theory we will use to talk about
theories, i.e.\ our {\it metatheory}.  We want our metatheory $M$ to
be able to describe theories, which we can take in the first instance
to be ``collections of sentences,'' or better, ``structured
collections of sentences.''\footnote{I don't mean to be begging the
  question here about what a theory is.  We could just as well think
  of a theory as a structured collection of models.  And just as
  sentences can be broken down into smaller components until we reach
  undefined primitives, so models can be broken down into smaller
  components until we reach undefined primitives.  In both cases,
  metatheory bottoms out in undefined primitives.  We can call these
  primitives ``symbols,'' or ``sets,'' or anything else we want.  But
  the name we choose doesn't affect the inferences we're permitted to
  draw.}  What's more, sentences themselves are structured collections
of symbols.  Fortunately, we won't need to press the inquiry further
into the question of the nature of symbols.  It will suffice to assume
that there are enough symbols, and that there is some primitive notion
of identity of symbols.  For example, I assume that you understand
that ``$p$'' is the same symbol as ``$p$'', and is different from
``$q$''.

Fortunately, there is a theory of collections of things lying close to
hand, namely ``the theory of sets.''  At the beginning of the 20th
century, much effort was given to clarifying the theory of sets, since
it was intended to serve as a foundation for all of mathematics.
Amazingly, the theory of sets can be formalized in first-order logic
with only one non-logical symbol, viz. a binary relation symbol
``$\in$''.  In the resulting first-order theory --- usually called
Zermelo-Frankel set theory --- the quantifiers can be thought of as
ranging over sets, and the relation symbol $\in$ can be used to define
further notions such as subset, Cartesian products of sets, functions
from one set to another, etc..

Set theory can be presented informally (sometimes called ``naive set
theory''), or formally (``axiomatic set theory'').  In both cases, the
relation $\in$ is primitive.  However, we're going to approach things
from a different angle.  We're not concerned as much with what sets
{\it are}, but with what we can {\it do} with them.  Thus, I'll
present a version of ETCS, the elementary theory of the category of
sets.\footnote{All credit to William Lawvere for introducing this
  approach to set theory.  For a gentle introduction, see
  \citep{lawvere-sets}.}  Here ``elementary theory'' indicates that
this theory can be formalized in elementary (i.e.\ first-order) logic.
The phrase ``category of sets'' indicates that this theory treats the
collection of sets as a structured object --- a category consisting of
sets and functions between them.

\begin{axi}[Sets is a category]{ax:set-cat} $\cat{Sets}$ is a
  \emph{category}, i.e.\ it consists of two kinds of things: objects,
  which we call \textbf{sets}, and arrows, which we call
  \textbf{functions}.  To say that $\cat{Sets}$ is a category means
  that: 
\begin{enumerate}
\item Every function $f$ has a domain set $d_0f$ and a codomain set
  $d_1f$.  We write $f:X\to Y$ to indicate that $X=d_0f$ and $Y=d_1f$.
\item Compatible functions can be composed.  For example, if $f:X\to
  Y$ and $g:Y\to Z$ are functions, then $g\circ f:X\to Z$ is a
  function.  (We frequently abbreviate $g\circ f$ as $gf$.)
\item Composition of functions is associative:
\[ h\circ (g\circ f) \: = \: (h\circ g)\circ f \]
when all these compositions are defined.
\item For each set $X$, there is a function $1_X:X\to X$ that acts as
  a left and right identity relative to composition.
\end{enumerate}
\label{ax:set-cat} \end{axi}

\begin{disc} If our goal was to formalize ETCS rigorously in
  first-order logic, we might use two-sorted logic, with one sort for
  sets, and one sort for functions.  We will introduce the apparatus
  of many-sorted logic in Chapter \ref{chap-second}.  The primitive
  vocabulary of this theory would include symbols $\circ ,d_0,d_1,1$,
  but it would \textit{not} include the symbol $\in$.  In other words,
  containment is \textit{not} a primitive notion of ETCS. \end{disc}

Set theory makes frequent use of bracket notation, such as:
\[ \{ n\in N \mid n>17 \} .\] These symbols should be read as, ``the
set of $n$ in $N$ such that $n>17$.''  Similarly, $\{ x,y\}$ designates
a set consisting of elements $x$ and $y$.  But so far, we have no
rules for reasoning about such sets.  In the following sections, we
will gradually add axioms until it becomes clear which rules of
inference are permitted vis-a-vis sets.  

Suppose for a moment that we understand the bracket notation, and
suppose that $X$ and $Y$ are sets.  Then given an element $x\in X$,
and an element $y\in Y$, we can take the set $\{ x,\{ x,y\}\}$ as an
``ordered pair'' consisting of $x$ and $y$.  The pair is ordered
because $x$ and $y$ play asymmetric roles: the element $x$ occurs by
itself, as well as with the element $y$.  If we could then gather
together these ordered pairs into a single set, we would designate it
by $X\times Y$, which we call the \emph{Cartesian product} of $X$ and
$Y$.  The Cartesian product construction should be familiar from high
school mathematics.  For example, the plane (with $x$ and $y$
coordinates) is the Cartesian product of two copies of the real number
line.

In typical presentations of set theory, the existence of product sets
is derived from other axioms.  Here we will proceed in the opposite
direction: we will take the notion of a product set as primitive.

\bigskip

\begin{axi}[Cartesian products]{ax:set-prod} For any two sets $X$ and
  $Y$, there is a set $X\times Y$, and functions $\pi _0:X\times Y\to
  X$ and $\pi _1:X\times Y\to Y$, such that: for any other set $Z$ and
  functions $f:Z\to X$ and $g:Z\to Y$, there is a unique function
  $\langle f,g\rangle :Z\to X\times Y$ such that $\pi _0\langle
  f,g\rangle = f$ and $\pi _1\langle f,g\rangle =g$.
 \label{ax:set-prod} \end{axi}

Here the angle brackets $\langle f,g\rangle$ are not intended to
indicate anything about the internal structure of the denoted
function.  This notation is chosen merely to indicate that $\langle
f,g\rangle$ is uniquely determined by $f$ and $g$.

The defining conditions of a product set can be visualized by means of
an arrow diagram.  \[ \begin{tikzcd}
  & Z \arrow[ddl,bend right,"f"'] \arrow[ddr,bend left,"g"] \arrow[dashed,d,"{\langle f,g\rangle}"] \\
  & X\times Y \arrow[dl,"\pi _0"] \arrow[dr,"\pi _1"'] \\
  X & & Y
\end{tikzcd} \] Here each node represents a set, and arrows between
nodes represent functions.  The dashed arrow is meant to indicate that
the axiom asserts the existence of such an arrow (dependent on the
existence of the other arrows in the diagram).

\begin{disc} There is a close analogy between the defining conditions
  of a Cartesian product and the introduction and elimination rules
  for conjunction.  If $\phi\wedge\psi$ is a conjunction, then there
  are arrows (i.e.\ derivations) $\phi\wedge\psi\to \phi$ and
  $\phi\wedge\psi\to \psi$.  That's the $\wedge$ elimination
  rule.\footnote{Here I'm intentionally being ambiguous between the
    relation $\vdash$ and the connective $\to$.  The analogy is more
    clear if we use the symbol $\to$ the latter.}  Moreover, for any
  sentence $\theta$, if there are derivations $\theta\to\phi$ and
  $\theta\to\psi$, then there is a unique derivation
  $\theta\to\phi\wedge \psi$.  That's the $\wedge$ introduction rule.
\end{disc}

\begin{defn} Let $\gamma$ and $\gamma '$ be paths of arrows in a
  diagram that begin and end at the same node.  We say that $\gamma$
  and $\gamma '$ \emph{commute} just in case the composition of the
  functions along $\gamma$ is equal to the composition of the
  functions along $\gamma '$.  We say that the diagram as a whole
  \emph{commutes} just in case any two paths between nodes are equal.
  Thus, for example, the product diagram above commutes. \end{defn}

The functions $\pi _0:X\times Y\to X$ and $\pi _1:X\times Y\to Y$ are
typically called \emph{projections} of the product.  What features do
these projections have?  Before we say more on that score, let's pause
to talk about features of functions.

You may have heard before of some properties of functions such as
being one-to-one, or onto, or continuous, etc..  For bare sets, there
is no notion of continuity of functions, per se.  And, with only the
first two axioms in place, we do not yet have the means to define what
it means for a function to be one-to-one or onto.  Indeed, recall that
a function $f:X\to Y$ is typically said to be one-to-one just in case
$f(x)=f(y)$ implies $x=y$ for any two ``points'' $x$ and $y$ of $X$.
But we don't yet have a notion of points!

Nonetheless, there are point-free surrogates for the notions of being
one-to-one and onto.

\begin{defn} A function $f:X\to Y$ is said to be a \emph{monomorphism}
  just in case for any two functions $g,h:Z\rightrightarrows X$, if
  $fg=fh$ then $g=h$.  \end{defn}

\begin{defn} A function $f:X\to Y$ is said to be a \emph{epimorphism}
  just in case for any two functions $g,h:Y\to Z$, if $gf=hf$ then
  $g=h$.  \end{defn}

We will frequently say, ``\dots is monic'' as shorthand for ``\dots is
a monomorphism,'' and ``\dots is epi'' for ``\dots is an
epimorphism.''

\begin{defn} A function $f:X\to Y$ is said to be an \emph{isomorphism}
  just in case there is a function $g:Y\to X$ such that $gf=1_X$ and
  $fg=1_Y$.  If there is an isomorphism $f:X\to Y$, we say that $X$
  and $Y$ are \emph{isomorphic}, and we write $X\cong Y$.  \end{defn}

\begin{exercise} \label{comps} Show the following:
\begin{enumerate}
\item If $gf$ is monic, then $f$ is monic.
\item If $fg$ is epi, then $f$ is epi.
\item If $f$ and $g$ are monic, then $gf$ is monic.
\item If $f$ and $g$ are epi, then $gf$ is epi.
\item If $f$ is an isomorphism, then $f$ is epi and
  monic. \end{enumerate} \end{exercise}

\begin{prop} Suppose that both $(W,\pi _0,\pi _1)$ and
  $(W',\pi '_0,\pi '_1)$ are Cartesian products of $X$ and $Y$.  Then
  there is an isomorphism $f:W\to W'$ such that $\pi '_0f=\pi _0$ and
  $\pi '_1f=\pi _1$. \end{prop}

\begin{proof} Since $(W',\pi '_0,\pi '_1)$ is a Cartesian product of
  $X$ and $Y$, there is a unique function $f:W\to W'$ such that
  $\pi '_0f= \pi _0$ and $\pi '_1f= \pi _1$.  Since
  $(W,\pi _0,\pi _1)$ is also a product of $X$ and $Y$, there is a
  unique function $g:W'\to W$ such that $\pi _0g=\pi '_0$ and
  $\pi _1g=\pi '_1$.  We claim that $f$ and $g$ are inverse to each
  other.  Indeed,
  \[ \pi '_i\circ (f\circ g) \: = \: \pi _i\circ g \: = \: \pi '_i ,\]
  for $i=0,1$.  Thus, by the uniqueness clause in the definition of
  Cartesian products, $f\circ g=1_{W'}$.  A similar argument shows
  that $g\circ f=1_W$.  \end{proof}


%% TO DO: define f\times g

%% TO DO: define diagonal

\newcommand{\dl}{\delta}

\begin{defn} If $X$ is a set, we let $\dl :X\to X\times X$ denote the
  unique arrow $\langle 1_X,1_X\rangle$ given by the definition of
  $X\times X$.  We call $\dl$ the \emph{diagonal} of $X$, or the
  \emph{equality relation} on $X$.  Note that $\dl$ is monic, since
  $\pi _0\dl =1_X$ is monic.  \end{defn}

\begin{defn} Suppose that $f:W\to Y$ and $g:X\to Z$ are functions.
  Consider the following diagram:
\[ \begin{tikzcd}
W \arrow[d,"f"'] & W\times X \arrow[l,"q_0"] \arrow[r,"q_1"']
\arrow[dashed,d,"f\times g"] & X
\arrow[d,"g"] \\
Y               & Y\times Z \arrow[l,"\pi _0"] \arrow[r,"\pi _1"'] & Z 
\end{tikzcd} \]
We let $f\times g=\langle fq_0,gq_1\rangle$ be the unique
function from $W\times X$ to $Y\times Z$ such that 
\[ \pi _0(f\times g) = fq_0,\qquad \pi _1(f\times g)=gq_1 .\] Recall
here that, by the definition of products, a function into $Y\times Z$
is uniquely defined by its compositions with the projections $\pi _0$
and $\pi _1$.
\label{unravel} \end{defn}

\begin{prop} Suppose that $f:A\to B$ and $g:B\to C$ are functions.
  Then $1_X\times (g\circ f)=(1_X\times g)\circ (1_X\times
  f)$. \end{prop}

\begin{proof} Consider the following diagram
\[ \begin{tikzcd} 
X \arrow[d,"1_X"'] &   X\times A \arrow[l] \arrow[r] \arrow[d,"1_X\times f"]
& A \arrow[d,"f"]  \\
X \arrow[d,"1_X"'] &   X\times B \arrow[l] \arrow[r] \arrow[d,"1_X\times g"]
& B \arrow[d,"g"] \\
X   &   X\times C \arrow[l] \arrow[r] & C 
\end{tikzcd} \] where $1_X\times f$ and $1_X\times g$ are constructed
as in Defn \ref{unravel}.  Since the top and bottom squares both
commute, the entire diagram commutes.  But then the composite arrow
$(1_X\times g)\circ (1_X\times f)$ satisfies the defining properties
of $1_X\times (g\circ f)$. \end{proof}

\begin{exercise} Show that $1_X\times 1_Y=1_{X\times
    Y}$. \end{exercise}

\begin{defn} Let $X$ be a fixed set.  Then $X$ induces two mappings,
  as follows:
\begin{enumerate}
\item A mapping $Y\mapsto X\times Y$ of sets to sets.  
\item A mapping $f\mapsto 1_X\times f$ of functions to functions.
  That is, if $f:Y\to Z$ is a function, then $1_X\times f:X\times Y\to
  X\times Z$ is a function.  \end{enumerate} By the previous results,
the second mapping is compatible with the composition structure on
arrows.  In this case, we call the pair of mappings a \emph{functor}
from $\cat{Sets}$ to $\cat{Sets}$. \end{defn}

\begin{exercise} Suppose that $f:X\to Y$ is a function.  Show that the
  following diagram commutes.
  \[ \begin{tikzcd}
    X \arrow[r,"f"] \arrow[d,"\dl _X"']  & Y \arrow[d,"\dl _Y"] \\
    X\times X \arrow[r,"f\times f"'] & Y\times
    Y \end{tikzcd} \] \end{exercise}


We will now recover the idea that sets consist of points by requiring
the existence of a single-point set $1$, which plays the privileged
role of determining identity of functions.

\begin{axi}[Terminal Object]{ax:terminal} There is a set $1$
  with the following two features:
\begin{enumerate}
\item For any set $X$, there is a unique function
  \[ \begin{tikzcd} X \arrow[r,"\beta _X"] & 1 \end{tikzcd} \] In this
  case, we say that $1$ is a \emph{terminal object} for $\cat{Sets}$.
\item For any sets $X$ and $Y$, and functions $f,g:X\rightrightarrows
  Y$, if $f\circ x=g\circ x$ for all functions $x:1\to X$, then $f=g$.
  In this case, we say that $1$ is a \emph{separator} for
  $\cat{Sets}$. \end{enumerate} \label{ax:terminal} \end{axi}

%%
%% introduce elements 
%%

The reader may wish to note that for a general category, a
\emph{terminal object} is required only to have the first of the two
properties above.  So, we are not merely requiring that $\cat{Sets}$
has a terminal object; we are requiring that it has a terminal object
that also serves as a separator for functions.

\begin{exercise} Show that if $X$ and $Y$ are terminal objects in a
  category, then $X\cong Y$.  \end{exercise}

\begin{defn} We write $x\in X$ to indicate that $x:1\to X$ is a
  function, and we say that $x$ is an \emph{element} of $X$.  We say
  that $X$ is \emph{non-empty} just in case it has at least one
  element.  If $f:X\to Y$ is a function, we sometimes write $f(x)$ for
  $f\circ x$.  With this notation, the statement that $1$ is a
  separator says: $f=g$ if and only if $f(x)=g(x)$, for all $x\in X$.
\end{defn}

\begin{disc} In ZF set theory, equality between functions is
  completely determined by equality between sets.  Indeed, in ZF,
  functions $f,g:X\rightrightarrows Y$ are defined to be certain
  subsets of $X\times Y$; and subsets of $X\times Y$ are defined to be
  equal just in case they contain the same elements.  In the ETCS
  approach to set theory, equality between functions is primitive, and
  Axiom \ref{ax:terminal} stipulates that this equality can be
  detected by checking elements. 

  Some might see this difference as arguing in favor of ZF: it is more
  parsimonious, because it derives $f=g$ from something more
  fundamental.  However, the defender of ETCS might claim in reply
  that her theory defines $x\in y$ from something more fundamental.
  Which is {\it really} more fundamental, equality between arrows
  (functions), or containment of objects (sets)?  We'll leave that for
  other philosophers to think about. \end{disc}

\begin{exercise} Show that any function $x:1\to X$ is
  monic. \end{exercise}

\begin{prop} A set $X$ has exactly one element if and only if $X\cong
  1$. \end{prop}

\begin{proof} The terminal object $1$ has exactly one element, since
  there is a unique function $1\to 1$.

  Suppose now that $X$ has exactly one element $x:1\to X$.  We will
  show that $X$ is a terminal object.  First, for any set $Y$, there
  is a function $x\circ \beta _Y$ from $Y$ to $X$.  Now suppose that
  $f,g$ are functions from $Y$ to $X$ such that $f\neq g$.  By Axiom
  \ref{ax:terminal}, there is an element $y\in Y$ such that $fy\neq
  gy$.  But then $X$ has more than one element, a contradiction.
  Therefore, there is a unique function from $Y$ to $X$, and $X$ is a
  terminal object.  \end{proof}

\begin{prop} In any category with a terminal object $1$, any object
  $X$ is itself a Cartesian product of $X$ and $1$.  \end{prop}

\begin{proof} We have the obvious projections $\pi _0=1_X:X\to X$ and
  $\pi _1=\beta _X:X\to 1$.  Now let $Y$ be an object, and let
  $f:Y\to X$ and $g:Y\to 1$ be arrows.  We claim that $f:Y\to X$ is
  the unique arrow such that $1_Xf=f$ and $\beta _Xf=g$.  To see that
  $f$ satisfies this condition, note that $g:Y\to 1$ must be
  $\beta _Y$, the unique arrow from $Y$ to the terminal object.  If
  $h$ is another arrow that satisfies this condition, then
  $h=1_Xh=f$. \end{proof}

\begin{prop} Let $a$ and $b$ be elements of $X\times Y$.  Then $a=b$
  if and only if $\pi _0(a)=\pi _0(b)$ and $\pi _1(a)=\pi
  _1(b)$. \end{prop}

\begin{proof} Suppose that $\pi _0(a)=\pi _0(b)$ and $\pi _1(a)=\pi
  _1(b)$.  By the uniqueness property of the product, there is a
  unique function $c:1\to X\times Y$ such that $\pi _0(c)=\pi _0(a)$
  and $\pi _1(c)=\pi _1(a)$.  Since $a$ and $b$ both satisfy this
  property, $a=b$.
\end{proof}

\begin{note} The previous proposition justifies the use of the
  notation
  \[ X\times Y \: = \: \{ \langle x,y\rangle \mid x\in X,y\in Y \} .\]
  Here the identity condition for ordered pairs is given by
  \[ \langle x,y\rangle = \langle x',y'\rangle \qquad \text{iff}
  \qquad x=x'\;\text{and}\; y=y' .\] \end{note}

\begin{prop} Let $(X\times Y,\pi _0,\pi _1)$ be the Cartesian product
  of $X$ and $Y$.  If $Y$ is non-empty, then $\pi _0$ is an
  epimorphism.  \label{proj-epi} \end{prop}

\begin{proof} Suppose that $Y$ is non-empty, and that $y:1\to Y$ is an
  element.  Let $\beta _X:X\to 1$ be the unique map, and let $f=y\circ
  \beta _X$.  Then $\langle 1_X,f\rangle :X\to X\times Y$ such that
  $\pi _0\langle 1_X,f\rangle =1_X$.  Since $1_X$ is epi, $\pi _0$ is
  epi. \end{proof}





\begin{defn} We say that $f:X\to Y$ is \emph{injective} just in case:
  for any $x,y\in X$ if $f(x)=f(y)$, then $x=y$.  Written more
  formally:
  \[ \forall x\forall y[f(x)=f(y)\to x=y ] \] \end{defn}

\begin{note} ``Injective'' is synonymous with
  ``one-to-one''. \end{note}

\begin{exercise} Let $f:X\to Y$ be a function.  Show that if $f$ is
  monic, then $f$ is injective. \end{exercise}

\begin{prop} Let $f:X\to Y$ be a function.  If $f$ is injective then
  $f$ is monic. \label{inj-mon} \end{prop}

\begin{proof} Suppose that $f$ is injective, and let $g,h:A\to X$ be
  functions such that $f\circ g=f\circ h$.  Then for any $a\in A$, we
  have $f(g(a))=f(h(a))$.  Since $f$ is injective, $g(a)=h(a)$.  Since
  $a$ was an arbitrary element of $A$, Axiom \ref{ax:terminal} entails
  that $g=h$.  Therefore, $f$ is monic. \end{proof}

\begin{defn} Let $f:X\to Y$ be a function.  We say that $f$ is
  \emph{surjective} just in case: for each $y\in Y$, there is an $x\in
  X$ such that $f(x)=y$.  Written formally:
\[ \forall y\exists x[f(x)=y] \]
And in diagrammatic form:
\[ \begin{tikzcd}
  & 1 \arrow[dashed,dl,"x"'] \arrow[dr,"y"] \\
X \arrow[rr,"f"] & & Y \end{tikzcd} \]
\end{defn}

\begin{note} ``Surjective'' is synonymous with ``onto''. \end{note}

\begin{exercise} Show that if $f:X\to Y$ is surjective then $f$ is an
  epimorphism. \end{exercise}

We will eventually establish that all epimorphisms are surjective.
However, first we need a couple more axioms.  Given a set $X$, and
some definable condition $\phi$ on $X$, we would like to be able to
construct a subset consisting of those elements in $X$ that satisfy
$\phi$.  The usual notation here is $\{ x\in X\mid \phi (x) \}$, which
we read as, ``the $x$ in $X$ such that $\phi (x)$.''  But the
important question is: which features $\phi$ do we allow?  As an
example of a definable condition $\phi$, consider the condition of,
``having the same value under the functions $f$ and $g$,'' that is,
$\phi (x)$ just in case $f(x)=g(x)$.  We call the subset $\{ x\in
X\mid f(x)=g(x) \}$ the \emph{equalizer} of $f$ and $g$.

\begin{axi}[Equalizers]{ax:set-eq} Suppose that
  $f,g:X\rightrightarrows Y$ are functions.  Then there is a set $E$
  and a function $m:E\to X$ with the following property: $fm=gm$, and
  for any other set $F$ and function $h:F\to X$, if $fh=gh$, then
  there is a unique function $k:F\to E$ such that $mk=h$.
\[ \begin{tikzcd}
E \arrow[r,"m"] & X \arrow[shift left,r,"f"] \arrow[shift right,r,"g"']
& Y \\
F \arrow[dashed,u,"k"] \arrow[ur,"h"'] 
\end{tikzcd} \] We call $(E,m)$ an \emph{equalizer} of $f$ and $g$.
If we don't need to mention the object $E$, we will call the arrow $m$
the equalizer of $f$ and $g$.  \label{ax:set-eq} \end{axi}

\begin{exercise} Suppose that $(E,m)$ and $(E',m')$ are both
  equalizers of $f$ and $g$.  Show that there is an isomorphism
  $k:E\to E'$.  \end{exercise}

\begin{defn} Let $A,B,C$ be sets, and let $f:A\to C$ and $g:B\to C$ be
  functions.  We say that $g$ \emph{factors through} $f$ just in case
  there is a function $h:B\to A$ such that $fh=g$. \end{defn}

\begin{exercise} Let $f,g:X\rightrightarrows Y$, and let $m:E\to X$ be
  the equalizer of $f$ and $g$.  Let $x\in X$.  Show that $x$ factors
  through $m$ if and only if $f(x)=g(x)$. \end{exercise}

\begin{prop} In any category, if $(E,m)$ is the equalizer of $f$ and
  $g$.  Then $m$ is a monomorphism. \end{prop}

\begin{proof} Let $x,y:Z\to E$ such that $mx=my$.  Since $fmx=gmx$,
  there is a unique arrow $z:Z\to E$ such that $mz=mx$.  Since both
  $mx=mx$ and $my=mx$, it follows that $x=y$.  Therefore, $m$ is
  monic. \end{proof}

\begin{defn} Let $f:X\to Y$ be a function.  We say that $f$ is a
  \emph{regular monomorphism} just in case $f$ is the equalizer (up to
  isomorphism) of a pair of arrows $g,h:Y\rightrightarrows
  Z$. \end{defn}

\begin{exercise} Show that if $f$ is an epimorphism and a regular
  monomorphism, then $f$ is an
  isomorphism. \label{herman} \end{exercise}


%% Can now construct pullbacks and preimages

In other approaches to set theory, one uses $\in$ to define a relation
of inclusion between sets:
\[ X\subseteq Y \: \Longleftrightarrow \: \forall x(x\in X\to x\in Y )
.\] We cannot define this exact notion in our approach since, for us,
elements are attached to some particular set.  However, for typical
applications, every set under consideration will come equipped with a
canonical monomorphism $m:X\to U$, where $U$ is some fixed set.  Thus,
it will suffice to consider a relativized notion.

\begin{defn} A \emph{subobject} or \emph{subset} of a set $X$ is a set
  $B$ and a monomorphism $m:B\to X$, called the \emph{inclusion} of
  $B$ in $X$.  Given two subsets $m:B\to X$ and $n:A\to X$, we say
  that $B$ is a subset of $A$ (relative to $X$), written $B\subseteq
  _X A$ just in case there is a function $k:B\to A$ such that $nk=m$.
  When no confusion can result, we omit $X$ and write $B\subseteq
  A$.  \end{defn}

Let $m:B\to Y$ be monic, and let $f:X\to Y$.  Consider the following
diagram:
  \[ \begin{tikzcd} f^{-1}(B) \arrow{r}{k} & X\times B \arrow[shift
    left=0.5ex]{r}[above]{fp _0} \arrow[shift
    right=0.5ex]{r}[below]{mp _1} & Y \end{tikzcd} \] where
  $f^{-1}(B)$ is defined as the equalizer of $f\pi _0$ and $mp_1$.
  Intuitively, we have
\[ \begin{array}{l l l}
f^{-1}(B) & = &  \{ \langle x,y\rangle \in X\times B\mid f(x)=y \} \\
& = & \{ \langle x,y\rangle \in X\times Y \mid f(x)=y \; \text{and} \; y\in B \} \\
& = & \{ x\in X \mid f(x)\in B \} .\end{array} \]
Now we verify that $f^{-1}(B)$ is a subset of $X$.

\begin{prop} The function $p_0k:f^{-1}(B)\to X$ is monic. \end{prop}

\begin{proof} To simplify notation, let $E=f^{-1}(B)$.  Let $x,y:Z\to
  E$ such that $p_0kx=p_0ky$.  Then $fp_0kx=fp_0ky$, and hence
  $mp_1kx=mp_1ky$.  Since $m$ is monic, $p_1kx=p_1ky$.  Thus, $kx=ky$.
  (The identity of a function into $X\times B$ is determined by the
  identity of its projections onto $X$ and $B$.)  Since $k$ is monic,
  $x=y$.  Therefore, $p_0k$ is monic.
\end{proof}

\begin{defn} Let $m:B\to X$ be a subobject, and let $x:1\to X$.  We
  say that $x\in B$ just in case $x$ factors through $m$ as follows:
\[ \begin{tikzcd} 
                                  & B \arrow{d}{m} \\
1 \arrow{r}{x} \arrow[dashed]{ur} &
X              \end{tikzcd} \] \end{defn}

\begin{prop} Let $A\subseteq B\subseteq X$.  If $x\in A$ then $x\in
  B$. \end{prop}

\begin{proof}
\[ \begin{tikzcd}
               & A \arrow{d} \arrow{r}{k} & B \arrow{d} \\
1 \arrow[r,"x"'] \arrow[dashed] {ur} & X \arrow[r,"1_X"']         & X \end{tikzcd} \] 
\end{proof}

Recall that $x\in f^{-1}(B)$ means: $x:1\to X$ factors
  through the inclusion of $f^{-1}(B)$ in $X$.  Consider the following
  diagram:
\[ \begin{tikzcd}
  1 \arrow{ddr}[left]{x} \arrow[dashed]{dr} \\
  & f^{-1}(B) \arrow{r}{p} \arrow{d}{m^*} & B \arrow{d}[right]{m} \\
  & X \arrow{r}[below]{f} & Y \end{tikzcd} \] First look just at the
lower-right square.  This square commutes, in the sense that following
the arrows from $f^{-1}(B)$ clockwise gives the same answer as
following the arrows from $f^{-1}(B)$ counterclockwise.  The square
has another property: for any set $Z$, and functions $g:Z\to X$ and
$h:Z\to B$, there is a unique function $k:Z\to f^{-1}(B)$ such that
$m^{*}k=g$ and $pk=h$.  [To understand better, draw a picture!]  When
a commuting square has this property, then it's said to be a
\emph{pullback}.

\begin{prop} Let $f:X\to Y$, and let $B\subseteq Y$.  Then $x\in
  f^{-1}(B)$ if and only if $f(x)\in B$.  \end{prop}

\begin{proof} If $x\in f^{-1}(B)$, then there is an arrow
  $\hat{x}:1\to f^{-1}(B)$ such that $m^*\hat{x}=x$.  Thus,
  $fx=mp\hat{x}$, which entails that the element $f(x)\in Y$ factors
  through $B$, i.e.\ $f(x)\in B$.  Conversely, if $f(x)\in B$, then
  since the square is a pullback, $x:1\to X$ factors through
  $f^{-1}(B)$, i.e.\ $x\in f^{-1}(B)$. \end{proof}

\begin{defn} Given functions $f:X\to Z$ and $g:Y\to Z$, we define
  \[ X\times _ZY \: = \: \{ \langle x,y\rangle \in X\times Y \mid
  f(x)=g(y) \} .\] In other words, $X\times _ZY$ is the equalizer of
  $f\pi _0$ and $g\pi _1$.  The set $X\times _ZY$, together with the
  functions $\pi _0:X\times _ZY\to X$ and $\pi _1:X\times _ZY\to Y$ is
  called the \emph{pullback} of $f$ and $g$, alternatively the
  \emph{fibered product} of $f$ and $g$.   \end{defn}

The pullback of $f$ and $g$ has the following distinguishing property:
for any set $A$, and functions $h:A\to X$ and $k:A\to Y$ such that
$fh=gk$, there is a unique function $j:A\to X\times _ZY$ such that
$\pi _0j=h$ and $\pi _1j=k$.
\[ \begin{tikzcd}
  A \arrow[ddr,"h"'] \arrow[drr,"k"] \arrow[dr,dashed] \\
  & X\times _ZY \arrow[d,"\pi _0"] \arrow[r,"\pi _1"'] & Y
  \arrow[d,"g"]
  \\
  & X \arrow[r,"f"'] & Z \end{tikzcd} \] The following is an
interesting special case of a pullback.

\begin{defn} Let $f:X\to Y$ be a function.  Then the \emph{kernel
    pair} of $f$ is the pullback $X\times _YX$, with projections
  $p_0:X\times _YX\to X$ and $p_1:X\times _YX\to X$.  Intuitively,
  $X\times _YX$ is the relation, ``having the same image under $f$.''
  Written in terms of braces, 
  \[ X\times _YX \: = \: \{ \langle x,x'\rangle \in X\times X \mid
  f(x)=f(x') \} .\] In particular, $f$ is injective if and only if,
  ``having the same image under $f$'' is coextensive with the equality
  relation on $X$.  That is, $X\times _YX=\{ \langle x,x\rangle \mid
  x\in X \}$, which is the diagonal of $X$. \end{defn}

\begin{exercise} Let $f:X\to Y$ be a function, and let
  $p_0,p_1:X\times _YX\rightrightarrows X$ be the kernel pair of $f$.
  Show that the following are equivalent:
\begin{enumerate}
\item $f$ is a monomorphism.
\item $p_0$ and $p_1$ are isomorphisms.
\item $p_0=p_1$.  \end{enumerate} \label{kkp} \end{exercise}



\section{Truth values and subsets}

\newcommand{\true}{\mathsf{t}} \newcommand{\false}{\mathsf{f}}

\newcommand{\tr}{\mathsf{t}} \newcommand{\f}{\mathsf{f}}

\begin{axi}[Truth-value object]{ax:classify} There is a set $\Omega$
  with the following features:
\begin{enumerate}
\item $\Omega$ has exactly two elements, which we denote by $\true
  :1\to \Omega$ and $\false :1\to \Omega$.
\item For any set $X$, and subobject $m:B\to X$, there is a unique
  function $\ch{B}:X\to \Omega$ such that the following diagram is a
  pullback:
  \[ \begin{tikzcd}
    B \arrow[d, "m"'] \arrow{r} & 1 \arrow{d}{\true} \\
    X \arrow[r, "\ch{B}"'] & \Omega \end{tikzcd} \] In other words,
  $B=\{ x\in X\mid \ch{B}(x)=\tr \}$.
\end{enumerate} \label{ax:classify}
\end{axi}

Intuitively speaking, the first part of Axiom \ref{ax:classify} says
that $\Omega$ is a two-element set, say $\Omega = \{ \f,\tr\}$.  The
second part of Axiom \ref{ax:classify} says that $\Omega$ classifies
the subobjects of a set $X$.  That is, each subobject $m:B\to X$
corresponds to a unique \emph{characteristic function} $\ch{B}:X\to
\{\f,\tr\}$ such that $\ch{B}(x)=\tr$ if and only if $x\in B$.

The terminal object $1$ is a set with one element.  Thus, it should be
the case that $1$ has two subsets, the empty set and $1$ itself.

\begin{prop} The terminal object $1$ has exactly two
  subobjects.  \label{sub-one} \end{prop}

\begin{proof} By Axiom \ref{ax:classify}, subobjects of $1$ correspond
  to functions $1\to\Omega$, that is, to elements of $\Omega$.  By
  Axiom \ref{ax:classify}, $\Omega$ has exactly two elements.
  Therefore, $1$ has exactly two subobjects. \end{proof}

Obviously the function $\tr:1\to\Omega$ corresponds to the subobject
$\mathrm{id}_1:1\to 1$.  Can we say more about the subobject $m:A\to1$
corresponding to the function $\f:1\to \Omega$?  Intuitively, we
should have $A=\{ x\in 1\mid \tr = \f\}$, in other words, the empty
set.  To confirm this intuition, consider the pullback diagram:
\[ \begin{tikzcd}
1 \arrow[dashed,dr,"x"] \\ 
&   A \arrow[d,"m"'] \arrow[r,"k"] & 1 \arrow[d,"\tr"] \\
&   1 \arrow[r,"\f"'] & \Omega
\end{tikzcd} \] Note that $m$ and $k$ must both be the unique function
from $A$ to $1$, that is $m=k=\beta _A$.  Suppose that $A$ is
nonempty, i.e.\ there is a function $x:1\to A$.  Then $\beta _A\circ
x$ is the identity $1\to 1$, and since the square commutes, $\tr =\f$,
a contradiction.  Therefore, $A$ has no elements.

% Conversely, suppose that $A$ is a set with no elements.  Then in the
% diagram above we can take $m=k=\beta _A$, and since $A$ has no
% elements, the diagram trivially commutes.  We claim that it is a
% pullback.  Indeed, for an arbitrary set $X$, there is only one
% function $\beta _X:X\to 1$.  But $\tr\circ \beta _X\neq \f\circ\beta
% _X$.  Thus, the conditions for a pullback a trivially satisifed.  By
% the uniqueness of pullbacks, it follows that there is a unique object
% in $\cat{Sets}$ that has no elements.

\begin{exercise} Show that $\Omega\times\Omega$ has exactly four
  elements.  \end{exercise}

% Axiom \ref{ax:terminal} tells us that functions are individuated by
% their values on elements.  The next proposition tells us that
% subobjects of a set $X$ are also individuated by their elements.

% \begin{prop} Let $m:A\to X$ and $n:B\to X$ be subobjects of $X$.  Then
%   the following are equivalent:
% \begin{enumerate}
% \item $A$ and $B$ are the same subobject; i.e. there is an isomorphism
%   $k:A\to B$ such that $nk=m$.
% \item For all $x\in X$, $x\in A$ if and only if $x\in B$. 
% \item $\ch{A}=\ch{B}$. \end{enumerate} \end{prop}

% \begin{proof} ($1\Rightarrow 2$) Let $k:A\to B$ be an isomorphism such
%   that $m=nk$.  Then $x:1\to X$ factors through $A$ if and only if it
%   factors through $B$.

%   ($2\Rightarrow 3$) Suppose that $x:1\to X$ factors through $A$ if
%   and only if it factors through $B$.  We claim that
%   $\ch{A}(x)=\ch{B}(x)$.  Indeed, $\ch{A}(x)=t$ iff $x$ factors
%   through $A$, iff $x$ factors through $B$, iff $\ch{B}(x)=t$.

%   ($3\Rightarrow 1$) This is just a restatement of Axiom
%   \ref{ax:classify}, which says that subobjects are in one-to-one
%   correspondence with characteristic functions.
% \end{proof}


We now use the existence of a truth-value object in $\cat{Sets}$ to
demonstrate further properties of functions.

\begin{exercise} Show that in any category, if $f:X\to Y$ is a regular
  monomorphism, then $f$ is monic. \end{exercise}

\begin{prop} Every monomorphism between sets is regular, i.e.\ an
  equalizer of a pair of parallel
  arrows.  \label{set-regular-monic} \end{prop}

\begin{proof} Let $m:B\to X$ be monic.  By Axiom \ref{ax:classify},
  the following is a pullback diagram:
  \[ \begin{tikzcd}
    B \arrow[d,"m"'] \arrow{r} & 1 \arrow{d}{t} \\
    X \arrow[r,"\ch{B}"'] & \Omega \end{tikzcd} \] A straightforward
  verification shows that $m$ is the equalizer of $X\stackrel{\beta
    _X}{\to} 1\stackrel{\tr}{\to}\Omega$ and $\ch{B}:X\to\Omega$.
  Therefore, $m$ is regular monic.
\end{proof}

Students with some background in mathematics might assume that if a
function $f:X\to Y$ is both a monomorphism and an epimorphism, then it
is an isomorphism.  However, that isn't true in all categories!  [For
example, it's not true in the category of monoids.]  Nonetheless,
$\cat{Sets}$ is a special category, and in this case we have the
result:

\begin{prop} In $\cat{Sets}$, if a function is both a monomorphism and
  an epimorphism, then it is an
  isomorphism. \label{sets:balanced} \end{prop}

\begin{proof} In any category, if $m$ is regular monic and epi, then
  $m$ is an isomorphism (Exercise \ref{herman}). \end{proof}

\begin{defn} Let $f:X\to Y$ be a function, and let $y\in Y$.  The
  \emph{fiber} over $y$ is the subset $f^{-1}\{ y\}$ of $X$ given by
  the following pullback:
  \[ \begin{tikzcd} f^{-1}\{ y\} \arrow[d] \arrow[r] & 1 \arrow[d,"y"] \\
    X \arrow[r,"f"] & Y \end{tikzcd} \] \end{defn}


\begin{prop} Let $p:X\to Y$.  If $p$ is not a surjection, then there
  is a $y_0\in Y$ such that the fiber $p^{-1}\{ y_0\}$ is
  empty. \label{empty-fiber} \end{prop}

\begin{proof} Since $p$ is not a surjection, there is a $y_0\in Y$
  such that for all $x\in X$, $p(x)\neq y_0$.  Now consider the
  pullback:
\[ \begin{tikzcd} 
1 \arrow[dashed]{rd}{z} \\
& {p^{-1}\{ y_0 \}} \arrow{d}{m} \arrow{r} &  1 \arrow{d}{y_0} \\
& X \arrow{r}{p}  & Y 
\end{tikzcd} \] If there were a morphism $z:1\to p^{-1}\{ y_0\}$, then
we would have $p(m(z))=y_0$, a contradiction.  Therefore, $p^{-1}\{
y_0\}$ is empty.
\end{proof}

\begin{prop} In $\cat{Sets}$, epimorphisms are
  surjective. \label{sets:es} \end{prop}

\begin{proof} Suppose that $p:X\to Y$ is not a surjection.  Then there
  is a $y_0\in Y$ such that for all $x\in X$, $p(x)\neq y_0$.  Since
  $1$ is terminal, the morphism $y_0:1\to Y$ is monic.  Consider the
  following diagram:
  \[ \begin{tikzcd}
    1 \arrow[ddr,"x"'] \arrow{rrd} \\
    & p^{-1}\{ y_0\} \arrow{d} \arrow{r} & 1 \arrow{d}{y_0} \arrow{r} & 1 \arrow{d}{\tr} \\
    & X \arrow{r}{p} & Y \arrow{r}{g} & \Omega \end{tikzcd} \] Here
  $g$ is the characteristic function of $\{ y_0\}$; by Axiom
  \ref{ax:classify}, $g$ is the unique function that makes the right
  hand square a pullback.  Let $x\in X$ be arbitrary. If we had
  $g(p(x))=\tr$, then there would be an element $x'\in p^{-1}\{y_0\}$,
  in contradiction with the fact that the latter is empty (Proposition
  \ref{empty-fiber}).  By Axiom \ref{ax:classify}, either
  $g(p(x))=\tr$ or $g(p(x))=\f$; therefore, $g(p(x))=\f$.  Now let $h$
  be the composite $Y\to 1\stackrel{\f}{\to} \Omega$.  Then, for any
  $x\in X$, we have $h(p(x))=\f$.  Since $g\circ p$ and $h\circ p$
  agree on arbitrary $x\in X$, we have $g\circ p=h\circ p$.  Since
  $g\neq h$, it follows that $p$ is not an epimorphism.
\end{proof}


In a general category, there is no guarantee that an epimorphism pulls
back to an epimorphism.  However, in $\cat{Sets}$, we have the
following:

\begin{prop} In $\cat{Sets}$, the pullback of an epimorphism is an
  epimorphism.  \label{sets-pullback-epi} \end{prop}

\begin{proof} Suppose that $f:Y\to Z$ is epi, and let $x\in X$.
  Consider the pullback diagram:
  \[ \begin{tikzcd}
    1 \arrow{ddr}[left]{x} \arrow{drr}{y} \arrow[dashed]{dr} \\
    & \ast \arrow{d}{q_0} \arrow[r,"q_1"'] &  Y \arrow{d}{f} \\
    & X \arrow[r,"g"'] & Z \end{tikzcd} \] By Proposition
  \ref{sets:es}, $f$ is surjective.  In particular, there is a $y\in
  Y$ such that $f(y)=g(x)$.  Since the diagram is a pullback, there is
  a unique $\langle x,y\rangle :1\to \ast$ such that $q_0\langle
  x,y\rangle =x$ and $q_1\langle x,y\rangle =y$.  Therefore, $q_0$ is
  surjective, and hence epi.
\end{proof}

\begin{prop} If $f:X\to Y$ and $g:W\to Z$ are epimorphisms, then so is
  $f\times g:X\times W\to Y\times Z$. \end{prop}

\begin{proof} Since $f\times g=(f\times 1)\circ (1\times g)$, it will
  suffice to show that $f\times 1$ is epi when $f$ is epi.  Now, the
  following diagram is a pullback:
\[ \begin{tikzcd}
  X\times W \arrow{r}{p_0} \arrow[d,"f\times 1"'] & X \arrow{d}{f} \\
  Y\times W \arrow{r}{p_0} & Y \end{tikzcd} \] By Proposition
\ref{sets-pullback-epi}, if $f$ is epi, then $f\times 1$ is
epi.  \end{proof}


Suppose that $f:X\to Y$ is a function, and that $p_0,p_1:X\times
_YX\rightrightarrows X$ is the kernel pair of $f$.  Suppose also that
$h:E\to Y$ is a function, that $q_0,q_1:E\times _YE\rightrightarrows
E$ is the kernel pair of $h$, and that $g:X\epi E$ is an epimorphism.
Then there is a unique function $b:X\times _YX\to E\times _YE$, such
that $q_0b=gp_0$ and $q_1b=gp_1$.
\[ \begin{tikzcd} X\times _Y X \arrow[d,dashed,"b"'] \arrow[r,shift
  left,"p_0"] \arrow[shift
  right,r,"p_1"'] & X \arrow[d,->>,"g"] \arrow[r,"f"] & Y \\
  E\times _Y E \arrow[shift left,r,"q_0"] \arrow[shift right,r,"q_1"']
  & E \arrow[ur,"h"'] \end{tikzcd} \] An argument similar to the one
above shows that $b$ is an epimorphism.  We will use this fact below
to describe the properties of epimorphisms in $\cat{Sets}$.


\section{Relations}

\subsection*{Equivalence relations and equivalence classes}

%% TO DO: subobject defined??

%% TO DO: for this we need image factorizations (regular category)

%% regular epis are stable under pullback

A relation $R$ on a set $X$ is a subset of $X\times X$; i.e.\ a set of
ordered-pairs.  A relation is said to be an equivalence relation just
in case it is reflexive, symmetric, and transitive.  One particular
way that equivalence relations on $X$ arise is from functions with $X$
as domain: given a function $f:X\to Y$, let say that $\langle
x,y\rangle \in R$ just in case $f(x)=f(y)$.  [Sometimes we say that,
``$x$ and $y$ lie in the same fiber over $Y$.'']  Then $R$ is an
equivalence relation on $X$.

Given an equivalence relation $R$ on $X$, and some element $x\in X$,
let $[x]=\{ y\in X \mid \langle x,y\rangle \in R \}$ denote the set of
all elements of $X$ that are equivalent to $X$.  We say that $[x]$ is
the {\bf equivalence class} of $x$.  It's straightforward to show that
for any $x,y\in X$, either $[x]=[y]$ or $[x]\cap [y]=\emptyset$.
Moreover, for any $x\in X$, we have $x\in [x]$.  Thus the equivalence
classes form a {\bf partition} of $X$ into disjoint subsets.

We'd like now to be able to talk about the set of these equivalence
classes, i.e.\ something that might intuitively be written as $\{
[x]\mid x\in X \}$.  The following axiom guarantees the existence of
such a set, called $X/R$, and a canonical mapping $q:X\to X/R$ that
takes each element $x\in X$ to its equivalence class $[x]\in X/R$.


% Intuitively speaking, we can gather together the fibers of $f$ into a
% set:
% \[ X/R \: = \: \{ f^{-1}\{ y\} \mid y\in Y \} .\] These fibers are
% called \emph{equivalence classes} of the relation $R$.  The set $E$
% has the following special properties: (1) there is a canonical
% epimorphism $q:X\to X/R$ that assigns each $x\in X$ its equivalence
% class in $X/R$, and (2) given a function $g:X\to Z$ that is constant
% on the equivalence classes [i.e.\ if $x$ and $y$ are equivalent, then
% $h(x)=h(x')$], there is a unique function $k:E\to Z$ such that $kg=h$.

Our next axiom guarantees the existence of the set of equivalence
classes.

\begin{axi}[Equivalence classes]{}
Let $R$ be an equivalence relation on $X$.  Then there is a
set $X/R$, and a function $q:X\to X/R$ with the properties: 
\begin{enumerate}
\item $\langle x,y\rangle \in R$ if and only if $q(x)=q(y)$.
\item For any set $Y$ and function $f:X\to Y$ that is constant on
  equivalence classes, there is a unique function $\overline{f}:X/R\to
  Y$ such that $\overline{f}\circ q=f$.  \end{enumerate}
\[ \begin{tikzcd} X \arrow[r,"f"] \arrow[d,"q"'] & Y \\
  X/R \arrow[dashed,ur,"\overline{f}"'] \end{tikzcd} \] Here $f$ is
constant on equivalence classes just in case $f(x)=f(y)$ whenever
$\langle x,y\rangle \in R$.  \label{ax:eq}
\end{axi}

An equivalence relation $R$ can be thought of as a subobject of
$X\times X$, i.e.\ a subset of ordered pairs.  Accordingly, there are
two functions $p_0:R\to X$ and $p_1:R\to X$ given by: $p_0\langle
x,y\rangle = x$ and $p_1\langle x,y\rangle =y$.  Then condition (1) in
the above axiom says that $q\circ p_0=q\circ p_1$.  And condition (2)
says that for any function $f:X\to Y$ such that $f\circ p_0=f\circ p
_1$, there is a unique function $\overline{f}:X/R \to Y$ such that
$\overline{f}\circ q=f$.  In this case, we say that $q$ is a
\emph{coequalizer} of $p_0$ and $p_1$.

\begin{exercise} Show that in any category, coequalizers are unique up
  to isomorphism. \end{exercise}

\begin{exercise} Show that in any category, a coequalizer is an
  epimorphism. \end{exercise}

\begin{exercise} For a function $f:X\to Y$, let $R=\{ \langle
  x,y\rangle \in X\times X \mid f(x)=f(y) \}$.  That is, $R$ is the
  kernel pair of $f$.  Show that $R$ is an equivalence
  relation. \end{exercise}

\begin{defn} A function $f:X\to Y$ is said to be a \emph{regular
    epimorphism} just in case $f$ is a coequalizer. \end{defn}

\begin{exercise} Show that in any category, if $f:X\to Y$ is both a
  monomorphism and a regular epimorphism, then $f$ is an
  isomorphism. \end{exercise}

\begin{prop} Every epimorphism in $\cat{Sets}$ is regular.  In
  particular, every epimorphism is the coequalizer of its kernel
  pair. \label{regs} \end{prop}

\begin{proof} Let $f:X\to Y$ be an epimorphism.  Let $p_0,p_1:X\times
  _YX\rightrightarrows X$ be the kernel pair of $f$.  By Axiom
  \ref{ax:eq}, the coequalizer $g:X\to E$ of $p_0$ and $p_1$ exists;
  and since $f$ also coequalizes $p_0$ and $p_1$, there is a unique
  function $m:E\to Y$ such that $f=mg$.
  \[ \begin{tikzcd} X\times _Y X \arrow[r,shift left,"p_0"]
    \arrow[r,shift right,"p_1"']
    \arrow[d,"b"] & X \arrow[r,"f"] \arrow[d,"g"] & Y \\
    E\times _Y E \arrow[r,shift left,"q_0"] \arrow[r,shift
    right,"q_1"'] & E \arrow[ur,"m"'] \end{tikzcd} \] Here $E\times
  _YE$ is the kernel pair of $m$.  Since $mgp_0=fp_0=fp_1=mgp_1$,
  there is a unique function $b:X\times _YX\to E\times _YE$ such that
  $gp_0=q_0b$ and $gp_1=q_1b$.  By the considerations at the end of
  the previous section, $b$ is an epimorphism.  Furthermore,
  \[ q_0b = gp_0 = gp_1 = q_1b ,\] and therefore $q_0=q_1$.  By
  Exercise \ref{kkp}, $m$ is a monomorphism.  Since $f=mg$, and $f$ is
  epi, $m$ is also epi.  Therefore, by Proposition
  \ref{sets:balanced}, $m$ is an isomorphism.
\end{proof}

This last proposition actually shows that $\cat{Sets}$ is what is
known as a \emph{regular category}.  In general, a category $\cat{C}$
is said to be \emph{regular} just in case it has all finite limits, if
coequalizers of kernel pairs exist, and if regular epimorphisms are
stable under pullback.  Now, it's known that if a category has
products and equalizers, then it has all finite limits.  Thus
$\cat{Sets}$ has all finite limits.  Our most recent axiom says that
$\cat{Sets}$ has coequalizers of kernel pairs.  And finally, all
epimorphisms in $\cat{Sets}$ are regular, and epimorphisms in
$\cat{Sets}$ are stable under pullback; therefore, regular
epimorphisms are stable under pullback.

Regular categories have several nice features that will prove quite
useful.  In the remainder of this section, we will discuss one such
features: factorization of functions into a regular epimorphism
followed by a monomorphism.

\subsection*{The epi-monic factorization}

Let $f:X\to Y$ be a function, and let $p_0,p_1:X\times
_YX\rightrightarrows X$ be the kernel pair of $f$.  By Axiom
\ref{ax:eq}, the kernel pair has a coequalizer $g:X\epi E$.  Since $f$
also coequalizes $p_0$ and $p_1$, there is a unique function $m:E\to
Y$ such that $f=mg$.
\[ \begin{tikzcd}
  X\times _YX \arrow[r,shift left,"p_0"] \arrow[r,shift right,"p_1"'] & X \arrow[rr,"f"] \arrow[->>,rd,"g"'] & & Y \\
  & & E \arrow[ur,"m"'] \end{tikzcd} \] An argument similar to the one
in Proposition \ref{regs} shows that $m$ is a monomorphism.  Thus,
$(E,m)$ is a subobject of $Y$, which we call the \emph{image} of $X$
under $f$, and we write $E=f(X)$.  The pair $(g,m)$ is called the
\emph{epi-monic factorization} of $f$.  Since epis are surjections,
and monics are injections, $(g,m)$ can also be called the
surjective-injective factorization.

%% TO DO: Show uniqueness of factorization

\begin{defn} Suppose that $A$ is a subset of $X$, in particular,
  $n:A\to X$ is monic.  Then $f\circ n:A\to Y$, and we let $f(A)$
  denote the image of $A$ under $f\circ n$.
  \[ \begin{tikzcd}
    A \arrow[d,"n"'] \arrow[dashed,r] & f(A) \arrow[>->,d] \\
    X \arrow[r,"f"] & Y \end{tikzcd} \] We also use the suggestive
  notation
\[ f(A) \: = \: \exists _f(A) \: = \: \{ y\in Y \mid \exists x\in
A.f(x)=y \} . \] \end{defn}

\begin{prop} Let $f:X\to Y$ be a function, and let $A$ be a subobject
  of $X$. The image $f(A)$ is the smallest subobject of $Y$ through
  which $f$ factors. \label{image-smallest} \end{prop}

\begin{proof} Let $e:X\to Q$ and $m:Q\to Y$ be the epi-monic
  factorization of $f$.  Suppose that $n:B\to Y$ is a subobject, and
  that $f$ factors through $n$, say $f=ng$.  Consider the following
  diagram.
  \[ \begin{tikzcd} E \arrow[shift left=0.5ex]{r}[above]{p_0}
    \arrow[shift right=0.5ex]{r}[below]{p_1} & X \arrow{r}{f}
    \arrow[->>]{d}[left]{e}
    \arrow{dr}{g} & Y \\
    & Q \arrow[dashed]{r}{k} & B
    \arrow[>->]{u}[right]{n} \end{tikzcd} \] Then
  $ngp_0=fp_0=fp_1=ngp_1$, since $p_0,p_1$ is the kernel pair of $f$.
  Since $n$ is monic, $gp_0=gp_1$, i.e.\ $g$ coequalizes $p_0$ and
  $p_1$.  Since $e:X\to Q$ is the coequalizer of $p_0$ and $p_1$,
  there is a unique function $k:Q\to B$ such that $ke=g$.  By
  uniqueness of the epi-monic factorization, $nk=m$.  Therefore,
  $Q\subseteq B$.
\end{proof}
% adjunction!




\begin{prop} For any $A\subseteq X$ and $B\subseteq Y$, we have 
  \[ A\subseteq f^{-1}(B) \quad \text{if and only if} \quad \exists
  _f(A)\subseteq B .\] \end{prop}

\begin{proof} Suppose first that $A\subseteq f^{-1}(B)$, in particular
  that $k:A\to f^{-1}(B)$.  Consider the following diagram:
  \[ \begin{tikzcd}
    A \arrow{d}[left]{k} \arrow{r}{e} & \exists _f(A) \arrow[bend left=60]{dd}{j} \arrow[dashed]{d} \\
    f^{-1}(B) \arrow{d}[left]{m^*} \arrow{r} & B \arrow{d}{m} \\
    X \arrow{r}[below]{f} & Y \end{tikzcd} \] By definition, $je$ is the
  epi-mono factorization of $fm^*k$.  Since $fm^*k$ also factors
  through $m:B\to Y$, we have $\exists _f(A)\subseteq B$, by
  Proposition \ref{image-smallest}.

Suppose now that $\exists _f(A)\subseteq B$.  Using the fact that the
lower square in the diagram is a pullback, we see that there is an
arrow $k:A\to f^{-1}(B)$ such that $m^*k$ is the inclusion of $A$ in
$X$.  That is, $A\subseteq f^{-1}(B)$.  
\end{proof}

\begin{exercise} Use the previous result to show that $A\subseteq
  f^{-1}(\exists _f(A))$, for any subset $A$ of $X$. \end{exercise}







% \begin{prop} If $f:X\to Y$ is epi, and $f=me$ is the
%   surjective-injective factorization, then $m$ is an
%   isomorphism. \end{prop}

% \begin{proof} If $f=me$ is epi, then $m$ is epi (Proposition \ref{?}).
%   But $m$ is also monic, and therefore an isomorphism (Corollary
%   \ref{sets:balanced}).  \end{proof}





% \begin{prop} Let $X,Y$ be sets, and let $f:X\to Y$ be a function.
%   Then $f$ is a surjection if and only if the kernel pair diagram for
%   $f$ is also a pushout. \end{prop}

% \begin{proof}
%   Consider the following pullback diagram:
%   \begin{equation} \begin{tikzcd}
%       X\times _f X \arrow{r}{\pi _1} \arrow{d}[left]{\pi _0} & X \arrow{d}{f} \\
%       X \arrow{r}[below]{f} &
%       Y \end{tikzcd} \label{push-pull} \end{equation} Suppose first
%   that $f$ is a surjection.  Let $g:X\to Z$ and $h:X\to Z$ such that
%   $g\pi _0=h\pi _1$.  Then $h=g$, and $g$ is constant on each fiber
%   $f^{-1}\{ y\}$.  Moreover, since $f$ is surjective, each fiber is
%   nonempty.  Thus, $k:Y\to Z$ can (and must) be defined by setting
%   $k(y)=g[f^{-1}\{ y\}]$.

%   Suppose now that (\ref{push-pull}) is a pushout.  Since $X\times
%   _fX$ contains the diagonal, $\pi _0$ is surjective and hence epi.
%   Since the pushout of an epi is epi, $f$ is epi.  Finally, since
%   epimorphisms in $\mathbf{Sets}$ are surjective, $f$ is
%   surjective. \end{proof}




% \subsection{The calculus of relations}


% Note that Axiom \ref{ax:eq} doesn't actually use the notion of an
% equivalence relation.  We now turn to giving a precise definition of
% this notion.

% \begin{defn} A \emph{relation} from $A$ to $X$ is a subobject $m:R\to
%   A\times X$.  A relation on $X$ is a subobject $m:R\to X\times X$.
% \end{defn}

% Recall that each function $m:R\to X\times X$ is determined by its
% projections $\pi _0m$ and $\pi _1m$.  Thus, we can also think of a
% relation as a pair of functions $m_0,m_1:R\rightrightarrows X$.
% Intuitively, two elements $x,y\in X$ stand in the relation $R$ just in
% case there is a $z\in R$ such that $m_0(z)=x$ and $m_1(z)=y$.

% \begin{defn} A relation $m:R\to X\times X$ is said to be
%   \emph{reflexive} just in case the diagonal $\dl :X\to X\times X$
%   factors through $m$.
%   \[ \begin{tikzcd}
%     & R \arrow{d}{m} \\
%     X \arrow[dashed]{ur} \arrow{r}[below]{\dl} & X\times X
% \end{tikzcd} \]
% In terms of elements, $\langle x,x\rangle \in R$ for all $x\in X$.
% \end{defn}


% Recall that a Cartesian product of $X$ and $Y$ is usually written as
% $X\times Y$, as if there were just one set that can play this role.
% In fact, a Cartesian product of $X$ and $Y$ is any triple $\{
% Z,p_0,p_1\}$, where $Z$ is a set, $p_0:Z\to X$, $p_1:Z\to Y$,
% satisfying the properties described in Axiom \ref{ax:set-prod}.
% Consider, in particular, the case where we take the Cartesian product
% of $X$ with itself.  Suppose that $\{ X\times X,p_0,p_1\}$ has the
% requisite properties.  Then it is easy to see that $\{ X\times
% X,p_1,p_0\}$ also has the requisite properties, although it differs
% from the former product in the order of its projections.  By
% Proposition \ref{unique-product}, there is an isomorphism $\sigma
% :X\times X\to X\times X$ such that $p_0\sigma =p_1$ and $p_1\sigma
% =p_0$.  Written in terms of elements, we have $\sigma \langle
% x,y\rangle = \langle y,x\rangle$.  We call $\sigma$ the \emph{twist
%   map} of $X\times X$.  (Technically, we could write $\sigma _X$, and
% then we could show that this family of maps is natural in $X$.)

% \begin{defn} Let $m:R\to X\times X$ be a relation on $X$.  We write
%   $R^{op}$ for the relation $\sigma m:R\to X\times X$.  Intuitively,
%   \[ R^{op} \: = \: \{ \langle x,y\rangle \in X\times X\mid \langle
%   y,x\rangle \in R \} .\] We say that $R$ is \emph{symmetric} just in
%   case $R^{op}\subseteq R$ as subobjects of $X\times X$.  Intuitively,
%   $R$ is symmetric just in case $\langle y,x\rangle \in R$ whenever
%   $\langle x,y\rangle \in R$. \end{defn}


% %% TO DO: existential quantifier as projection

% Suppose that $m:B\to X\times Y$ is a subobject, i.e.\ a subset of
% $X\times Y$.  We would like to create a subset of $X$ of elements that
% satisfy the following condition:
% \[ \{ x\in X \mid \exists y\in Y.\langle x,y\rangle \in B \} .\]
% Consider the following diagram:
% \[ \begin{tikzcd}
%   B \arrow{d}[left]{m} \arrow[->>]{r}{p}  &  \pi _0(B)  \arrow[>->]{d}{i} \\
%   X\times Y \arrow{r}[below]{\pi _0} & X \end{tikzcd} \] Here $ip:B\to
% X$ is the surjective-injective factorization of $\pi _0m$.  In other
% words, $\pi _0(B)$ is the image of $B$ under $\pi _0$.


% \begin{defn} Suppose that $m:R\to X\times Y$ and $n:S\to Y\times Z$
%   are relations.  We define a relation $S\circ R$ on $X\times Z$ as
%   follows: first we think of $R$ and $S$ as subsets of $X\times
%   Y\times Y\times Z$ by pulling back along the projections to $X\times
%   Y$ and $Y\times Z$.  Then we pull back along the diagonal $\dl
%   :Y\to Y\times Y$ to give us the subset 
%   \[ M \: = \: \{ \langle x,y,z\rangle \mid \langle x,y\rangle \in R\;
%   \text{and}\; \langle y,z\rangle \in S \} ,\] of $X\times Y\times Z$.
%   Finally, we project out $Y$ to give us the set
%   \[ S\circ R \: = \: \{ \langle x,z\rangle \mid \exists y\in
%   Y.\langle x,y\rangle \in R\;\text{and}\;\langle y,z\rangle \in S \}
%   .\] The first two steps required taking pullbacks, which preserve
%   monics (i.e.\ subsets).  The final step takes a projection ... [[TO
%   DO]]
% \end{defn}

% \begin{defn} We say that a relation $R$ on $X$ is \emph{transitive}
%   just in case $R\circ R\subseteq R$.  In other words: for any
%   $x,y,z\in X$, if $\langle x,y\rangle \in R$ and $\langle y,z\rangle
%   \in R$, then $\langle x,z\rangle\in R$. \end{defn}

% \begin{defn} We say that a relation $R$ on $X$ is an \emph{equivalence
%     relation} just in case $R$ is reflexive, symmetric, and
%   transitive. \end{defn}


% \begin{exercise} Given a surjection $f:X\to Y$, let
%   $p_0,p_1:E_f\rightrightarrows X$ be the kernel pair of $f$.  Show
%   that $E_f$, considered as a subobject of $X\times X$, is an
%   equivalence relation on $X$. \end{exercise}

% Given an equivalence relation $E$ on $X$, we let $f_E:X\to Q$ denote
% the [[start here]]


%   \begin{prop} For any surjection $f:X\to Y$, we have $f=f_{E_f}$.
%     For any equivalence relation $E$ on $X$, we have
%     $E=E_{f_E}$.  \end{prop}


% \begin{proof}  Suppose that $f:X\to Y$ is an epimorphism.  Consider
%   the following diagram, where $p_0,p_1:E_f\rightrightarrows X$ is the
%   kernel pair of $f$, and $q:X\to Z$ is the coequalizer of $p_0$ and
%   $p_1$.   
%   \[ \begin{tikzcd} E_f \arrow[shift left=0.5ex]{r}[above]{p_0} \arrow[shift
%     right=0.5ex]{r}[below]{p_1} &
%     X \arrow[->>]{d}{q} \arrow[->>]{r}{f} & Y \\
%     & Z \arrow[dashed]{ur}{k} \end{tikzcd} \] Since $fp_0=fp_1$, the
%   universal property of the coequalizer entails that there is a unique
%   $k:Z\to Y$ such that $kq=f$.  By Proposition \ref{comp-epi}, $k$ is
%   an epimorphism.  

%   We claim that $k$ is injective.  Suppose that $z,z'\in Z$ such that
%   $k(z)=k(z')$.  Since $q$ is surjective, there are $x,x'\in X$ such
%   that $q(x)=z$ and $q(x')=z'$.  Hence $f(x)=kq(x)=kq(x')=f(x')$, and
%   thus $\langle x,x'\rangle \in E_f$.  Since $qp_0=qp_1$, it follows
%   that 
%   \[ z=q(x)=qp_0\langle x,x'\rangle = qp_1\langle x,x'\rangle =
%   q(x')=z' . \] Therefore, $k$ is injective.  By Proposition
%   \ref{sets:balanced}, $k$ is an isomorphism.

%   Now let $p_0,p_1:E\to X$ be an equivalence relation on $X$, and let
%   $f:X\to Q$ be the coequalizer of $p_0$ and $p_1$.  Let
%   $q_0,q_1:E_f\to X$ be the kernel pair of $f$ so that we have the
%   following:
%   \[ \begin{tikzcd} & E \arrow[dashed]{dl}[left]{k} \arrow[shift
%     right=0.5ex]{d}[left]{p_0} \arrow[shift left=0.5ex]{d}[right]{p_1} \\
%     E_f \arrow[shift left=0.5ex]{r}[above]{q_0} \arrow[shift
%     right=0.5ex]{r}[below]{q_1}
%     & X \arrow[->>]{d}{f} \\
%     & Q \end{tikzcd} \] Since $fp_0=fp_1$, and since $\{
%   E_f,q_0,q_1\}$ is the pullback of $f$ along itself, $k:E\to E_f$ is
%   the unique arrow such that $q_ik=p_i$.  We then have $\langle
%   q_0,q_1\rangle k=\langle p_0,p_1\rangle$, from which it follows that
%   $k$ is monic.  In other words, $E\subseteq E_f$ as subobjects of
%   $X\times X$.  

%   To conclude the argument, we need to show that $E_f\subseteq E$.
%   For this, suppose that $\langle x_0,y_0\rangle\not\in E$.  Let $B$
%   be the equivalence class of $x_0$, i.e.\ $B = \{ x\in X \mid \langle
%   x_0,x\rangle \in E \}$.  By Axiom \ref{ax:classify}, there is a
%   function $\ch{B}:X\to\Omega$ such that $\ch{B}(x)=t$ if $x\in B$,
%   and $\ch{B}(x)=f$ if $x\not\in B$.  In particular, $\ch{B}(x_0)=t$
%   and $\ch{B}(y_0)=f$.  We claim that $\chi _B$ equalizes $p_0$ and
%   $p_1$.  Indeed, for any $\langle x,y\rangle\in E$, if $x\in B$ if
%   and only if $y\in B$, hence $\ch{B}(x)=\ch{B}(y)$.  Since $(Q,f)$ is
%   the coequalizer of $p_0,p_1$, there is a unique morphism
%   $g:Q\to\Omega$ such that $gf=\ch{B}$.  Thus, if $f(x_0)=f(y_0)$,
%   then $\ch{B}(x_0)=gf(x_0)=gf(y_0)=\ch{B}(y_0)$, contradicting the
%   fact that $x_0\in B$ but $y_0\not\in B$.  Therefore, $f(x_0)\neq
%   f(y_0)$ and $\langle x_0,y_0\rangle\not\in E_f$.
% \end{proof}

\subsection*{Functional relations}

\begin{defn} A relation $R\subseteq X\times Y$ is said to be
  \emph{functional} just in case for each $x\in X$ there is a unique
  $y\in Y$ such that $\langle x,y\rangle \in R$. \end{defn}

\begin{defn} Suppose that $f:X\to Y$ is a function.  We let
  $\mathrm{graph}(f)= \{ \langle x,y\rangle \mid f(x)=y
  \}$. \end{defn}

\begin{exercise} Show that $\mathrm{graph}(f)$ is a functional
  relation. \end{exercise}

The following result is helpful for establishing the existence of
arrows $f:X\to Y$.

\begin{prop} Let $R\subseteq X\times Y$ be a functional relation.
  Then there is a unique function $f:X\to Y$ such that
  $R=\mathrm{graph}(f)$. \end{prop}

The proof of this result is somewhat complicated, and we omit it (for
the time being).

%% TO DO: this shows that any two empty sets are isomorphic!


\section{Colimits}


\newcommand{\cpr}{\amalg}

\begin{axi}[Coproducts]{ax:set-coprod} For any two sets $X,Y$, there
  is a set $X\amalg Y$ and functions $i_0:X\to X\amalg Y$ and
  $i_1:Y\to X\amalg Y$ with the feature that for any set $Z$, and
  functions $f:X\to Z$ and $g:Y\to Z$, there is a unique function
  $f\cpr g:X\cpr Y\to Z$ such that $(f\cpr g)\circ i_0=f$ and
  $(f\cpr g)\circ i_1=g$.
  \[ \begin{tikzcd}
    & Z  \\
    & X \cpr Y \arrow[dashed,u,"f\cpr g"] \\
    X \arrow[bend left,uur,"f"] \arrow[ur,"i_0"'] & & Y \arrow[bend
    right,uul,"g"'] \arrow[ul,"i_1"] \end{tikzcd} \] We call $X\cpr Y$
  the \emph{coproduct} of $X$ and $Y$.  We call $i_0$ and $i_1$ the
  coprojections of the coproduct.
   \end{axi}

   Intuitively speaking, the coproduct $X\cpr Y$ is the disjoint union
   of the sets $X$ and $Y$.  What we mean by ``disjoint'' here is that
   if $X$ and $Y$ share elements in common (which doesn't make sense
   in our framework, but does in some frameworks), then these elements
   are dis-identified before the union is taken.  For example, in
   terms of elements, we could think of $X\cpr Y$ as consisting of
   elements of the form $\langle x,0\rangle$, with $x\in X$, and
   elements of the form $\langle y,1\rangle$, with $y\in Y$.  Thus, if
   $x$ is contained in both $X$ and $Y$, then $X\cpr Y$ contains two
   separate copies of $x$, namely $\langle x,0\rangle$ and $\langle
   x,1\rangle$.

   We now show that that the inclusions $i_0:X\to X\cpr Y$ and
   $i_1:Y\to X\cpr Y$ do, in fact, have disjoint images.

   \begin{prop} Coproducts in $\cat{Sets}$ are disjoint.  In other
     words, if $i_0:X\to X\amalg Y$ and $i_1:Y\to X\amalg Y$ are the
     coprojections, then $i_0(x)\neq i_1(y)$ for all $x\in X$ and
     $y\in Y$.  \label{coprod-disjoint} \end{prop}

   \begin{proof} Suppose for reductio ad absurdum that
     $i_0(x)=i_1(y)$.  Let $g:X\to \Omega$ be the unique map that
     factors through $\tr:1\to \Omega$.  Let $h:Y\to\Omega$ be the
     unique map that factors through $\f:1\to\Omega$.  By the
     universal property of the coproduct, there is a unique function
     $g\cpr h:X\cpr Y\to \Omega$ such that $(g\cpr h)i_0=g$ and
     $(g\cpr h)i_1=h$.  Thus, we have
     \[ \tr=g(x)=(g\cpr h)i_0x = (g\cpr h)i_1y=h(y)=\f ,\] a
     contradiction.  Therefore, $i_0(x)\neq i_1(y)$, and the ranges of
     $i_0$ and $i_1$ are disjoint. \end{proof}

\begin{prop} The coprojections $i_0:X\to X\cpr Y$ and $i_1:Y\to X\cpr
  Y$ are monomorphisms.  \label{coprod-monic} \end{prop}

\begin{proof} We will show that $i_0$ is monic; the result then
  follows by symmetry.  Suppose first that $X$ has no elements.  Then
  $i_0$ is trivially injective, hence monic by Proposition \ref{inj-mon}.
  Suppose now that $X$ has an element $x:1\to X$.  Let $g=x\circ \beta
  _Y$, where $\beta _Y:Y\to 1$.  Then $(1_X\cpr g)i_0=1_X$, and Exercise
  \ref{comps} entails that $i_0$ is monic.
\end{proof}

% \begin{prop} Let $x:1\to X$, and let $g:X\to\Omega$ be the
%   characteristic function of $\{ x\}$, considered as a subobject of
%   $X$.  If $y:1\to X$ such that $y\neq x$, then $gy=0$. \end{prop}

% \begin{proof} The following diagram is a pullback:
% \[ \begin{tikzcd} 
% 1 \arrow{r} \arrow{d}{x} & 1 \arrow{d}{t} \\
% X \arrow{r}{g} & \Omega \end{tikzcd} \]
% By Axiom \ref{classify}, there are exactly two functions
%   from $1$ to $\Omega$.  Thus, either $gy=0$ or $gy=1$.  If $gy=1$  If
%   $gy=1$, then $y$ factors through $x$.  Therefore, $y=x$.  
% \end{proof}

\begin{prop} The coprojections are jointly surjective.  That is, for
  each $z\in X\cpr Y$, either there is an $x\in X$ such that
  $z=i_0(x)$, or there is a $y\in Y$ such that
  $z=i_1(y)$. \label{coprod-cover} \end{prop}

\begin{proof} Suppose for reductio ad absurdum that $z$ is neither in
  the image of $i_0$ nor in the image of $i_1$.  Let $g:(X\cpr
  Y)\to\Omega$ be the characteristic function of $\{ z_0\}$. Then for
  all $x\in X$, $g(i_0(x))=f$.  And for all $y\in Y$, $g(i_1(y))=f$.
  Now let $h:(X\cpr Y)\to \Omega$ be the constant $f$ function, i.e.\
  $h(z)=f$ for all $z\in X\cpr Y$.  Then $gi_0=hi_0$ and $gi_1=hi_1$.
  Since functions from $X\cpr Y$ are determined by their
  coprojections, $g=h$, a contradiction.  Therefore, all $z\in X\cpr
  Y$ are either in the range of $i_0$ or in the range of $i_1$.
\end{proof}

\begin{prop} The function $\tr\amalg \f:1\amalg 1\to \Omega$ is an
  isomorphism.  \label{set:boolean} \end{prop}

\begin{proof}  Consider the diagram:
  \[ \begin{tikzcd}
    & \Omega \\
    & 1 \cpr  1 \arrow[dashed,u,"\tr\amalg\f"] \\
    1 \arrow[bend left,uur,"\tr"] \arrow[ur,"i_0"'] & & 1 \arrow[bend
    right,uul,"\f"'] \arrow[ul,"i_1"] \end{tikzcd} \] Then
  $\tr\amalg\f$ is monic since every element of $1\cpr 1$ factors
  through either $i_0$ or $i_1$ (Proposition \ref{coprod-cover}), and
  since $\tr\neq \f$.  Furthermore, $\tr\amalg\f$ is epi since $\tr$
  and $\f$ are the only elements of $\Omega$.  By Proposition
  \ref{sets:balanced}, $\tr\amalg\f$ is an isomorphism. \end{proof}

\begin{prop} Let $X$ be a set, and let $B$ be a subset of $X$.  Then
  the inclusion $B\amalg X\backslash B\to X$ is an
  isomorphism. \label{xmid} \end{prop}

\begin{proof} Using the fact that $\Omega$ is Boolean, for every $x\in
  X$, either $x\in B$ or $x\in X\backslash B$.  Thus the inclusion
  $B\amalg X\backslash B\to X$ is a bijection, hence an
  isomorphism. \end{proof}


% \begin{defn} Let $A$ and $B$ be subsets of $X$.  We say that $A$ and
%   $B$ are \emph{disjoint} just in case there is no element $x\in X$
%   such that $x\in A$ and $x\in B$. \end{defn}

% \begin{prop} Suppose that $A$ and $B$ are disjoint subsets of $X$.
%   Then $A\cpr B$ is a subobject of $X$, and is the smallest subobjet of
%   $X$ that contains both $A$ and $B$. \end{prop}

% \begin{proof} Let $m:A\to X$ and $n:B\to X$ be the inclusions.  First
%   we need to show that $(m\cpr n):A\cpr B\to X$ is monic.  Suppose that
%   $x,y\in A\cpr B$ such that $(m\cpr n)(x)=(m\cpr n)(y)$.  By Proposition
%   \ref{coprod-cover}, both $x$ and $y$ are either in $A$ or $B$.
%   Since $A$ and $B$ are disjoint, it's not the case that $x\in X$ and
%   $y\in Y$, or vice versa.  If $x,y\in X$, then \[
%   m(x)=(m\cpr n)(x)=(m\cpr n)(y)=m(y) ,\] and since $m$ is monic, $x=y$.
%   Similarly, if $x,y\in Y$, then $x=y$.  In either case, $x=y$, and
%   therefore $m\cpr n$ is monic. \end{proof}

\newcommand{\es}{\emptyset}

\begin{axi}[Empty set]{ax:zero} There is a set $\es$ with the
  following properties:
\begin{enumerate}
\item For any set $X$, there is a unique function 
  \[ \begin{tikzcd} \es \arrow[r,"\alpha _X"] & X \end{tikzcd} \] In
  this case, we say that $\es$ is an \emph{initial object} in
  $\cat{Sets}$.
\item $\es$ is empty, i.e.\ there is no function $x:1\to\es$.
\end{enumerate} \label{ax:es} \end{axi}




\begin{exercise} Show that in any category with coproducts, if $A$ is
  an initial object, then $X\cpr A\cong X$, for any object
  $X$. \end{exercise}


\begin{prop} Any function $f:X\to\es$ is an isomorphism. \end{prop}

\begin{proof} Since $\es$ has no elements, $f$ is trivially
  surjective.  We now claim that $X$ has no elements.  Indeed, if
  $x:1\to X$ is an element of $X$, then $f(x)$ is an element of $\es$.
  Since $X$ has no elements, $f$ is trivially injective.  By
  Proposition \ref{sets:balanced}, $f$ is an isomorphism. \end{proof}

\begin{prop} A set $X$ has no elements if and only if $X\cong
  \es$. \label{empty-init} \end{prop}

\begin{proof} By Axiom \ref{ax:es}, the set $\es$ has no elements.
  Thus if $X\cong\es$, then $X$ has no elements.

  Suppose now that $X$ has no elements.  Since $\es$ is an initial
  object, there is a unique arrow $\alpha _X:\es\to X$.  Since $X$ has
  no elements, $\alpha _X$ is trivially surjective.  Since $\es$ has
  no elements, $\alpha _X$ is trivially injective.  By Proposition
  \ref{sets:balanced}, $f$ is an isomorphism. \end{proof}


%% TO DO: pushouts


% \section{Subobjects}

% \newcommand{\sub}[1]{\mathrm{Sub}(#1)}

% \begin{defn} For a set $X$, let $\mathrm{Sub}(X)$ denote the set of
%   equivalence classes of subobjects of $X$.  In other words, we treat
%   two subobjects $m:A\to X$ and $n:B\to X$ as equal just in case there
%   is an isomorphism $k:A\to B$ such that $m=nk$.  By Axiom
%   \ref{ax:classify}, subobjects of $X$ correspond one-to-one with
%   functions $X\to\Omega$.  \end{defn}

% \begin{prop} $\sub{X}$ is a partially ordered when equipped with the
%   relation $\subseteq$. \end{prop}

% \begin{proof} Given $m:B\to X$, the identity arrow $1_B:B\to B$ shows
%   that $B\subseteq B$.  Suppose now that $m:B\to X$ and $n:A\to X$ are
%   subobjects such that $A\subseteq B$ and $B\subseteq A$.  That is,
%   there are functions $f:A\to B$ and $g:B\to A$ such that $mf=n$ and
%   $ng=m$.  Then, $m(fg)=(mf)g=ng=m$.  Since $m$ is monic, $fg=1_B$.
%   Similarly, $gf=1_A$.  Therefore $A\cong B$.  The last step of the
%   proof is left to the following exercise.
% \end{proof}

% \begin{exercise} Show that if $A\subseteq B$ and $B\subseteq C$, then
%   $A\subseteq C$. \end{exercise}

% \begin{defn} Given subobjects $A$ and $B$ of $X$, we define another
%   subobject $A\cap B$ by taking the pullback:
%   \[ \begin{tikzcd}
%     A\cap B \arrow{d} \arrow{r} & B \arrow{d}{n} \\
%     A \arrow[r,"m"'] & X \end{tikzcd} \] Thus, $A\cap B\subseteq A$ and
%   $A\cap B\subseteq B$.  \end{defn}

% Recall that a pullback of two arrows $f$ and $g$ can be alternately
% constructed as an equalizer: the set of pairs $\langle x,y\rangle$
% such that $fx=gy$.  In this case, $A\cap B$ can be described as the
% set of pairs $a\in A$ and $b\in B$ such that $m(a)=n(b)$.  In
% diagrammatic form, the top square is a pullback:
% \[ \begin{tikzcd}
%   A\cap B \arrow{d} \arrow{r} & X \arrow{d}{\dl} \\
%   A\times B \arrow{r}{m\times n} \arrow{d} & X\times X
%   \arrow{d}{\chi _A\times \chi _B} \\
%   1\times 1 \arrow{r}{t\times t} & \Omega\times\Omega \end{tikzcd} \]
% The bottom square, being the product of two pullbacks, is also a
% pullback.  Using the fact that $(\chi _A\times \chi _B)\dl =
% \langle \chi _A,\chi _B\rangle$, we conclude that the following is a
% pullback:
% \[ \begin{tikzcd}
% A\cap B \arrow{r} \arrow{d} &  1 \arrow{d}{\langle t,t\rangle} \\
% X \arrow{r}[below]{\langle \chi _A,\chi _B\rangle} &
% \Omega\times\Omega \end{tikzcd} \]

% \begin{defn} Let $B\subseteq X$.  We define the \emph{complement}
%   $X\backslash B$ as follows: let $\ch{B}:X\to\Omega$ be the
%   characteristic function of $B$, and take the pullback:
%   \[ \begin{tikzcd}
%     X\backslash B \arrow{r} \arrow{d} & 1 \arrow{d}{f} \\
%     X \arrow[r, "\ch{B}"'] & \Omega \end{tikzcd} \] where $f:1\to
%   \Omega$ is the ``false'' element.  That is, $X\backslash B$ is the
%   set of $x\in X$ such that $\ch{B}(x)=f$.  More formally:
%   \[ X\backslash B \: = \: \{ x\in X\mid \ch{B}(x)=f \} .\] By Axiom
%   \ref{ax:classify}, for every $x\in X$, either $\ch{B}(x)=t$ or
%   $\ch{B}(x)=f$, but not both.  Moreover, $\ch{B}(x)=t$ if and only if
%   $x\in B$.  Thus, we have
% \[ X\backslash B \: = \: \{ x\in X\mid x\not\in B  \} .\] \end{defn}





% \begin{prop} Let $\ell :C\to X$ be a subobject such that $C\subseteq
%   A$ and $C\subseteq B$.  Then $C\subseteq A\cap B$. \end{prop}

% \begin{proof} Since $C\subseteq A$, there is a function $j:C\to A$
%   such that $mj=\ell$.  And since $C\subseteq B$, there is a function
%   $k:C\to B$ such that $nk=\ell$.  Consider the pullback diagram:
%   \[ \begin{tikzcd}
%     C \arrow{ddr}[left]{j} \arrow{drr}{k} \arrow[dashed]{dr}{f} \\
%     & A \cap B \arrow[r,"i_1"'] \arrow{d}{i_0} & B \arrow{d}{n} \\
%     & A \arrow{r}{m} & X \end{tikzcd} \] Here $f:C\to A\cap B$ is the
%   unique function such that $i_0f=j$ and $i_1f=k$.  Therefore,
%   $C\subseteq A\cap B$.
% \end{proof}


% \subsection{Boolean structure of truth-values}

% We now show that $\Omega$ comes equipped with functions that act as
% truth-functional connectives: $\neg ,\vee,\wedge$.

% Since $\Omega$ has two elements, $\Omega$ has four subobjects.
% (Recall that distinct subobjects can be separated by elements.)  The
% subobjects of $\Omega$ are $\alpha :0\to\Omega$, $t:1\to\Omega$,
% $f:1\to\Omega$, and $\mathrm{id}:\Omega\to\Omega$.  Since subobjects
% of $\Omega$ correspond to functions $\Omega\to\Omega$, there are four
% such functions.

% \begin{defn} We define the function $\neg :\Omega\to\Omega$ to be the
%   characteristic function of the subobject $f:1\to\Omega$.  Thus, the
%   following is a pullback:
%   \[ \begin{tikzcd}
%     1  \arrow[d,"f"'] \arrow[r,"\mathrm{id}"] & 1 \arrow[d,"t"] \\
%     \Omega \arrow[r,"\neg"'] & \Omega \end{tikzcd} \] In other words,
%   $\neg f=t$.  \end{defn}

% We claim that $\neg t=f$.  If $\neg t=t$, then since the diagram above
% is a pullback, $t:1\to\Omega$ would factor through $f:1\to\Omega$,
% which would entail that $t=f$, a contradiction.
% \[ \begin{tikzcd}
%   1 \arrow[ddr,"t"'] \arrow[drr,"\mathrm{id}"] \arrow[dashed,dr] \\
%   & 1  \arrow[d,"f"] \arrow[r,"\mathrm{id}"'] & 1 \arrow[d,"t"] \\
%   & \Omega \arrow[r,"\neg"'] & \Omega \end{tikzcd} \] Therefore, $\neg
% \neg f=f$ and $\neg\neg t=t$.  Using the fact that functions are
% individuated by their elements, we have $\neg\neg =
% \mathrm{id}_{\Omega}$.

% We now define the binary connectives on $\Omega$.  The set
% $\Omega\times\Omega$ has four elements: $\langle t,t\rangle ,\langle
% t,f\rangle ,\langle f,t\rangle ,\langle f,f\rangle$.  Consider
% $\langle f,f\rangle :1\to\Omega\times\Omega$ as a subobject, and let
% $B$ be its complement in $\Omega\times\Omega$.  We define
% \[ \begin{tikzcd} \Omega\times\Omega \arrow[r,"\vee"] &
%   \Omega \end{tikzcd} \] to be the characteristic function of
% $(\Omega\times\Omega )\backslash B$.  In terms of elements, $\vee
% \langle x,y\rangle = t$ iff either $x=t$ or $y=t$.  Similarly, we
% define
% \[ \begin{tikzcd} \Omega\times\Omega \arrow[r,"\wedge"] &
%   \Omega \end{tikzcd} \] to be the characteristic function of the
% subobject $\langle t,t\rangle :1\to\Omega\times\Omega$.  In other
% words, the following is a pullback
% \[ \begin{tikzcd}
% 1 \arrow[d,"{\langle t,t\rangle}"'] \arrow[r] & 1 \arrow[d,"t"] \\
% \Omega\times\Omega \arrow[r,"\wedge"'] & \Omega  
% \end{tikzcd} \]  

% %% transfer Boolean structure to characteristic functions

% \subsection{Boolean structure of $\mathrm{Sub}(X)$}

% We now show that the Boolean operations on $\Omega$ induce Boolean
% operations on characteristic functions.  In short, if a set $Y$ has an
% operation, say $g:Y\times Y\to Y$, then the set $\hom (X,Y)$ of
% functions from $X$ to $Y$ can naturally be equipped with a
% corresponding operation: $a:X\to Y$ and $b:X\to Y$, then let
% $g(a,b)=g\circ (a\times a)$, where $a\times b:X\times X\to Y\times Y$
% is the function defined by the Cartesian product.  In the particular
% case at hand, suppose that $\chi _A$ and $\chi _B$ are characteristic
% functions of subobjects of $X$.  We define $\chi _A\wedge \chi _B$ to
% be the function
% \[ \begin{tikzcd} X \arrow{r}{\langle\chi _A ,\chi _B\rangle} &
%   \Omega\times\Omega \arrow{r}{\wedge} & \Omega \end{tikzcd} \]
% Evaluating on elements, it's easy to see that $(\chi _A\wedge \chi
% _B)(x)=t$ if and only if $\chi _A(x)=t$ and $\chi _B(x)=t$.
% Similarly, we define $\chi _A\vee \chi _B$ to be the function
% \[ \begin{tikzcd} X \arrow{r}{\langle\chi _A ,\chi _B\rangle} &
%   \Omega\times\Omega \arrow{r}{\vee} & \Omega \end{tikzcd} \] Finally,
% we define $\neg \chi _A$ to be the function
% \[ \begin{tikzcd} X \arrow{r}{\chi _A} &
%   \Omega \arrow{r}{\neg} & \Omega \end{tikzcd} \]


% \begin{prop} Let $A$ and $B$ be subobjects of $X$.  Then $\chi
%   _A\wedge\chi _B$ is the characteristic function of $A\cap
%   B$. \end{prop}

% \begin{proof} By the definition of $\wedge :\Omega\times\Omega\to
%   \Omega$, the square on the right below is a pullback.
%   \[ \begin{tikzcd}
%     A\cap B \arrow{d} \arrow{r} & 1 \arrow{d}{\langle t,t\rangle} \arrow[r,"i"] & 1 \arrow{d}{t} \\
%     X \arrow{r}[below]{\langle\chi _A ,\chi _B\rangle} &
%     \Omega\times\Omega \arrow{r}[below]{\wedge} &
%     \Omega \end{tikzcd} \]
% As we discussed in the definition of $A\cap B$, the square on the left
% is also a pullback.  Since pullbacks compose to give pullbacks, the
% outer square is a pullback.  Thus, by the correspondence between
% subobjects and characteristic functions, $\wedge \langle \chi _A,\chi
% _B\rangle$ is the characteristic function of $A\cap B$.
% \end{proof}

% \begin{prop} Let $B$ be a subobject of $X$.  Then $\neg \chi _B$ is
%   the characteristic function of $X\backslash B$. \end{prop}

% \begin{proof} Consider the following diagram:
% \[ \begin{tikzcd}
% X\backslash B \arrow[d] \arrow[r] & 1 \arrow[r] \arrow[d,"f"] & 1
% \arrow[d,"t"] \\
% X             \arrow[r,"\chi _B"'] & \Omega \arrow[r,"\neg"'] & \Omega 
% \end{tikzcd} \] The square on the left is a pullback, by the
% definition of $X\backslash B$.  The square on the right is a pullback,
% by the definition of $\neg$.  Therefore, the outer square is a
% pullback, and $\neg\chi _B$ is the characteristic function of
% $X\backslash B$.
% \end{proof}



% \subsection{Relating $\sub{X}$ to $\sub{Y}$}

% Suppose that $f:X\to Y$ is a function.  We now show that $f$ induces a
% systematic relationship between subobjects of $X$ and subobjects of
% $Y$.

% For the next couple of propositions, we need to know that composing
% pullback diagrams gives a pullback diagram.  For a proof of that fact,
% see ???.

% \begin{prop} $f^{-1}$ is order-preserving, i.e.\ if $A\subseteq B$
%   then $f^{-1}(A)\subseteq f^{-1}(B)$. \end{prop}

% \begin{proof} Suppose that $n:A\to Y$ and $m:B\to Y$ are monic, and
%   that $k:A\to B$ such that $mk=n$.  Consider the following diagram:
% \[ \begin{tikzcd} f^{-1}(A) \arrow[bend right=60]{dd}[left]{n^*}
%   \arrow{r}{h} \arrow[dashed]{d}{j}
%   &  A \arrow{d}{k} \arrow[bend left=60]{dd}{n} \\
%   f^{-1}(B) \arrow{r}{g} \arrow{d}{m^*}    &  B \arrow{d}{m} \\
%   X \arrow{r}{f} & Y \end{tikzcd} \] Here $n^*$ is the pullback of
% $n=mk$ along $f$.  Hence, $mkh = fn^*$.  Since the bottom square is a
% pullback, there is a unique arrow $j:f^{-1}(A)\to f^{-1}(B)$ such that
% $m^*j=n^*$ and $gj=kh$.  Since $n$ is monic, $n^*$ is monic
% (Proposition \ref{pull-monic}), and therefore $j$ is monic.  That is,
% $f^{-1}(B)\subseteq f^{-1}(A)$. \end{proof}

% \begin{prop} $f^{-1}(A\cap B)=f^{-1}(A)\cap f^{-1}(B)$. \end{prop}

% \begin{proof}
%   %% box diagram


% \end{proof}




% \begin{prop} For any $x\in X$, $x\in X\backslash B$ if and only if
%   $x\not\in B$. \end{prop}

% \begin{proof} The idea here is simple: we have assumed that the
%   truth-value object $\Omega$. is Boolean and two-valued.  Thus, for
%   every element $x\in X$, either $\chi _B(x)=t$ or $\chi _B(x)=f$, but
%   not both.  But now for more detail:

%   Recall that $x\in B$ means that $x:1\to X$ factors through $m:B\to
%   X$.  Suppose first that $x\in X\backslash B$, so that $x$ factors
%   through $n:X\backslash B\to X$.  Then $\chi _B(x)=f$.  If $x$ also
%   factored through $B$, then we would have $\chi _B(x)=t$.  By Axiom
%   \ref{ax:classify}, $t\neq f$.  Therefore, $x\not\in B$.

%   Suppose now that $x\not\in B$.  Again by Axiom \ref{ax:classify},
%   either $\chi _B(x)=t$ or $\chi _B(x)=f$.  If $\chi _B(x)=t$, then
%   $x\in B$.  Thus, $\chi _B(x)=f$.  In the definition of $X\backslash
%   B$, the diagram is a pullback, which shows that $x:1\to X$ factors
%   through $X\backslash B$.  That is, $x\in X\backslash B$.
% \end{proof}

% By Axiom \ref{ax:classify}, there are precisely two elements
% $t,f:1\rightrightarrows \Omega$.  Thus, for any element $x:1\to X$,
% either $\chi _B(x)=t$ of $\chi _B(x)=f$.  We claim now that $\chi
% _B(x)=t$ iff $\chi _{X\backslash B}=f$.  Suppose first that $\chi
% _B(x)=t$.   
%   \[ \begin{tikzcd}
% 1\arrow[ddr, "x"'] \\ 
% &    X\backslash B \arrow{r} \arrow{d} & 1 \arrow{d}{f} \\
%     & X \arrow[r, "\ch{B}"'] & \Omega \end{tikzcd} \]
% If $x\in X\backslash B$ then 








% \begin{prop} Suppose that $A$ and $B$ are subsets of $X$.  Then $A$
%   and $B$ are disjoint iff the following is a pullback:
%   \[ \begin{tikzcd}
%     \es \arrow{r} \arrow{d} & B \arrow{d} \\
%     A \arrow{r} & X \end{tikzcd} \] \end{prop}

% \begin{proof} Suppose first that $A$ and $B$ are disjoint.  Since
%   $\alpha _X:\es\to X$ is unique, the square commutes.  Now suppose
%   that $f:C\to A$ and $g:C\to B$ such that $mf=ng$.  Suppose for
%   reductio ad absurdum that $x:1\to C$ is an element.  Then
%   $mf(x)=ng(x)$, and $A$ and $B$ are not disjoint, a contradiction.
%   Thus, $C$ has no elements.  By Proposition \ref{empty-init}, $C\cong
%   \es$, and there is a unique isomorphism $\alpha _C:C\to\es$.

%   Suppose now that the diagram is a pullback.  And suppose for
%   reductio ad absurdum that $x:1\to X$ factors both through $A$ and
%   $B$.  Using the fact that the diagram is a pullback, $x:1\to X$
%   would factor through $\es$, a contradiction.  Therefore, $A$ and $B$
%   are disjoint.
% \end{proof}



% \begin{defn} Let $A$ and $B$ be subobjects of $X$.  We define $A\cup
%   B$ to be the subobject of $X$ corresponding to the characteristic
%   function $\chi _A\vee \chi _B$.  Recall that $(\chi _A\vee \chi
%   _B)(x)=t$ if and only if either $\chi _A(x)=t$ or $\chi _B(x)=t$.
%   Thus, $x\in A\cup B$ if and only if $x\in A$ or $x\in
%   B$.  \end{defn}






% % \begin{proof} By Proposition \ref{?}, it will suffice to show that
% %   $\alpha _X$ is injective.  But $0$ has no elements (Axiom
% %  \ref{ax:zero}), and therefore is trivially injective. \end{proof}









% %% graph of a function always exists in a regular category??  See Butz

% %% \bax Suppose that $f:X\to Y$ is a function.  Then there is a set
% %% $\mathrm{Gr}(f)$, and a monomorphism $m:\mathrm{Gr}(f)\to X\times Y$
% %% such that for any element $x\in X$, 

% %% try to define coequalizers from equivalence classes






\section{Sets of functions and sets of subsets}

(Note: The following section is highly technical, and can be skipped
on a first reading.)

One distinctive feature of the category of sets is its ability to
model almost any mathematical construction.  One such construction is
gathering together old things into a new set.  For example, given two
sets $A$ and $X$, can we form a set $X^A$ of all functions from $A$ to
$X$?  Similarly, given a set $X$, can we form a set $\2P X$ of all
subsets of $X$?

As usual, we won't be interested in hard questions about what it takes
to be a set.  Rather, we're interested in hypothetical questions: if
such a set existed, what would it be like?  The crucial features of
$X^A$ seem to be captured by the following axiom:

%% TO DO: mention terminology of "Cartesian closed"

%% TO DO: Analogy with A -> B and intro - elim rules

\begin{axi}[Exponential objects]{ax:expo} Suppose that $A$ and $X$ are
  sets.  Then there is a set $X^A$, and a function $e_X:A\times X^A\to
  X$ such that for any set $Z$, and function $f:A\times Z\to X$, there
  is a unique function $f^\sharp :Z\to X^A$ such that $e_X\circ
  (1_A\times f^\sharp )=f$.  \[ \begin{tikzcd}
    A\times X^A \arrow{r}{e_X} & X  \\
    A\times Z \arrow[dashed]{u}{1_A\times f^\sharp }
    \arrow{ur}[below]{f}
  \end{tikzcd} \] The set $X^A$ is called an \emph{exponential
    object}, and the function $f^\sharp :Z\to X^A$ is called the
  \emph{transpose} of $f:A\times Z\to X$.  \label{ax:expo}
\end{axi}


The way to remember this axiom is to think of $Y^X$ as the set of
functions from $X$ to $Y$, and to think of $e:X\times Y^X\to Y$ as a
meta-function that takes an element $f\in Y^X$, and an element $x\in
X$, and returns the value $e(f,x)=f(x)$.  For this reason, $e:X\times
Y^X\to Y$ is sometimes called the \emph{evaluation function}.

Note further that if $f:X\times Z\to Y$ is a function, then for each
$z\in Z$, $f(-,z)$ is a function from $X\to Y$.  In other words, $f$
corresponds uniquely to a function from $Z$ to functions from $Y$ to
$X$.  This latter function is the transpose $f^\sharp :Z\to Y^X$ of
$f$.

We have written Axiom \ref{ax:expo} in first-order fashion, but it
might help to think of it as stating that there is a one-to-one
correspondence between two sets:
\[ \hom (X\times Z,Y) \: \cong \: \hom (Z,Y^X ) , \] where $\hom
(A,B)$ is though of as the set of functions from $A$ to $B$.  As a
particular case, when $Z=1$, the terminal object, we have 
\[ \hom (X,Y) \: \cong \: \hom (1,Y^X) .\] In other words, elements of
$Y^X$ in the ``internal sense'' correspond to elements of $\hom (X,Y)$
in the ``external sense.''

Consider now the following special case of the above construction:
\[ \begin{tikzcd}
  A\times X^A \arrow{r}{e_X} & X^A  \\
  A\times X^A \arrow[dashed]{u}{1\times e^\sharp } \arrow[ur,"e_X"']
\end{tikzcd} \] Thus, $e_X^\sharp = 1_{X^A}$.

\begin{defn} Suppose that $g:Y\to Z$ is a function.  We let
  $g^A:X^A\to Y^A$ denote the transpose of the function:
  \[ \begin{tikzcd} A\times Y^A \arrow[r,"e_Y"] & Y \arrow[r,"g"] &
    Z \end{tikzcd} \] That is, $g^A=(g\circ e_Y)^\sharp$, and the
  following diagram commutes:
\[ \begin{tikzcd}
A\times Z^A \arrow[r,"e_Z"] & Z \\
A\times Y^A \arrow[u,"1\times g^A"] \arrow[r,"e_Y"] & Y \arrow[u,"g"] 
\end{tikzcd} \]
\end{defn}

\begin{prop} Let $f:A\times X\to Y$ and $g:Y\to Z$ be functions.
  Then $(g\circ f )^\sharp =g^A\circ f^\sharp$. \end{prop}

\begin{proof} Consider the following diagram:
\[ \begin{tikzcd}
A \times Z^A \arrow[rr,"e_Z"] & & Z \\
& A\times Y^A \arrow[ul,"1\times g^A"] \arrow[dr,"e_Y"] \\
A\times X \arrow[uu,"1\times (g\circ f)^\sharp"] \arrow[ur,"1\times f^\sharp"]
\arrow[rr,"f"'] & & Y \arrow[uu,"g"] \end{tikzcd} \]
The bottom triangle commutes by the definition of $f^\sharp$.  The
upper right triangle commutes by the definition of $g^A$.  And the
outer square commutes by the definition of $(g\circ f)^\sharp$.  It
follows that 
\[ e_Z\circ (1\times (g^A\circ f^\sharp )) \: = \: g\circ f ,\]
and hence $g^A\circ f^\sharp=(g\circ f)^\sharp$. 
\end{proof}

Consider now the following particular case:
\[ \begin{tikzcd}
    A\times (A\times X)^A \arrow{r}{e} & A\times X  \\
    A\times  X \arrow[dashed]{u}{1\times p}
    \arrow[ur,"1"']
  \end{tikzcd} \] Here $p=1^\sharp$ is the unique function such that
  $e(1_A\times p)=1_{A\times X}$.  Intuitively, we can think of $p$ as
  the function that takes an element $x\in X$, and returns the
  function $p_x:A\to A\times X$ such that $p_x(a)=\langle a,x\rangle$.
  Thus, $(1\times p)\langle a,x\rangle=\langle a,p_x\rangle$, and
  $e(1\times p)\langle a,x\rangle =p_x(a)=\langle a,x\rangle$.

% TO DO: define flat

\begin{defn} Suppose that $f:Z\to X^A$ is a function.  We define
  $f^\flat :Z\times A\to X$ to be the following composite function:
\[ \begin{tikzcd}
A\times Z \arrow[r,"1\times f"] & A\times X^A \arrow[r,"e_X"] &
X \end{tikzcd} \] \end{defn}

\begin{prop} Let $f:X\to Y$ and $g:Y\to Z^A$ be functions.  Then
  $(g\circ f)^\flat = g^\flat \circ (1_A\times
  f)$.  \label{zuerst} \end{prop}

\begin{proof} By definition, 
\[ (g\circ f)^\flat = e_X\circ (1\times (g\circ f))=e_X\circ (1\times
g)\circ (1\times f)=g^\flat \circ (1\times f) .\]
\end{proof}




\begin{prop} For any function $f:A\times Z\to X$, we have $(f^\sharp
  )^\flat =f$. \end{prop}

\begin{proof} By the definitions, we have 
\[ (f^\sharp)^\flat \: = \: e_X\circ (1\times f^\sharp) = f .\]
\end{proof}

\begin{prop} For any function $f:Z\to X^A$, we have $(f^\flat )^\sharp
  = f$. \end{prop}

\begin{proof} By definition, $(f^\flat )^\sharp$ is the unique
  function such that $e_X\circ (1\times (f^\flat )^\sharp)=f^\flat$.
  But also $e_X\circ (1\times f)=f^\flat$.  Therefore,
  $(f^\flat)^\sharp=f$.
\end{proof}


% \begin{prop} Let $f:X\to Y^A$ and $g^A:Y^A\to Z^A$ be functions.  Then
%   $(g^A\circ f)^\flat = g\circ f^\flat$. \label{zwo} \end{prop}

% \begin{proof} TO DO [[needed?]]


% \end{proof}



\begin{prop} For any set $X$, we have $X^1\cong X$. \end{prop}

\begin{proof} Let $e:1\times X^1\to X$ be the evaluation function from
  Axiom \ref{ax:expo}.  We claim that $e$ is a bijection.  Recall that
  there is a natural isomorphism $i:1\times 1\to 1$.  Consider the
  following diagram:
  \[ \begin{tikzcd}
    1\times X^1 \arrow{r}{e} & X \\
    1\times 1 \arrow{u}[left]{1\times x^\sharp} \arrow{r}{i} & 1
    \arrow{u}[right]{x} \end{tikzcd} \] That is, for any element
  $x:1\to X$, there is a unique element $x^\sharp$ of $X^1$ such that
  $e(1\times x^\sharp)=x$.  Thus, $e$ is a bijection, and $X\cong
  1\times X^1$ is isomorphic to $X$.
\end{proof}

%% the following is true in any category?  no need to invoke elements?

\begin{prop} For any set $X$ we have $X^0\cong 1$. \end{prop}

\begin{proof} Elements of $X^\es$ correspond functions $\es\to X$.
  There is exactly one such function, hence $X^{\es}$ has exactly one
  element $x:1\to X^{\es}$.  Thus, $x$ is a bijection, and
  $X^{\es}\cong 1$.
\end{proof}

\begin{prop} For any sets $A,X,Y$, we have $(X\times Y)^A\cong
  X^A\times Y^A$. \end{prop}

\begin{proof} An elegant proof of this proposition would note that
  $(-)^A$ is a functor, and is right adjoint to the functor $A\times
  (-)$.  Since right adjoints preserve products, $(X\times Y)^A\cong
  X^A\times Y^A$.  Nonetheless, we will go into further detail.

  By uniqueness of Cartesian products, it will suffice to show that
  $(X\times Y)^A$ is a Cartesian product of $X^A$ and $Y^A$, with
  projections $\pi _0^A$ and $\pi _1^A$.  Let $Z$ be an arbitrary set,
  and let $f:Z\to X^A$ and $g:Z\to Y^A$ be functions.  Now take
  $\gamma = \langle f^\flat ,g^\flat \rangle ^\sharp$, where
  $f^\flat:A\times Z\to X$ and $g^\flat:A\times Z\to Y$.
  \[ \begin{tikzcd} & Z \arrow[bend right,ddl,"f"'] \arrow[bend
    left,ddr,"g"]
    \arrow[dashed,d,"\gamma"] \\
    & (X\times Y)^A \arrow[dl,"\pi _0^A"] \arrow[dr,"\pi _1^A"'] \\
    X^A & & Y^A \end{tikzcd} \] We claim that $\pi _0^A\gamma =f$ and
  $\pi _1^A\gamma = g$.  Indeed,
  \[ \pi _0^A\circ \gamma = \pi _0^A \circ \langle f^\flat ,g^\flat
  \rangle ^\sharp = (\pi _0\circ \langle f^\flat ,g^\flat \rangle
  )^\sharp = (f^\flat )^\sharp = f .\] Thus, $\pi _0^A\gamma =f$, and
  similarly, $\pi _1^A\gamma =g$.

Suppose now that $h:Z\to (X\times Y)^A$ such that $\pi _0^Ah=f$ and
$\pi _1^Ah=g$.  Then 
\[ f = \pi _0^A\circ (h^\flat )^\sharp = (\pi _0\circ h^\flat )^\sharp
.\] Hence, $\pi _0\circ h^\flat=f^\flat$, and similarly, $\pi _1\circ
h^\flat=g^\flat$.  That is, $h^\flat = \langle f^\flat
,g^\flat\rangle$, and $h=\langle f^\flat ,g^\flat \rangle^\sharp
=\gamma$.
\end{proof}


\begin{prop} For any sets $A,X,Y$, we have $A\times (X\cpr Y)\cong
  (A\times X)\cpr (A\times Y)$. \end{prop}

\begin{proof} Even without Axiom \ref{ax:expo}, there is always a
  canonical function from $(A\times X)\cpr (A\times Y)$ to $A\times
  (X\cpr Y)$, namely $\phi :=(1_A\times i_0)\cpr (1_A\times i_1)$, where $i_0$
  and $i_1$ are the coproduct inclusions of $X\cpr Y$.  That is,
  \[ \phi \circ j_0=1_A\times i_0,\qquad \text{and} \qquad \phi\circ
  j_1=1_A\times i_1 ,\] where $j_0$ and $j_1$ are the coproduct
  inclusions of $(A\times X)\cpr (A\times Y)$. 

\[ \begin{tikzcd}
 & X\cpr Y  \\
X \arrow[ur,"i_0"]  &  A\times (X\cpr Y) \arrow[u,"p_1"] &  Y
\arrow[ul,"i_1"'] \\
A\times X \arrow[ur,"1_A\times i_0"] \arrow[r,"j_0"'] \arrow[u,"q_1"]  &  (A\times X)\cpr (A\times
Y) \arrow[dashed,u,"\phi"] & A\times Y
\arrow[l,"j_1"] \arrow[ul,"1_A\times i_1"'] \arrow[u,"r_1"'] \
\end{tikzcd} \] We will show that Axiom \ref{ax:expo} entails that
$\phi$ is invertible.  

Let $g:A\times (X\cpr Y)\to A\times (X\cpr Y)$ be the identity, i.e.\
$g=1_{A\times (X\cpr Y)}$.  Then $g^\sharp :X\cpr Y\to (A\times (X\cpr
Y))^A$ is the unique function such that $e(1_A\times g^\sharp)=g$.  By
Proposition \ref{zuerst},
\[ (g^\sharp \circ i_0)^\flat = g\circ (1_A\times i_0) = 1_A\times i_0
.\]  Similarly, $(g^\sharp\circ i_1)^\flat = 1_A\times i_1$.  Thus,
\[ g^\sharp \: = \: (1_A\times i_0 )^\sharp \cpr  (1_A\times i_1)^\sharp
.\] We also have $(1_A\times i_0)^\sharp = (\phi\circ j_0 )^{\sharp} =
\phi ^A\circ j_0^\sharp$, and $(1^A\times i_1)^\sharp = \phi ^A\circ
j_1^\sharp$.  Hence
\[ g^\sharp \: = \: (\phi ^A\circ j_0^\sharp ) \cpr  (\phi ^A\circ
j_1^\sharp ) \: = \: \phi ^A\circ (j_0^\sharp \cpr j_1^\sharp ) .\] Now
for the inverse of $\phi$, we take $\psi = (j_0^\sharp\cpr j_1^\sharp
)^\flat$.
\[ \begin{tikzcd}
& ((A\times X)\cpr (A\times Y))^A \\
& X\cpr Y \arrow[dashed,u,"j_0^\sharp\cpr j_1^\sharp"] \\
X \arrow[ur,"i_0"'] \arrow[uur,"j_0^\sharp"] & & Y
\arrow[uul,"j_1^\sharp"'] \arrow[ul,"i_1"] \end{tikzcd} \] It then follows that
\[ (\phi \circ \psi )^\sharp \:=\: \phi ^A\circ (j_0^\sharp
\cpr j_1^\sharp ) \: = \: g^\sharp , \] and therefore $\phi\circ\psi
=1_{A\times (X\cpr Y)}$.  Similarly, \[ (\psi\circ\phi\circ
j_0)^\sharp=\psi ^A\circ (\phi \circ j_0)^\sharp = \psi ^A\circ
g^\sharp\circ i_0 = \psi^\sharp\circ i_0= j_0^\sharp . \] Thus,
$\psi\circ\phi\circ j_0=j_0$, and a similar calculation shows that
$\psi\circ\phi\circ j_1=j_1$.  It follows that $\psi\circ\phi
=1_{(A\times X)\cpr (A\times Y)}$.  Thus, $\psi$ is a two-sided inverse
for $\phi$, and $A\times (X\cpr Y)$ is isomorphic to $(A\times X)\cpr (A\times
Y)$. \end{proof}

% Since $X^A$ is the set of functions from $A$ to $X$, if we have a
% function $f:B\to A$, then precomposition with $f$ should give a
% function from $X^A$ to $X^B$.

% \begin{defn} Let $f:B\to A$ be a function.  We define a function
%   $X^f:X^B\to X^A$ as follows.  First consider the composite function
%   $e_X\circ (f\times 1)$:
%   \[ \begin{tikzcd} B\times X^A \arrow[r,"f\times 1"] & A\times X^A
%     \arrow[r,"e_X"] & X
% \end{tikzcd} \] Then apply the sharp operation to obtain a function
% $X^f:X^A\to X^B$.
% \end{defn}

% \begin{prop} Suppose that $g:Y\to X^A$ and $X^f:X^A\to X^B$.  Then
%   $(X^f\circ g)^\flat = g^\flat \circ (f\times
%   1_Y)$. \label{zweit} \end{prop}

% TO DO


%% Question: given an endomorphism $f:X\to X$, can we define a
%% subobject of $X^A$, those 'functions' that commute with $f$?  As a
%% specific example, can we define the set of fixed points of $f$?

\newcommand{\sub}[1]{\mathrm{Sub}(#1)}

\begin{defn}[Powerset] If $X$ is a set, we let $\2P X =\Omega
  ^X$.  \end{defn}

Intuitively speaking, $\2P X$ is the set of all subsets of $X$.  For
example, if $X=\{ a,b\}$, then $\2P X=\{ \emptyset ,\{a \} ,\{ b\} ,\{
a,b\} \}$.  More rigorously, each element of $\Omega ^X$ corresponds
to a function $1\to \Omega ^X$, which in turn corresponds to a
function $X\cong 1\times X\to\Omega$, which corresponds to a subobject
of $X$.  Thus, we can think of $\2P X$ as another name for $\sub{X}$,
although $\sub{X}$ is not really an object in $\cat{Sets}$.

% It is easy to verify, however, that the structure we've defined on
% $\sub{X}$ is internally definable on $\Omega ^X$.  For example, recall
% that the intersection operation $A,B\mapsto A\cap B$ was defined in
% terms of the composite $\wedge \langle \chi _A,\chi _B\rangle$.
% Similarly, we define intersection on $(\Omega ^X\times \Omega ^X)\cong
% (\Omega\times\Omega )^X$ as follows: TO DO







\section{Cardinality}

%% TO DO: define infinite

{\it Summary:} When mathematics was rigorized in the 19th century, one
of the important advances was a rigorous definition of ``infinite
set.''  It came as something of a suprise that there are different
sizes of infinity, and that some infinite sets (e.g.\ the real
numbers) are strictly larger than the natural numbers.  In this
section, we define ``finite'' and ``infinite.''  We then add an axiom
which says there is a specific set $N$ that behaves like the natural
numbers; in particular, $N$ is infinite.  Finally, we show that the
powerset $\2P X$ of a set $X$ is always larger than $X$.

\begin{defn} A set $X$ is said to be \emph{finite} if and only if for
  any function $m:X\to X$, if $m$ is monic, then $m$ is an
  isomorphism.  A set $X$ is said to be \emph{infinite} if and only if
  there is a function $m:X\to X$ that is monic and not
  surjective. \end{defn}

We are already guaranteed the existence of finite sets: for example,
the terminal object $1$ is finite, as is the subobject classifier
$\Omega$.  But the axioms we have stated thus far do not guarantee the
existence of any infinite sets.  We won't know that there are infinite
sets until we add the ``natural number object'' axiom below.

% \begin{prop} $X$ is infinite if and only if there is a function
%   $g:X\to X$ that is surjective but not injective. \end{prop}

% %% do we know yet that we can define functions such as $g'$ below?

% \begin{proof} Suppose that $X$ is infinite, that is, there a function
%   $f:X\to X$ that is injective but not surjective.  Then there is a
%   surjective function $g:f(X)\to X$.  Pick an arbitrary point $x_0\in
%   X$, and define $g':X\to X$ by letting $g'$ agree with $g$ on $f(X)$,
%   and $g'(x)=x_0$ for $x\in X\backslash f(X)$.  Then $g'$ is
%   surjective since $g:f(X)\to X$ is surjective.  But $g'$ is not
%   injective.  \end{proof}

% \begin{prop} If $X$ is infinite then $X\backslash \{ x\}$ is infinite,
%   for any $x\in X$.  \end{prop}

% \begin{proof} Let $X$ be infinite, and let $f:X\to X$ be injective but
%   not surjective.  Fix $x_0\in X$.  Since $f$ is not surjective, there
%   is a $y\in X\backslash f(X)$.  We may, in fact, suppose that $y\neq
%   x_0$; because there is a permutation $p$ of $X$ such that
%   $p(x_0)\neq x_0$ and $p\circ f$ is also injective but not
%   surjective.

%   If $f(y)$ were in the image of $f\circ f$, then we would have
%   $f(y)=f(f(z))$ for some $z$, and since $f$ is injective $y=f(z)$,
%   contradicting the fact that $y$ is not in the image of $f$.
%   Therefore, neither $y$ nor $f(y)$ is in the image of $f\circ f$.
%   Moreover, $y\neq f(y)$ since $y$ is not in the image of $f$.

%   Let $g$ be $f\circ f$ restricted to $X\backslash\{ x_0\}$.  If $x_0$
%   is in the image of $g$, then replace $g$ with $q\circ g$, where
%   $q:X\to X$ is a permutation that switches $x_0$ and $f(y)$.  Call
%   the resulting function $g$ again, but now think of $g$ as a function
%   from $X\backslash\{ x_0\}$ to $X\backslash\{ x_0\}$.  Then $g$ is
%   injective, but $y$ is not in the image of $g$.  \end{proof}

% \begin{cor} If $X$ is infinite then $X\backslash\{ x_1,\dots ,x_n\}$
%   is infinite. \end{cor}

% \begin{exercise} Use mathematical induction to prove the
%   corollary. \end{exercise}


\begin{defn} We say that $Y$ is at least as large as $X$, written
  $|X|\leq |Y|$, just in case there is a monomorphism $m:X\to
  Y$.  \end{defn}

\begin{prop} $|X|\leq |X\cpr Y|$. \end{prop}

\begin{proof} Proposition \ref{coprod-monic} shows that $i_0:X\to
  X\cpr Y$ is monic.  \end{proof}

\begin{prop} If $Y$ is non-empty, then $|X|\leq |X\times
  Y|$. \end{prop}

\begin{proof} Consider the function $\langle 1_X,f\rangle :X\to
  X\times Y$, where $f:X\to 1\to Y$. \end{proof}






\begin{axi}[Natural Number Object]{ax:nno} There is an object $N$, and
  functions $z:1\to N$ and $s:N\to N$ such that for any other set $X$
  with functions $q:1\to X$ and $f:X\to X$, there is a unique function
  $u:N\to X$ such that the following diagram commutes:
\[ \begin{tikzcd} 
1 \arrow[r,"z"] \arrow[rd,"q"'] & N \arrow[r,"s"] \arrow[d,"u"] & N \arrow[d,"u"] \\
 & X \arrow[r,"f"] & X 
\end{tikzcd} \] The set $N$ is called a \emph{natural number
  object}. \label{ax:nno} \end{axi}

\begin{exercise} Let $N'$ be a set, and let $z':1\to N'$ and $s':N'\to
  N'$ be functions that satisfy the conditions in the axiom above.
  Show that $N'$ is isomorphic to $N$. \end{exercise}

% Let's consider some of the properties of these functions $1\to N\to
% N$.  First of all, note that $s$ cannot be the identity $1_N$ of $N$.
% Let $t:1\to\Omega$ be the true element, and let $\neg
% :\Omega\to\Omega$ be negation, so that $\neg t=f$.  Then we are
% promised a unique function $u:N\to \Omega$ such that the following
% diagram commutes:
% \[ \begin{tikzcd} 
% 1 \arrow[r,"z"] \arrow[d] & N \arrow[r,"s"] \arrow[d,"u"] & N \arrow[d,"u"] \\
% 1 \arrow[r,"t"] & \Omega \arrow[r,"\neg"] & \Omega 
% \end{tikzcd} \] If $s=1_N$, then $\neg u=u$, which means that the
% subobject corresponding to $u$ is its own complement.  In this case,
% we would have $N=0$, contradicting the fact that $z:1\to N$.


% Let $X_4=1\cpr 1\cpr 1\cpr 1$, with its four distinct coprojections $i_j:1\to
% X_4$.  Since $X_4$ has four elements, we will write $X_4=\{
% 0,1,2,3\}$.  By using the properties of the coproduct, it's clear that
% there is a function $f:X_4\to X_4$ with the feature that $f(x)=x\cpr 1$
% for $x<3$, and $f(3)=3$.  Now consider the diagram:
% \[ \begin{tikzcd}
%   & N \arrow[r,"s"] \arrow[dd,"u"] & N \arrow[dd,"u"] \\
%   1 \arrow[ur,"z"] \arrow[dr,"i_0"'] \\
%   & X_4 \arrow[r,"f"] & X_4 \end{tikzcd} \] Thus, $u(z)=0$,
% $u(s(z))=f(u(z))=1$, etc..  Clearly, $u$ is surjective. 




\begin{prop} $z\cpr s:1\cpr N\to N$ is an
  isomorphism. \label{nno-coproduct}
\end{prop}

\begin{proof} Let $i_0:1\to 1\cpr N$ and $i_1:N\to 1\cpr N$ be the
  coproduct inclusions.  Using the NNO axiom, there is a unique
  function $g:N\to 1\cpr N$ such that the following diagram commutes:
  \[ \begin{tikzcd}[column sep=large] 1 \arrow[r,"z"]
    \arrow[dr,"i_0"'] & N \arrow[r,"s"] \arrow[d,"g"] & N
    \arrow[d,"g"] \\
    & 1\cpr N \arrow[r,"i_1z\cpr i_1s"'] & 1\cpr N \end{tikzcd} \] We will
  show that $g$ is a two-sided inverse of $z\cpr s$.  To this end,
  we first establish that $g\circ s=i_1$.  Consider the following
  diagram:
  \[ \begin{tikzcd}
    & N \arrow[r,"s"] \arrow[d,"s"] & N \arrow[d,"s"] \\
    1 \arrow[ur,"z"] \arrow[r,"sz"] \arrow[dr,"i_1z"'] & N
    \arrow[d,"g"]
    \arrow[r,"s"] & N \arrow[d,"g"] \\
    & 1\cpr N \arrow[r,"i_1z\cpr i_1s"] & 1\cpr N \end{tikzcd} \] The lower
  triangle commutes because of the commutativity of the previous
  diagram.  Thus, the entire diagram commutes.  The outer triangle and
  square would also commute with $i_1$ in place of $g\circ s$.  By the
  NNO axiom, $g\circ s=i_1$.  Now, to see that $(z\cpr s)\circ
  g=1_N$, note first that
\[ (z\cpr s)\circ g\circ z = (z\cpr s)\circ i_0 = z .\]
Furthermore,
\[ (z\cpr s)\circ g\circ s = (z\cpr s)\circ i_1 = s .\] Thus the
NNO axiom entails that $(z\cpr s)\circ g=\mathrm{id}_N$.  Finally,
to see that $g\circ (z\cpr s)=\mathrm{id}_{1\cpr N}$, we calculate:
\[ g\circ (z\cpr s)\circ i_0 = g\circ z = i_0 .\]
Furthermore, 
\[ g\circ (z\cpr s)\circ i_1 = g\circ s = i_1 .\] Therefore, $g\circ
(z\cpr s)=\mathrm{id}_{1\cpr N}$.  This establishes that $g$ is a
two-sided inverse of $z\cpr s$, and $1\cpr N$ is isomorphic to $N$.
\end{proof}


\begin{prop} The function $s:N\to N$ is injective, but not surjective.
  Thus, $N$ is infinite. \label{succ-monic} \end{prop}

\begin{proof} By Proposition \ref{coprod-monic}, the function
  $i_1:N\to 1\cpr N$ is monic.  Since the images of $i_0$ and $i_1$
  are disjoint, $i_0$ is not surjective.  Since $z\cpr s$ is an
  isomorphism, $(z\cpr s)\circ i_1=s$ is monic, but not surjective.
  Therefore, $N$ is infinite.
\end{proof}




% [[The Peano axioms]]

% The definition of $N$ includes the zero element $0\in N$.  

% We are now going to define addition of natural numbers as a function
% $+:N\times N\to N$.  Consider the following diagram:
% \[ \begin{tikzcd} 1 \arrow[r,"0"] \arrow[dr,"1_N^\sharp"'] & N
%   \arrow[r,"s"]
%   \arrow[d,"g"] & N \arrow[d,"g"] \\
%   & N^N \arrow[r,"N^s"] & N^N \end{tikzcd} \] Intuitively, $g$ takes a
% natural number $n$, and produces a function $g_n:N\to N$ such that
% $g_n(x)=(s\circ \cdots \circ s)(x)=x+n$.  [Since $s$ is monic, so is
% $g_n$.]  Thus,
% \[ g^\flat (x,y) = g_x(y) = x+y .\] 
% More rigorously, if we let $p=g^\flat$, then we have:

% % \begin{prop} $p\circ (s\times 1_N)=s\circ p$ and $p\circ (1_N\times
% %   s)=s\circ p$.  \end{prop}

% % \begin{proof} By Propositions \ref{zuerst} and \ref{zweit}, we have
% % \[ g^\flat \circ (1_N\times s) = (g\circ s)^\flat =(N^s\circ g)^\flat
% % = (s^N\circ g)^\flat =   .\]
% % \end{proof}

% % \begin{exercise} Show that $+$ is commutative.  Show that $+$ is the
% %   unique function from $N\times N$ to $N$ such that $0+0=0$, and both
% %   $s(x)+y=s(x+y)$ and $x+s(y)=s(x+y)$.  [Warning: this is more of a
% %   project than an exercise.] \end{exercise}

% A similar maneuver can be used to define multiplication as a function
% $m:N\times N\to N$.  However, for now, we merely note that
% multiplication by $2$ can be defined as the composition of $+:N\times
% N\to N$ with the diagonal $\delta _N:N\to N\times N$.

% \begin{prop} Let $f:X\to N$ be a function such that $0=f(x)$, for some
%   $x\in X$, and for each $n\in N$, $s(n)=f(y)$ for some $y\in Y$.
%   Then $f$ is an epimorphism. \end{prop}

% \begin{proof} Let $B=f(X)$ be the image of $X$ in $N$. The hypotheses
%   say that $0\in B$, and that if $x\in B$, then $s(x)\in B$.
%   Therefore, $B\cong N$.
% \end{proof}

% \begin{prop} $N\cpr N\cong N$.  In particular, $N\cpr N$ is
%   countable.  \end{prop}

% \begin{proof}[Sketch of proof] Consider the map $f:N\to N$ that takes
%   each natural number $x$ to $2x$, and consider the map $g:N\to N$
%   that takes each natural number $x$ to $2x+1$.  Then $f\cpr
%   g:N\cpr N\to N$ is an isomorphism.
% \end{proof}

% TO DO: integers, rational numbers, real numbers --- uncountably
% infinite



\begin{prop} If $m:B\to X$ is a nonempty subobject, then there is an
  epimorphism $f:X\to B$.  \label{retraction} \end{prop}

\begin{proof} Since $B$ is nonempty, there is a function
  $g:X\backslash B\to B$.  By Proposition \ref{xmid}, $B\cong B\cpr
  X\backslash B$.  Finally, $1_B\cpr g:B\cpr X\backslash B\to B$ is an
  epimorphism, since $1_B$ is an epimorphism. \end{proof}

% TO DO: Pullback of a monic is monic.  In any category, if $f$ and $g$
% are monic, then $f\times g$ is monic.  3. Are the projections $\pi _0$
% and $\pi _1$ epi?

% \begin{prop} If $X$ is infinite, then $X\times X$ is
%   infinite. \end{prop}

% \begin{proof} Suppose that $m:X\to X$ is monic but not surjective.
%   Then $m\times m$ is monic.  We claim that $m\times m$ is not
%   epi.  


% \end{proof}

\begin{defn} We say that a set $X$ is \emph{countable} just in case
  there is an epimorphism $f:N\to X$, where $N$ is the natural
  numbers. \end{defn}

\begin{prop} $N\times N$ is countably infinite. \end{prop}

\begin{proof}[Sketch of proof] We will give two arguments: one quick,
  and one slow (but hopefully more illuminating).  For the quick
  argument, define a function $g:N\times N\to N$ by $g(x,y)=2^x3^y$.
  If $\langle x,y\rangle \neq \langle x',y'\rangle$, then either
  $x\neq x'$ or $y\neq y'$.  In either case, unique factorizability of
  integers gives $2^x3^y\neq 2^{x'}3^{y'}$.  Therefore, $g:N\times
  N\to N$ is monic.  Since $N\times N$ is nonempty, Proposition
  \ref{retraction} entails that there is an epimorphism $f:N\epi
  N\times N$.  Therefore, $N\times N$ is countable.

  %% TO DO: Do we need Cantor Bernstein here?  Can we prove
  %% Cantor-Bernstein in ETCS?
  %% Or perhaps just show: if $N ->> X$, then either $X$ is finite, or
  %% $X$ is countable.

%% perhaps use recursion to define the following:

  Now for the slow argument.  Imagine writing down all elements in
  $N\times N$ in an infinite table, whose first few elements look like
  this:
\[ \left( \begin{array}{ccccc} \langle 0,0\rangle & \langle 1,0\rangle
    & \langle 2,0\rangle & \cdots
    \\
    \langle 0,1\rangle & \langle 1,1\rangle & \langle 2,1\rangle &
    \cdots
    \\
    \langle 0,2\rangle & \langle 1,2\rangle & \langle 2,2\rangle &
    \cdots
    \\
    \vdots & \vdots & \vdots & \vdots \end{array} \right. \] Now
imagine running a thread diagonally through the numbers: begin with
$\langle 0,0\rangle$, then move down to $\langle 1,0\rangle$ and up to
$\langle 0,1\rangle$, then over to $\langle 2,0\rangle$ and down its
diagonal, etc.. This process defines a function $f:N\to N\times N$
whose first few values are
\[ \begin{array}{lll}
  f(0) = \langle 0,0\rangle \\
  f(1) = \langle 0,1\rangle \\
  f(2) = \langle 1,0\rangle \\
  \vdots \end{array} \] It is not difficult to show that $f$ is
surjective, and so $N\times N$ is countable. \end{proof}

\begin{exercise} Show that if $A$ and $B$ are countable then $A\cup B$
  is countable. \end{exercise}


We're now going to show that exponentiation allows us to construct
sets of larger and larger size.  In the case of finite sets $A$ and
$X$, it's easy to see that the following equation holds:
\[ |A^X| \: = \: |A|^{|X|} ,\] where $|X|$ denotes the number of
elements in $X$.  In particular, $\Omega ^X$ can be thought of as the
set of binary sequences indexed by $X$.  We're now going to show that
for any set $X$, the set $\Omega ^X$ is larger than $X$.

\begin{defn} Let $g:A\to A$ be a function.  We say that $a\in A$ is a
  \emph{fixed point} of $g$ just in case $g(a)=a$.  We say that $A$
  has the \emph{fixed point property} just in case any function
  $g:A\to A$ has a fixed point. \end{defn}

\begin{prop} Let $A$ and $X$ be sets.  If there is a surjective
  function $p:X\to A^X$, then $A$ has the fixed point
  property. \label{cantor} \end{prop}

\begin{proof} Suppose that $p:X\to A^X$ is surjective.  That is, for
  any function $f:X\to A$, there is an $x_f\in X$ such that
  $f=p(x_f)$.  Let $\vp =p^\flat$, so that $f=\vp (x_f,-)$.  Now let
  $g:A\to A$ be any function.  We need to show that $g$ has a fixed
  point.  Consider the function $f:X\to A$ defined by $f=g\circ \vp
  \circ \delta _X$, where $\delta _X:X\to X\times X$ is the diagonal
  map.  Then we have
  \[ g\vp (x,x) \: = \: f(x) \: = \: \vp (x_f,x) ,\] for all $x\in X$.
  In particular, $g\vp (x_f,x_f)=\vp (x_f,x_f)$, which means that
  $a=\vp (x_f,x_f)$ is a fixed point of $g$.  Since $g:A\to A$ was
  arbitrary, it follows that $A$ has the fixed point property.
\end{proof}

\begin{prop} There is no surjective function $X\to \Omega
  ^X$. \end{prop}

%% TO DO: what does this tell us about cardinality?

\begin{proof} The function $\Omega\to \Omega$ that permutes $\tr$ and
  $\f$ has no fixed points.  The result then follows from Proposition
  \ref{cantor}. \end{proof}

\begin{exercise} Show that there is an injective function $X\to \Omega
  ^X$.  [The proof is easy if you simply think of $\Omega ^X$ as
  functions from $X$ to $\{ \tr ,\f\}$.  For a bigger challenge, try
  to prove that it's true using the definition of the exponential set
  $\Omega ^X$.]  \end{exercise}




\begin{cor} For any set $X$, the set $\2P X$ of its subsets is
  strictly larger than $X$. \end{cor}


There are several other facts about cardinality that are important for
certain parts of mathematics --- in our case, they will be important
for the study of topology.  For example, if $X$ is an infinite set,
then the set $\2F X$ of all finite subsets of a set $X$ has the same
cardinality as $X$.  Similarly, a countable coproduct of countable
sets is countable.  However, these facts --- well known from ZF set
theory --- are not obviously provable in ETCS.

\begin{disc} Intuitively speaking, $X^N$ is the set of all sequences
  with values in $X$.  Thus, we should have something like
  \[ X^N \: \cong \: X\times X\times \cdots \] However, we don't have
  any axiom telling us that $\cat{Sets}$ has infinite products such as
  the one on the right hand side above.  Can it be proven that $X^N$
  satisfies the definition of an infinite product?  In other words,
  are there projections $\pi _i:X^N\to X$ which satisfy an appropriate
  universal property? \end{disc}






%% conjecture: N is the colimit of the X_i

%% conjecture: if $X$ is infinite, then there is a monic $f:N\to X$.


%% TO DO: define set of all finite subsets of a set ... initial
%% segments of $N$
%% conjecture: every finite set is isomorphic to 1+...+1.



% Some sets are so big that they are not countable.  For example, a
% famous proof by Georg Cantor shows that the set $\mathbb{R}$ of real
% numbers is uncountable.  The proof goes something like this: consider
% the set $S$ of all numbers of the form $0.a_1a_2a_3\cdots $ where each
% $a_i$ is equal to $0$ or $1$.  These numbers all lie in the interval
% $[0,1]$ in $R$.  If we can show that there is no surjection $N\to S$,
% then there is no surjection $N\to R$.  Suppose for reductio ad
% absurdum that there is a surjection $s:\mathbb{N}\to S$.  So, for each
% $i\in \mathbb{N}$, $s_i$ is an element of $S$.  We let $s_{ij}$ denote
% the $j$-th term in the decimal expansion of $s_i$.  We claim that
% there is some $z\in S$ that is not in the image of $s$.  Indeed,
% define
% $$ z = 0.\overline{s}_{11}\overline{s}_{22}\overline{s}_{33}\cdots ,$$
% where $\overline{s}_{ii}$ is the opposite of $s_{ii}$.  Then $z\neq
% s_i$ for all $i$, since the $i$-th term of $z$ doesn't equal the
% $i$-th term of $z_i$.  



% \begin{prop} Let $m:B\to N$ be a subobject such that $a\in B$, and
%   $s(B)\subseteq B$.  Then $m$ is an isomorphism. \end{prop}

% \begin{proof} Suppose that $a\in B$, and $s(B)\subseteq B$.  Since
%   $a\in B$, $a:1\to N$ factors uniquely through $m:B\to N$.  Since
%   also $s(B)\subseteq B$, there is a function $g:B\to B$ such that the
%   bottom square and triangle commute.  [Here $g$ is simply the
%   restriction of $s$ to $B$.]
%   \[ \begin{tikzcd}
%     & N \arrow[r,"s"] \arrow[d,"u"] & N \arrow[d,"u"] \\
%     1 \arrow[ur,"a"] \arrow[dr,"a"'] \arrow[r] & B \arrow[r,"g"]
%     \arrow[d,"m"] &
%     B \arrow[d,"m"] \\
%     & N \arrow[r,"s"] & N \end{tikzcd} \] By the NNO axiom, there is a
%   $u:N\to B$ that makes the top triangle and square commute.  Invoking
%   the NNO axiom again, there is a unique $f:N\to N$ such that $fa=a$
%   and $fs=sf$.  Since both $mu$ and $1_N$ have this feature, $mu=1_N$.
%   Thus, $m$ is a split epimorphism.  Since $m$ is also monic, $m$ is
%   an isomorphism. \end{proof}

% \begin{prop} Suppose that $f,g:N\rightrightarrows X$ such that
%   $fa=ga$, and for all $x\in N$, $(fx=gx\Rightarrow f(sx)=g(sx))$.
%   Then $f=g$. \label{supervene} \end{prop}

% \begin{proof} Suppose that $fa=ga$, and for all $x\in N$,
%   $(fx=gx\Rightarrow f(sx)=g(sx))$.  Let $m:B\to N$ be the equalizer
%   of $f$ and $g$.  Since $fa=ga$, we have $a\in B$.  If $y\in S(B)$,
%   then $y=sx$, with $x\in B$.  Thus, $fx=gx$.  By assumption,
%   $f(sx)=g(sx)$, hence $y=sx\in B$.  Thus, $S(B)\subseteq B$, and the
%   previous proposition entails that $m$ is an isomorphism.  Therefore,
%   $f=g$.  \end{proof}






%% TO DO: construct rational numbers, real numbers









\section{The axiom of choice}

The axioms we've given so far correspond roughly to what's typically
called Zermelo-Fraenkel (ZF) set theory.  In recent years it has
become routine to supplement ZF with a further axiom, the so-called
{\it axiom of choice}.  (The axiom of choice is regularly used in
fields such as functional analysis, e.g.\ to prove the existence of an
orthonormal basis for Hilbert spaces of arbitrarily large dimension.)
While the name of this axiom suggests that it has something to do with
our choices, in fact it really just asserts the existence of further
sets.  Following our typical procedure in this chapter, we will
provide a structural version of the axiom.

\begin{defn} Let $f:X\to Y$ be a function.  We say that $f$ is a
  \emph{split epimorphism} just in case there is a function $s:Y\to X$
  such that $fs=1_Y$. In this case, we say that $s$ is a
  \emph{section} of $f$. \end{defn}

\begin{exercise} Prove that if $f$ is a split epimorphism, then $f$ is
  a regular epimorphism.  Prove that if $s$ is a section, then $s$ is
  a regular monomorphism. \end{exercise}

\begin{axi}[Axiom of choice]{ax:choice}
  Every epimorphism in $\cat{Sets}$ has a section. \end{axi}

A more typical formulation of the axiom of choice might say that for
any set-indexed collection of sets, say $\{ X_i\mid i\in I\}$, the
product set $\prod _{i\in I}X_i$ is non-empty.  To translate that
version of the axiom of choice into our version, suppose that the sets
$X_i$ are stacked side by side, and that $f$ is the map that projects
each $x\in X_i$ to the value $i$.  Then a section $s$ of $f$ is a
function with domain $I$ that returns an element $s(i)\in X_i$ for
each $i\in I$.  If such a function exists, then $\prod _{i\in I}X_i$
is non-empty.

In this book, we will never have to invoke the full axiom of choice.
However, we will use a couple of weaker versions of it, specifically
in the proofs of the completeness theorems for propositional and
predicate logic.  For propositional logic, we will assume the Boolean
prime ideal theorem; and for predicate logic, we will use a version of
the axiom of dependent choices to prove the Baire category theorem.



\section{Notes}

\begin{itemize}
\item There are many good books on category theory.  The classic
  reference is \citep{cwm}, but it can be difficult going for
  philosophers without extensive mathematical training.  We also find
  the following useful: \citep{borceux,awodey-book,vanoosten}.  The
  latter two are good entry points for philosophers with some previous
  training in formal logic.
\item The elementary theory of the category of sets (ETCS) was first
  presented by \cite{lawvere-etcs}.  For pedagogical presentations,
  see \cite{lawvere-sets,leinster}.
\end{itemize}


%% to do: proof by induction




% \begin{prop} $X$ is non-empty if and only if $X\to 1$ is
%   epi. \label{el-epi} \end{prop}

% \begin{proof} ($\Rightarrow$) Let $x:1\to X$, and let $f:X\to 1$ be
%   the unique function.  Then $fx:1\to 1$ is an isomorphism, hence epi.
%   Therefore, $f$ is epi.

%   ($\Leftarrow$) Suppose that $\eta _X:X\to 1$ is epi.  By the Axiom
%   of choice, $\eta _X$ has a section $x:1\to X$.  Therefore, $X$ is
%   non-empty. \end{proof}



%% is Sets a regular category?  must show that regular epis are stable
%% under pullback.  Would suffice to show that all epis are regular!



















%% TO DO: image factorization -- regular epimorphisms


%% TO DO: 0 is unique without elements

%% TO DO: Every monic (from nonempty) has a retraction

%% TO DO: balanced -- monic + epi => iso

















% \begin{facts} Given a function $f:X\to Y$, and a subset $A\subseteq
%   X$, we define
%   \[ f(A) \:= \: \{ y\in Y \mid \exists x\in A.f(x)=y \} .\] Given a
%   subset $B\subseteq Y$, we define
% \[ f^{-1}(B) \: = \: \{ x\in X \mid \exists y\in B.f(x)=y \} .\]
% Then the following hold of $f^{-1}$ and $f$. \end{facts}

% \begin{enumerate}

% \item $f^{-1}$ is order preserving: if $A\subseteq B$ then
%   $f^{-1}(A)\subseteq f^{-1}(B)$.

%   Proof: Let $x\in f^{-1}(A)$.  Then $f(x)\in A\subseteq B$.
%   Therefore $x\in f^{-1}(B)$.

% \item $A\subseteq f^{-1}(f(A))$.

% Let $x\in A$.  Then $x\in f^{-1}\{ f(x) \}\subseteq f^{-1}(f(A))$.
% Thus, $x\in f^{-1}(f(A))$.  

% The inclusion is not reversible.  For example, let $f:2\to 1$, and let
% $A$ be a singleton subset of $2$.  Then $f(A)=1$, and hence
% $f^{-1}(f(A))=2$.

% \item If $f$ is injective, then $f^{-1}(f(A))=A$.

%   Let $x\in f^{-1}(f(A))$.  Then there is some $y\in A$ such that
%   $f(x)=f(y)$.  Since $f$ is injective, $x=y$, and $x\in A$.

% \item $f(f^{-1}(B))\subseteq B$.

%   Let $y\in f(f^{-1}(B))$.  Thus, there is an $x\in f^{-1}(B)$ such
%   that $y=f(x)$.  Since $x\in f^{-1}(B)$, $y=f(x)\in B$.

%   The inclusion is not reversible.  For example, let $f:1\to 2$, and
%   let $B=2\backslash f(1)$.  Then $f^{-1}(B)=\emptyset$, and
%   $f(f^{-1}(B))=\emptyset$.

% \item If $f$ is surjective, then $f(f^{-1}(B))=B$.

%   Let $y\in B$.  Since $f$ is surjective, there is an $x\in X$ such
%   that $f(x)=y$.  Thus, $x\in f^{-1}(B)$ and $y=f(x)\in
%   f(f^{-1}(B))$.  


% \end{enumerate}







% \begin{exercises} \mbox{ }\newline \begin{enumerate}
%   \item Suppose that $A\subseteq X$.  Show that $(X\backslash A)\times
%     Y\cong (X\times Y)\backslash (A\times Y)$.  [To keep things
%     simple, argue in terms of elements and the relation $\in$.]
% \item Suppose that $A_1,A_2$ are nonempty subsets of $X$, and
%   $B_1,B_2$ are nonempty subsets of $Y$.  Show that if $A_1\times
%   B_1\subseteq A_2\times B_2$, then $A_1\subseteq A_2$ and
%   $B_1\subseteq B_2$. 
% \item Suppose that $R$ and $R'$ are equivalence relations.  Show that
%   $R\cap R'$ is an equivalence relation.  Describe the equivalence
%   classes of $R\cap R'$.  \end{enumerate} \end{exercises}



% \end{document}



% \begin{defn} We say that an object $X$ is \emph{injective} just in
%   case if $m:X\to Y$ is monic, and $f:X\to Z$ is a function, then
%   there is a function $\overline{f}:Y\to Z$ such that
%   $\overline{f}m=f$. 
% \[ \begin{tikzcd} 
% Y \arrow[dashed]{dr}{\overline{f}}  \\
% & Z  \\
% X \arrow[>->]{uu}{m} \arrow{ur}{f}
% \end{tikzcd} \]
% \end{defn}

% \begin{prop} Every non-empty set is injective. \end{prop}

% TO DO


% \begin{prop} Suppose that $X$ is finite, and that $m:B\to X$ is a
%   subobject of $X$.  Then $B$ is finite. \end{prop}

% TO DO

% Suppose that $f:B\to B$ is monic.  Let $g$ be the identity on
% $X\backslash B$.  Then $f+g$ is monic.  Since $X$ is finite, $f+g$ is
% iso.  Therefore, $f$ is iso.



% \begin{prop} If $X$ is infinite, then $X$ is non-empty. \end{prop}

% \begin{proof} Suppose that $X$ is empty.  Then every function $f:X\to
%   X$ is trivially surjective.  Therefore, $X$ is not
%   infinite. \end{proof}


% \begin{prop} $X\cup Y$ is infinite if and only if either $X$ or $Y$ is
%   infinite. \end{prop}






% \section{Power sets}

% TO DO: Cantor's diagonalization argument




% TO DO: Show that coproduct maps are jointly surjective.  (Note that
% first Lawvere paper takes this as an axiom.)

% \begin{prop} Let $q_0:X\to Z$ and $q_1:Y\to Z$.  If $\{ Z,q_0,q_1\}$
%   is a coproduct of $X$ and $Y$, then for each $z\in Z$, there is
%   either an $x\in X$ such that $q_0(x)=z$, or there is a $y\in Y$ such
%   that $q_1(y)=z$. \end{prop}

% \begin{proof} Suppose that $z\in Z$, but there is no $x\in X$ with
%   $q_0(x)=z$, and no $y\in Y$ with $q_1(y)=z$.  If $X$ and $Y$ were
%   both empty, then they both would be isomorphic to the initial object
%   Since $Z$ is non-empty, it's not possible that both $X$ and $Y$ are
%   non-empty.  Define $f:Z\to Z$ by
% \end{proof}


% TO DO: complement of a subobject, Lawvere p 114

% \begin{note} By Axiom \ref{ax:two}, there are exactly two functions
%   from $1$ to $\Omega$.  Thus, by Axiom \ref{ax:classifier}, there are
%   exactly two subobjects of $1$.  \end{note}

% %% TO DO: Show that $\Omega \cong \2P (1)$



% \begin{prop} The initial object $0$ is empty, i.e.\ has no
%   elements. \label{initial-empty} \end{prop}

% \begin{proof} Suppose for reductio ad absurdum that there is a
%   function $f:1\to 0$.  Let $g:0\to 1$ be the unique function from $0$
%   to $1$.  Then $gf=\mathrm{id}_1$, since $\mathrm{id}_1:1\to 1$ is
%   unique.  And $fg=\mathrm{id}_0$, since $\mathrm{id}_0:0\to 0$ is
%   unique.  Therefore, $0\cong 1$, in contradiction with Axiom
%   \ref{nontrivial}. \end{proof}

%%% Local Variables:
%%% mode: latex
%%% TeX-master: "main"
%%% End:
