% \documentclass[11pt,toc,fleqn]{article}

% \usepackage{array,multirow,amsthm,amsmath,amssymb,url}
% % \usepackage{bussproofs}
% % \usepackage{showlabels}
% \usepackage{natbib}
% \usepackage{mathrsfs}
% \usepackage{tikz-cd}
% \newtheorem{prop}{Proposition}[section]
% \newtheorem{lemma}[prop]{Lemma}
% \newtheorem{thm}[prop]{Theorem}
% \newtheorem{cor}[prop]{Corollary}
% \newtheorem{conj}[prop]{Conjecture}
% \newtheorem{fact}[prop]{Fact}
% % \newtheorem*{stone}{Stone Duality Theorem}
% \theoremstyle{definition}
% \newtheorem*{defn}{Definition}
% \newtheorem*{exercise}{Exercise}
% \newtheorem*{exercises}{Exercises}
% \newtheorem*{example}{Example}
% \newtheorem*{remark}{Remark}
% \theoremstyle{remark}
% \newtheorem*{disc}{Discussion}
% \newtheorem*{note}{Note}

% \author{Hans Halvorson}
% \date{\today}
% \newcommand{\RA}{\vdash}
% \newcommand{\7}{\mathbb}
% \newcommand{\2}{\mathscr}
% \newcommand{\lra}{\leftrightarrow}
% \newcommand{\cat}[1]{\mathbf{#1}}
% \renewcommand{\emph}{\textbf}
% \newcommand{\monic}{\rightarrowtail}
% \newcommand{\epi}{\twoheadrightarrow}
% \usepackage[framemethod=TikZ]{mdframed}
% \newcounter{axi}\setcounter{axi}{0}
% 


% \begin{document}



\chapter{The Category of Propositional Theories} \label{cat-prop}


%% further reading: Just and Weese, Discovering Modern Set Theory, vol 2

%% http://plato.stanford.edu/entries/boolalg-math/

%% high priority: in Stone, epi are surjective


%% TO DO: show existence of generator and co-generator in $\cat{Th}$.
%% The generator is the empty theory in signature { p}.

One of the primary goals of this book is to provide a formal model of
``the universe of all scientific theories.''  Now, in the 20th
century, mathematics stepped up another level of abstraction, and it
began to talk of structured collections of mathematical objects ---
e.g.\ the category of groups, topological spaces, manifolds, Hilbert
spaces, or sets.  This maneuver can be a little bit challenging for
foundationally oriented thinkers, viz.\ philosophers, because we are
now asked to consider collections that are bigger than any set.
However, mathematicians know very well how to proceed in this manner
without falling into contradictions (e.g.\ by availing themselves of
Grothendieck universes).

We want to follow the lead of the mathematicians, but instead of
talking about the category of groups, or manifolds, or Hilbert spaces,
etc., we want to talk about the category of all {\it theories}.  In
the present chapter we work out one special case: the category of all
propositional theories.  Of course, this category is too simple to
serve as a good model for the category of all scientific theories.
However, already for predicate logic, the category of theories becomes
extremely complex, almost to the point of mathematical intractability.
In subsequent chapters, we will make some headway with that case; for
the remainder of this chapter, we restrict ourselves to the
propositional case.

After defining the relevant category $\mathbf{Th}$ of propositional
theories, we will show that $\mathbf{Th}$ is equivalent to the
category $\cat{Bool}$ of Boolean algebras.  We then prove a version of
the famous Stone duality theorem, which shows that $\cat{Bool}$ is
dual to a certain category $\cat{Stone}$ of topological spaces.  This
duality shows that each propositional theory corresponds to a unique
topological space, viz.\ the space of its models; and each translation
between theories corresponds to a continuous mapping between their
spaces of models.


\section{Basics}

\newcommand{\sg}{\mathsf{Sent}(\Sigma )}

\begin{defn} We let $\mathbf{Th}$ denote the category whose objects
  are propositional theories, and whose arrows are translations
  between theories.  We say that two translations
  $f,g:T\rightrightarrows T'$ are equal, written $f=g$, just in case
  $T'\vdash f(\phi )\leftrightarrow g(\phi )$ for every $\phi\in\sg$.
  [Note well: equality between translations is weaker than
  set-theoretic equality.]  \end{defn}

\begin{defn} We say that a translation $f:T\to T'$ is
  \emph{conservative} just in case: for any
  $\phi\in\mathsf{Sent}(\Sigma )$, if $T'\vdash f(\phi )$ then
  $T\vdash \phi$. \end{defn}

\begin{prop} A translation $f:T\to T'$ is conservative if and only if
  $f$ is a monomorphism in the category
  $\cat{Th}$. \label{test} \end{prop}

\begin{proof} Suppose first that $f$ is conservative, and let
  $g,h:T''\to T$ be translations such that $f\circ g=f\circ h$.  That
  is, $T'\vdash fg(\phi )\leftrightarrow fh(\phi )$ for every sentence
  $\phi$ of $\Sigma ''$.  Since $f$ is conservative, $T\vdash g(\phi
  )\leftrightarrow h(\phi )$ for every sentence $\phi$ of $\Sigma ''$.
  Thus, $g=h$, and $f$ is a monomorphism in $\cat{Th}$.

  Conversely, suppose that $f$ is a monomorphism in the category
  $\cat{Th}$.  Let $\phi$ be a $\Sigma$ sentence such that $T'\vdash
  f(\phi )$.  Thus, $T'\vdash f(\phi )\leftrightarrow f(\psi )$, where
  $\psi$ is any $\Sigma$ sentence such that $T\vdash \psi$.  Now let
  $T''$ be the empty theory in signature $\Sigma ''=\{ p\}$.  Define
  $g:\Sigma ''\to \mathsf{Sent}(\Sigma )$ by $g(p)=\phi$, and define
  $h:\Sigma ''\to \mathsf{Sent}(\Sigma )$ by $h(p)=\psi$.  It's easy
  to see then that $f\circ g=f\circ h$.  Since $f$ is monic, $g=h$,
  which means that $T\vdash g(p)\leftrightarrow h(p)$.  Therefore,
  $T\vdash \phi$, and $f$ is conservative. \end{proof}

\begin{defn} We say that $f:T\to T'$ is \emph{essentially surjective}
  just in case for any sentence $\phi$ of $\Sigma '$, there is a
  sentence $\psi$ of $\Sigma$ such that $T'\vdash \phi\lra f(\psi
  )$. (Sometimes we use the abbreviation ``eso'' for essentially
  surjective.)  \end{defn}

\begin{prop} If $f:T\to T'$ is essentially surjective, then $f$ is an
  epimorphism in $\cat{Th}$.  \end{prop}

\begin{proof} Suppose that $f:T\to T'$ is eso.  Let
  $g,h:T'\rightrightarrows T''$ such that $g\circ f=h\circ f$.  Let
  $\phi$ be an arbitrary $\Sigma '$ sentence.  Since $f$ is eso, there
  is a sentence $\psi$ of $\Sigma$ such that $T'\vdash
  \phi\leftrightarrow f(\psi )$.  But then $T''\vdash g(\phi )\lra
  h(\phi )$.  Since $\phi$ was arbitrary, $g=h$.  Therefore, $f$ is an
  epimorphism.
\end{proof}

What about the converse of this proposition?  Are all epimorphisms in
$\cat{Th}$ essentially surjective?  The answer is Yes, but the result
is not easy to prove.  We'll prove it later on, by means of the
correspondence that we establish between theories, Boolean algebras,
and Stone spaces.


%% TO DO -- have we defined equivalence?

\begin{prop} Let $f:T\to T'$ be a translation.  If $f$ is conservative
  and essentially surjective, then $f$ is a homotopy
  equivalence. \end{prop}

\begin{proof} Let $p\in \Sigma '$.  Since $f$ is eso, there is some
  $\phi _p\in\mathsf{Sent}(\Sigma )$ such that $T'\vdash
  p\leftrightarrow f(\phi _p)$.  Define a reconstrual $g:\Sigma '\to
  \mathsf{Sent}(\Sigma )$ by setting $g(p)=\phi _p$.  As usual, $g$
  extends naturally to a function from $\mathsf{Sent}(\Sigma ')$ to
  $\mathsf{Sent}(\Sigma )$, and it immediately follows that $T'\vdash
  \psi \leftrightarrow fg(\psi )$, for every sentence $\psi$ of
  $\Sigma '$.

  We claim now that $g$ is a translation from $T'$ to $T$.  Suppose
  that $T'\vdash \psi$.  Since $T'\vdash \psi\lra fg(\psi )$, it
  follows that $T'\vdash fg(\psi )$.  Since $f$ is conservative,
  $T\vdash g(\psi )$.  Thus, for all sentences $\psi$ of $\Sigma '$,
  if $T'\vdash \psi$ then $T\vdash g(\psi )$, which means that
  $g:T'\to T$ is a translation.  By the previous paragraph,
  $1_{T'}\simeq fg$.

  It remains to show that $1_T\simeq gf$.  Let $\phi$ be an arbitrary
  sentence of $\Sigma$.  Since $f$ is conservative, it will suffice to
  show that $T'\vdash f(\phi )\lra fgf(\phi )$.  But by the previous
  paragraph, $T'\vdash \psi \lra fg(\psi )$ for all sentences $\psi$
  of $\Sigma'$.  Therefore, $1_{T}\simeq gf$, and $f$ is a homotopy
  equivalence.
\end{proof}


Before proceeding, let's remind ourselves of some of the motivations
for these technical investigations.

The category $\cat{Sets}$ is, without a doubt, extremely useful.
However, a person who is familiar with $\cat{Sets}$ might have
developed some intuitions that could be misleading when applied to
other categories.  For example, in $\cat{Sets}$, if there are
injections $f:X\to Y$ and $g:Y\to X$, then there is a bijection
between $X$ and $Y$.  Thus, it's tempting to think, for example, that
if there are embeddings $f:T\to T'$ and $g:T'\to T$ of theories, then
$T$ and $T'$ are equivalent.  [Here an embedding between theories is a
monomorphism in $\cat{Th}$, i.e.\ a conservative translation.]
Similarly, in $\cat{Sets}$, if there is an injection $f:X\to Y$ and a
surjection $g:X\to Y$, then there is a bijection between $X$ and $Y$.
However, in $\cat{Th}$ the analogous result fails to hold.

\begin{aside} For those familiar with the category $\cat{Vect}$ of
  vector spaces: $\cat{Vect}$ is similar to $\cat{Sets}$ in that
  mutually embeddable vector spaces are isomorphic.  That is, if
  $f:V\to W$ and $g:W\to V$ are monomorphisms (i.e.\ injective linear
  maps), then $V$ and $W$ have the same dimension, hence are
  isomorphic.

  The categories $\cat{Sets}$ and $\cat{Vect}$ share in common the
  feature that the objects can be classified by cardinal numbers.  In
  the case of sets, if $|X|=|Y|$, then $X\cong Y$.  In the case of
  vector spaces, if $\dim (V)=\dim (W)$, then $V\cong W$.
\end{aside}

\begin{prop} Let $f:T\to T'$ be a translation.  If $f^*:M(T')\to M(T)$
  is surjective, then $f$ is conservative. \end{prop}

\begin{proof} Suppose that $f^*$ is surjective, and suppose that
  $\phi$ is a sentence of $\Sigma$ such that $T\not\vdash \phi$.  Then
  there is a $v\in M(T)$ such that $v(\phi )=0$.\footnote{Here we have
    invoked the completeness theorem, but we haven't proven it yet.
    Note that our proof of the completeness theorem (page
    \pageref{page:complete}) does not cite this result, or any that
    depend on it.}  Since $f^*$ is surjective, there is a $w\in M(T')$
  such that $f^*(w)=v$.  But then
  \[ w(f(\phi )) = f^*w(\phi ) = v(\phi )=0 ,\] from which it follows
  that $T'\not\vdash f(\phi )$.  Therefore, $f$ is conservative.
\end{proof}


\begin{example} Let $\Sigma = \{ p_0,p_1,\dots \}$, and let $T$ be the
  empty theory in $\Sigma$.  Let $\Sigma '=\{ q_0,q_1,\dots \}$, and
  let $T'$ be the theory with axioms $q_0\to q_i$, for $i=0,1,\dots$.
  We already know that $T$ and $T'$ are not equivalent.  We will now
  show that there are embeddings $f:T\to T'$ and $g:T'\to T$.

  Define $f:\Sigma\to \mathsf{Sent}(\Sigma ')$ by $f(p_i)=q_{i+1}$.
  Since $T$ is the empty theory, $f$ is a translation.  Then for any
  valuation $v$ of $\Sigma '$, we have \[
  f^*v(p_i)=v(f(p_i))=v(q_{i+1}) .\] Furthermore, for any sequence of
  zeros and ones, there is a valuation $v$ of $\Sigma '$ that assigns
  that sequence to $q_1,q_2,\dots $.  Thus, $f^*$ is surjective, and
  $f$ is conservative.

  Now define $g:\Sigma '\to\mathsf{Sent}(\Sigma )$ by setting
  $g(q_i)=p_0\vee p_i$.  Since $T\vdash p_0\vee p_0\to p_0\vee p_i$,
  it follows that $g$ is a translation.  Furthermore, for any
  valuation $v$ of $\Sigma$, we have
  \[ g^*v(q_i) = v(g(q_i)) = v(p_0\vee p_i) .\] Recall that $M(T')$
  splits into two parts: (1) a singleton set containing the valuation
  $z$ where $z(q_i)=1$ for all $i$, and (2) the infinitely many other
  valuations which assign $0$ to $q_0$.  Clearly, $z=g^*v$, where $v$
  is any valuation such that $v(p_0)=1$.  Furthermore, for any
  valuation $w$ of $\Sigma '$ such that $w(p_0)=0$, we have $w=g^*v$,
  where $v(p_i)=w(q_i)$.  Therefore, $g^*$ is surjective, and $g$ is
  conservative.
\end{example}

\begin{exercise} In the example above: show that $f$ and $g$ are not
  essentially surjective.  \end{exercise}

\begin{example} Let $T$ and $T'$ be as in the previous example.  Now
  we'll show that there are essentially surjective (eso) translations
  $k:T\to T'$ and $h:T'\to T$.  The first is easy: the translation
  $k(p_i)=q_i$ is obviously eso.  For the second, define
  $h(q_0)=\bot$, where $\bot$ is some contradiction, and define
  $h(q_i)=p_{i-1}$ for $i>0$.  \end{example}

\begin{disc} Let's pause for just a moment and think about some of the
  questions we might want to ask about theories.  We arrange these in
  roughly decreasing order of technical tractability.
\begin{enumerate}
\item Does $\cat{Th}$ have the \emph{Cantor-Bernstein property}?  That
  is, if there are monomorphisms $f:T\to T'$ and $g:T'\to T$, then is
  there an isomorphism $h:T\to T'$?
\item Is $\cat{Th}$ balanced, in the sense that if $f:T\to T'$ is both
  a monomorphism and an epimorphism, then $f$ is an isomorphism?
\item If there is both a monomorphism $f:T\to T'$ and an epimorphism
  $g:T'\to T$, then are $T$ and $T'$ homotopy equivalent?
\item Can an arbitrary theory $T$ be embedded into a theory $T_0$ that
  has no axioms?  \cite{quine-goodman} present a proof of this claim
  --- and they argue that it undercuts the analytic-synthetic
  distinction.  They are right about the technical claim (see
  \ref{qgood} below), but have perhaps misconstrued its philosophical
  implications.
\item If theories have the same number of models, then are they
  equivalent?  If not, then can we determine whether $T$ and $T'$ are
  equivalent by inspecting $M(T)$ and $M(T')$?
\item How many theories (up to isomorphism) are there with $n$ models?
\item (Supervenience implies Reduction) Suppose that the truth value
  of a sentence $\psi$ \emph{supervenes} on the truth value of some
  other sentences $\phi _1,\dots ,\phi _n$, i.e., for any valuations
  $v,w$ of the propositional constants occurring in
  $\phi _1,\dots ,\phi _n,\psi$, if $v(\phi _i)=w(\phi _i)$, for
  $i=1,\dots ,n$, then $v(\psi )=w(\psi )$.  Does it follow then that
  $\vdash\psi \leftrightarrow \theta$, where $\theta$ contains only
  the propositional constants that occur in $\phi _1,\dots ,\phi _n$?
  (The answer is Yes, as shown by \emph{Beth's theorem}, Section
  \ref{go-beth}.)
\item Suppose that $f:T\to T'$ is conservative.  Suppose also that
  every model of $T$ extends uniquely to a model of $T'$.  Does it
  follow that $T\cong T'$?
\item Suppose that $T$ and $T'$ are consistent in the sense that there
  is no sentence $\theta$ in $\Sigma \cap \Sigma '$ such that $T\vdash
  \theta$ and $T'\vdash \neg \theta$.  Is there a unified theory $T''$
  which extends both $T$ and $T'$?  (The answer is Yes, as shown by
  \emph{Robinson's theorem}.)
\item What does it mean for one theory to be \emph{reducible} to
  another?  Can we explicate this notion in terms of a certain sort of
  translation between the relevant theories?  Some philosophers have
  claimed that the reduction relation ought to be treated
  semantically, rather than syntactically.  In other words, they would
  have us consider functions from $M(T')$ to $M(T)$, rather than
  translations from $T$ to $T'$.  In light of the Stone duality
  theorem proved below, it appears that syntactic and semantic
  approaches are equivalent to each other.
\item Consider various formally definable notions of theoretical
  equivalence.  What are the advantages and disadvantages of the
  various notions?  Is homotopy equivalence too liberal?  Is it too
  conservative?
\end{enumerate}
\end{disc}



%% To do: Proof of Th <-> Bool and Bool <-> Stone

\section{Boolean algebras} \label{sec:bool}

\begin{defn} A \emph{Boolean algebra} is a set $B$ together with a
  unary operation $\neg$, two binary operations $\wedge$ and $\vee$,
  and designated elements $0\in B$ and $1\in B$, which satisfy the
  following equations:
\begin{enumerate} 
\item Top and Bottom \\ $a\wedge 1=a\vee 0=a$
\item Idempotence \\ $a\wedge a=a\vee a=a$
\item De Morgan's rules \\ $\neg (a\wedge b)=\neg a\vee\neg b, \quad \neg
  (a\vee b)=\neg a\wedge\neg b$
\item Commutativity \\
$a\wedge b=b\wedge a, \quad a\vee b=b\vee a$
\item Associativity \\
$(a\wedge b)\wedge c=a\wedge (b\wedge c), \quad (a\vee b)\vee c=a\vee (b\vee
c)$
\item Distribution \\ $a\wedge (b\vee c)=(a\wedge b)\vee (a\wedge
  c),\quad a\vee (b\wedge c)=(a\vee b)\wedge (a\vee c)$
\item Excluded Middle \\ 
$a\wedge \neg a=0,\quad a\vee \neg a=1$
\end{enumerate}
Here we are implicitly universally quantifying over $a,b,c$.
\end{defn}

\begin{example} Let $2$ denote the unique Boolean algebra with two
  elements $\emptyset$ and $1$.  We can think of $2$ as the powerset
  of a one-element set $1$, where $\wedge$ is intersection, $\vee$ is
  union, and $\neg$ is complement.  

  Note that $2$ looks just like the truth-value set $\Omega$.  Indeed,
  $\Omega$ is equipped with operations $\wedge ,\vee$ and $\neg$ that
  make it into a Boolean algebra.  \end{example}

\begin{example} Let $F$ denote the unique Boolean algebra with four
  elements.  We can think of $F$ as the powerset of a two-element set,
  where $\wedge$ is intersection, $\vee$ is union, and $\neg$ is
  complement.  

  Let $\Sigma = \{ p\}$.  Define an equivalence relation $\simeq$ on
  sentences of $\Sigma$ by $\phi\simeq\psi$ just in case $\vdash
  \phi\lra\psi$.  The resulting set of equivalence classes naturally
  carries the structure of a Boolean algebra with four
  elements.  \end{example}



We now derive some basic consequences from the axioms.  The first two
results are called the \emph{absorption laws}.
\begin{enumerate}
\item $a\wedge (a\vee b) = a$ \\
  \[ a\wedge (a\vee b)=(a\vee 0)\wedge (a\vee b)=a\vee (0\wedge
  b)=a\vee 0=a .\]
\item $a\vee (a\wedge b)=a$ \\
\[ a\vee (a\wedge b)=(a\wedge 1)\vee (a\wedge b) = a\wedge (1\vee b)
=a\wedge 1=a . \]
\item $a\vee 1=1$ \\
\[ a\vee 1=a\vee (a\vee \neg a)=a\vee \neg a = 1 . \]
\item $a\wedge 0=0$ \\
\[ a\wedge 0=a\wedge (a\wedge \neg a)=a\wedge\neg a= 0. \] \end{enumerate}

\begin{defn} If $B$ is a Boolean algebra and $a,b\in B$, we write
  $a\leq b$ when $a\wedge b=a$. \end{defn}

Since $a\wedge 1=a$, it follows that $a\leq 1$, for all $a\in B$.
Since $a\wedge 0=0$, it follows that $0\leq a$, for all $a\in B$.  Now
we will show that $\leq$ is a partial order, i.e.\ reflexive,
transitive, and asymmetric.

\begin{prop} The relation $\leq$ on a Boolean algebra $B$ is a partial
  order.
\end{prop} 

\begin{proof} (Reflexive) Since $a\wedge a=a$, it follows that $a\leq
  a$.

(Transitive) Suppose that $a\wedge b=a$ and
$b\wedge c=b$.  Then
\[ \begin{array}{l l} a\wedge c = (a\wedge b)\wedge c = a\wedge
  (b\wedge c)=a\wedge b = a, \end{array} \] which means that $a\leq
c$.

(Asymmetric) Suppose that $a\wedge b=a$ and $b\wedge a=b$.  By
commutativity of $\wedge$, it follows that $a=b$.
\end{proof}

We now show how $\leq$ interacts with $\wedge,\vee$, and $\neg$.  In
particular, we show that if $\leq$ is thought of as implication, then
$\wedge$ behaves like conjunction, $\vee$ behaves like disjunction,
$\neg$ behaves like negation, $1$ behaves like a tautology, and $0$
behaves like a contradiction.

\begin{prop} $c\leq a\wedge b$ iff $c\leq a$ and $c\leq
  b$.  \end{prop}

\begin{proof} Since $a\wedge (a\wedge b)=a\wedge b$, it follows that
  $a\wedge b\leq a$.  By similar reasoning, $a\wedge b\leq b$.  Thus
  if $c\leq a\wedge b$, then transitivity of $\leq$ entails that both
  $c\leq a$ and $c\leq b$.

  Now suppose that $c\leq a$ and $c\leq b$.  That is, $c\wedge a=c$
  and $c\wedge b=c$.  Then $c\wedge (a\wedge b)=(c\wedge a)\wedge
  (c\wedge b)=c\wedge c=c$.  Therefore $c\leq a\wedge b$.  \end{proof}

Notice that $\leq$ and $\wedge$ interact precisely as implication and
conjunction interact in propositional logic.  The elimination rule
says that $a\wedge b$ implies $a$ and $b$.  Hence, if $c$ implies
$a\wedge b$, then $c$ implies $a$ and $b$.  The introduction rule says
that $a$ and $b$ imply $a\wedge b$.  Hence if $c$ implies $a$ and $b$,
then $c$ implies $a\wedge b$.

\begin{prop} $a\leq c$ and $b\leq c$ iff $a\vee b\leq c$ \end{prop}

\begin{proof} Suppose first that $a\leq c$ and $b\leq c$.  Then
  \[ \begin{array}{l l} (a\vee b)\wedge c = (a\wedge c)\vee (b\wedge
    c)=a\vee b .\end{array} \] Therefore $a\vee b\leq c$. 

  Suppose now that $a\vee b\leq c$.  By the absorption law, $a\wedge
  (a\vee b)=a$, which implies that $a\leq a\vee b$.  By transitivity
  $a\leq c$.  Similarly, $b\leq a\vee b$, and by transitivity, $b\leq
  c$.  \end{proof}

Now we show that the connectives $\wedge$ and $\vee$ are monotonic.

\begin{prop} If $a\leq b$ then $a\wedge c\leq b\wedge c$, for any
  $c\in B$.  \end{prop}

\begin{proof} 
\[ \begin{array}{l  l  l}
(a\wedge c)\wedge (b\wedge c) = (a\wedge b)\wedge c=a\wedge c
.\end{array} \] \end{proof}

\begin{prop} If $a\leq b$ then $a\vee c\leq b\vee c$, for any $c\in
  B$. \end{prop}

\begin{proof} 
\[ \begin{array}{l  l  l}
(a\vee c)\wedge (b\vee c) = (a\wedge b)\vee c = a\vee c .\end{array} \]
\end{proof}


\begin{prop} If $a\wedge b=a$ and $a\vee b=a$ then $a=b$. \end{prop}

\begin{proof} $a\wedge b=a$ means that $a\leq b$.  We now claim that
  $a\vee b=a$ iff $b\wedge a=b$ iff $b\leq a$.  Indeed, if $a\vee b=a$
  then
$$ b\wedge a = b\wedge (a\vee b)=(0\vee b)\wedge (a\vee b)=(0\wedge
a)\vee b = b .$$
Conversely, if $b\wedge a=b$, then 
$$ a\vee b = a\vee (a\wedge b) = (a\wedge 1)\vee (a\wedge b)= a\wedge
(1\vee b) = a .$$ Thus, if $a\wedge b=a$ and $a\vee b=a$, then $a\leq
b$ and $b\leq a$.  By asymmetry of $\leq$, it follows that
$a=b$. \end{proof}

We now show that $\neg a$ is the unique complement of $a$ in $B$.

\begin{prop} If $a\wedge b=0$ and $a\vee b=1$ then $b=\neg
  a$. \label{unique-complement} \end{prop}

\begin{proof} Since $b\vee a=1$, we have
\[ 
b = b\vee 0 = b\vee (a\wedge \neg a) = (b\vee a)\wedge (b\vee \neg a)
= b\vee \neg a .\] Since $b\wedge a=0$, we also have 
\[ b=b\wedge 1=b\wedge (a\vee \neg a) = (b\wedge a)\vee (b\wedge \neg
a)=b\wedge \neg a .\]
By the preceding proposition, $b=\neg a$. 
\end{proof}

\begin{prop} $\neg 1=0$. \end{prop}

\begin{proof} We have $1\wedge 0=0$ and $1\vee 0=1$.  By the preceding
  proposition, $0=\neg 1$. \end{proof}


\begin{prop} If $a\leq b$ then $\neg b\leq \neg a$. \end{prop}

\begin{proof} Suppose that $a\leq b$, which means that $a\wedge b=a$,
  and equivalently, $a\vee b=b$.  Thus, $\neg a\wedge \neg b =\neg
  (a\vee b)=\neg b$, which means that $\neg b\leq\neg a$. \end{proof}

\begin{prop} $\neg \neg a=a$.  \end{prop}

\begin{proof} We have $\neg a\vee \neg\neg a=1$ and $\neg a\wedge \neg
  \neg a=1$.  By Proposition \ref{unique-complement}, it follows that
  $\neg\neg a=a$. \end{proof}


\begin{defn} Let $A$ and $B$ be Boolean algebras.  A
  \emph{homomorphism} is a map $\phi :A\to B$ such that $\phi (0)=0$,
  $\phi (1)=1$, and for all $a,b\in A$, $\phi (\neg a)=\neg \phi (a)$,
  $\phi (a\wedge b)=\phi (a)\wedge \phi (b)$ and $\phi (a\vee b)=\phi
  (a)\vee \phi (b)$. \end{defn}

It is easy to see that if $\phi :A\to B$ and $\psi :B\to C$ are
homomorphisms, then $\psi\circ \phi :A\to C$ is also a homomorphism.
Moreover, $1_A:A\to A$ is a homomorphism, and composition of
homomorphisms is associative.  

\begin{defn} We let $\cat{Bool}$ denote the category whose objects are
  Boolean algebras, and whose arrows are homomorphisms of Boolean
  algebras. \end{defn}

Since $\cat{Bool}$ is a
category, we have notions of \emph{monomorphisms},
\emph{epimorphisms}, \emph{isomorphisms}, etc..  Once again, it is
easy to see that an injective homomorphism is a monomorphism, and a
surjective homomorphism is an epimorphism.

\begin{prop} Monomorphisms in $\cat{Bool}$ are injective.  \end{prop}

\begin{proof} Let $f:A\to B$ be a monomorphism, and let $a,b\in A$.
  Let $F$ denote the Boolean algebra with four elements, and let $p$
  denote one of the two elements in $F$ that is neither $0$ nor $1$.
  Define $\hat{a}:F\to A$ by $\hat{a}(p)=a$, and define $\hat{b}:F\to
  A$ by $\hat{b}(p)=b$.  It is easy to see that $\hat{a}$ and
  $\hat{b}$ are uniquely defined by these conditions, and that they
  are Boolean homomorphisms.  Suppose now that $f(a)=f(b)$.  Then
  $f\hat{a}=f\hat{b}$, and since $f$ is a monomorphism,
  $\hat{a}=\hat{b}$, and therefore $a=b$.  Therefore $f$ is injective.
\end{proof}

It is also true that epimorphisms in $\cat{Bool}$ are surjective.
However, proving that fact is no easy task.  We will return to it
later in the chapter.

\begin{prop} If $f:A\to B$ is a homomorphism of Boolean algebras, then
  $a\leq b$ only if $f(a)\leq f(b)$. \end{prop}

\begin{proof} $a\leq b$ means that $a\wedge b=a$.  Thus, 
\[ 
f(a)\wedge f(b) = f(a\wedge b) = f(a) ,\] which means that $f(a)\leq
f(b)$. \end{proof}

\begin{defn} A homomorphism $\phi :B\to 2$ is called a \emph{state} of
  $B$. \end{defn}


\section{Equivalent categories}

We now have two categories on the table: the category $\cat{Th}$ of
theories, and the category $\cat{Bool}$ of Boolean algebras.  Our next
goal is to show that these categories are \emph{structurally
  identical}.  But what do we mean by this?  What we mean is that they
are \emph{equivalent categories}.  In order to explain what that
means, we need a few more definitions.

%% TO DO: define subalgebra

\begin{defn} Suppose that $\cat{C}$ and $\cat{D}$ are categories.  We
  let $\cat{C}_0$ denote the objects of $\cat{C}$, and we let
  $\cat{C}_1$ denote the arrows of $\cat{C}$.  A (covariant)
  \emph{functor} $F:\cat{C}\to\cat{D}$ consists of a pair of maps:
  $F_0:\cat{C}_0\to\cat{D}_0$, and $F_1:\cat{C}_1\to\cat{D}_1$ with
  the following properties:
  \begin{enumerate} 
  \item $F_0$ and $F_1$ are compatible in the sense that if $f:X\to Y$
    in $\cat{C}$,then $F_1(f):F_0(X)\to F_0(Y)$ in $\cat{D}$.
  \item $F_1$ preserves identities and composition in the following
    sense: $F_1(1_X)=1_{F_0(X)}$, and $F_1(g\circ f)=F_1(g)\circ
    F_1(f)$.
\end{enumerate}
When no confusion can result, we simply use $F$ in place of $F_0$ and
$F_1$.
\end{defn}

\begin{note} There is also a notion of a \emph{contravariant functor},
  where $F_1$ reverses the direction of arrows: if $f:X\to Y$ in
  $\cat{C}$, then $F_1(f):F_0(Y)\to F_0(X)$ in $\cat{D}$.
  Contravariant functors will be especially useful for examining the
  relation between a theory and its set of models.  We've already seen
  that a translation $f:T\to T'$ induces a function
  $f^*:M(T')\to M(T)$.  In Section \ref{sec:stone} we will see that
  $f\mapsto f^*$ is part of a contravariant functor. \end{note}

\begin{example} For any category $\cat{C}$, there is a functor
  $1_{\cat{C}}$ that acts as the identity on both objects and arrows.
  That is, for any object $X$ of $\cat{C}$, $1_{\cat{C}}(X)=X$.  And
  for any arrow $f$ of $\cat{C}$, $1_{\cat{C}}(f)=f$. \end{example}

\begin{defn} Let $F:\cat{C}\to \cat{D}$ and $G:\cat{C}\to \cat{D}$ be
  functors.  A \emph{natural transformation} $\eta :F\Rightarrow G$
  consists of a family $\{ \eta _X:F(X)\to G(X) \mid X\in \cat{C}_0
  \}$ of arrows in $\cat{D}$, such that for any arrow $f:X\to Y$ in
  $\cat{C}$, the following diagram commutes:
  \[ \begin{tikzcd} F(X) \arrow[r,"F(f)"] \arrow[d,"\eta _X"] & F(Y)
    \arrow[d,"\eta _Y"]
    \\
    G(X) \arrow[r,"G(f)"] & G(Y) \end{tikzcd} \] \end{defn}

\begin{defn} A natural transformation $\eta :F\Rightarrow G$ is said
  to be a \emph{natural isomorphism} just in case each arrow $\eta
  _X:F(X)\to G(X)$ is an isomorphism.  In this case, we write $F\cong
  G$. \end{defn}

\begin{defn} Let $F:\cat{C}\to \cat{D}$ and $G:\cat{D}\to \cat{C}$ be
  functors.  We say that $F$ and $G$ are a \emph{categorical
    equivalence} just in case $GF\cong 1_{\cat{C}}$ and
  $FG\cong 1_{\cat{D}}$. \label{df:cateq} \end{defn}





\section{Propositional theories are Boolean algebras}

In this section, we show that there is a one-to-one correspondence
between theories (in propositional logic) and Boolean algebras.  We
first need some preliminaries.

\begin{defn} Let $\Sigma$ be a propositional signature (i.e.\ a set),
  let $B$ be a Boolean algebra, and let $f:\Sigma\to B$ be an
  arbitrary function.  [Here we use $\cap ,\cup$ and $-$ for the
  Boolean operations, in order to avoid confusion with the logical
  connectives $\wedge ,\vee$ and $\neg$.]  Then $f$ naturally extends
  to a map $f:\mathsf{Sent}(\Sigma )\to B$ as
  follows: \begin{enumerate}
  \item $f(\phi\wedge\psi )=f(\phi )\cap f(\psi )$;
  \item $f(\phi\vee \psi )=f(\phi )\cup f(\psi )$;
  \item $f(\neg \phi) = -f(\phi )$.
\end{enumerate}
Now let $T$ be a theory in $\Sigma$.  We say that $f$ is an
\emph{interpretation} of $T$ in $B$ just in case: for all sentences
$\phi$, if $T\vdash \phi$ then $f(\phi )=1$.  \end{defn}

\begin{defn} Let $f:T\to B$ be an interpretation.  We say that:
\begin{enumerate}
\item $f$ is \emph{conservative} just in case: for all sentences
  $\phi$, if $f(\phi )=1$ then $T\vdash\phi$.
\item $f$ \emph{surjective} just in case: for each $a\in B$, there is
  a $\phi\in\mathsf{Sent}(\Sigma )$ such that $f(\phi
  )=a$. \end{enumerate} \end{defn}

\begin{lemma} Let $f:T\to B$ be an interpretation.  Then the following
  are equivalent:
  \begin{enumerate}
\item $f$ is conservative.
\item For any $\phi ,\psi\in\sg$, if $f(\phi )=f(\psi )$ then
  $T\vdash \phi\lra\psi$. \end{enumerate} \end{lemma}

\begin{proof} Note first that $f(\phi )=f(\psi )$ if and only if
  $f(\phi\lra \psi)=1$.  Suppose then that $f$ is conservative.  If
  $f(\phi )=f(\psi )$ then $f(\phi\lra\psi )=1$, and hence
  $T\vdash\phi\lra\psi$.  Suppose now that (2) holds.  If $f(\phi
  )=1$, then $f(\phi )=f(\phi\vee\neg\phi )$, and hence $T\vdash
  (\phi\vee\neg\phi )\lra \phi$.  Therefore $T\vdash\phi$, and $f$ is
  conservative.
\end{proof}

\begin{lemma} If $f:T\to B$ is an interpretation, and $g:B\to A$ is a
  homomorphism, then $g\circ f$ is an interpretation.  \end{lemma}

\begin{proof} This is almost obvious. \end{proof}

\begin{lemma} If $f:T\to B$ is an interpretation, and $g:T'\to T$ is a
  translation, then $f\circ g:T'\to B$ is an
  interpretation.  \end{lemma}

\begin{proof} This is almost obvious.  \end{proof}

\begin{lemma} Suppose that $T$ is a theory, and $e:T\to B$ is a
  surjective interpretation.  If $f,g:B\rightrightarrows A$ are
  homomorphisms such that $fe=ge$, then $f=g$.  \end{lemma}

\begin{proof} Suppose that $fe=ge$, and let $a\in B$.  Since $e$ is
  surjective, there is a $\phi\in\mathsf{Sent}(\Sigma )$ such that
  $e(\phi )=a$.  Thus, $f(a)=fe(\phi )=ge(\phi )=g(a)$.  Since $a$ was
  arbitrary, $f=g$. \end{proof}

Let $T'$ and $T$ be theories, and let $f,g:T'\rightrightarrows T$ be
translations.  Recall that we defined identity between translations as
follows: $f=g$ if and only if $T\vdash f(\phi )\lra g(\phi )$ for all
$\phi \in \mathsf{Sent}(\Sigma ')$.

\begin{lemma} Suppose that $m:T\to B$ is a conservative
  interpretation.  If $f,g:T'\rightrightarrows T$ are translations
  such that $mf=mg$, then $f=g$.  \end{lemma}

\begin{proof} Let $\phi\in \mathsf{Sent}(\Sigma ')$, where $\Sigma '$
  is the signature of $T'$.  Then $mf(\phi )=mg(\phi )$.  Since $m$ is
  conservative, $T\vdash f(\phi )\lra g(\phi )$.  Since this holds for
  all sentences, it follows that $f=g$.
\end{proof}



\begin{prop} For each theory $T$, there is a Boolean algebra $L(T)$,
  and a conservative, surjective interpretation $i_T:T\to L(T)$ such
  that for any Boolean algebra $B$, and interpretation $f:T\to B$,
  there is a unique homomorphism $\overline{f}:L(T)\to B$ such that
  $\overline{f}i_T=f$.  \label{lindenbaum} \end{prop}

\[ \begin{tikzcd} T \arrow[r,"i_T"] \arrow[dr,"f"'] & L(T)
  \arrow[dashed,d,"\overline{f}"] \\
& B \end{tikzcd} \]

  We define an equivalence relation $\equiv$ on the
sentences of $\Sigma$:
\[ \phi \equiv \psi \quad \text{iff} \quad T\vDash \phi\lra\psi ,\]
and we let \[ E_{\phi} \: := \: \{ \psi \mid \phi\equiv \psi \} .\]
Finally, let
\[ L(T) \: := \: \{ E_\phi \mid \phi \in \mathsf{Sent}(\Sigma ) \} .\]
We now equip $L(T)$ with the structure of a Boolean algebra.  To this
end, we need the following facts, which correspond to easy proofs in
propositional logic.

\begin{fact} If $E_{\phi}=E_{\phi '}$ and $E_{\psi}=E_{\psi '}$, then:
\begin{enumerate}
\item $E_{\phi\wedge\psi}=E_{\phi '\wedge \psi '}$;
\item $E_{\phi\vee\psi}=E_{\phi '\vee \psi '}$;
\item $E_{\neg \phi}=E_{\neg \phi '}$. \end{enumerate} \end{fact}

\noindent We then define a unary operation $-$ on $L(T)$ by:
\[ -E_{\phi } := E_{\neg \phi } , \] and we define two binary
operations on $L(T)$ by: \[ E_\phi\cap E_\psi := E_{\phi\wedge \psi }
,\qquad E_\phi\cup E_\psi := E_{\phi\vee \psi } .\] Finally, let
$\phi$ be an arbitrary $\Sigma$ sentence, and let
$0=E_{\phi\wedge\neg\phi}$ and $1=E_{\phi\vee\neg\phi}$.  The proof
that $\langle L(T),\cap ,\cup ,-,0,1\rangle$ is a Boolean algebra
requires a series of straightforward verifications.  For example,
let's show that $1\cap E_\psi =E_\psi$, for all sentences $\psi$.
Recall that $1=E_{\phi\vee\neg \phi}$ for some arbitrarily chosen
sentence $\phi$.  Thus,
\[ 1\cap E_\psi \: = \: E_{\phi\vee\neg\phi}\cap E_\psi \: = \:
E_{(\phi\vee\neg\phi)\wedge \psi} .\] Moreover, $T\vdash \psi\lra
((\phi\vee\neg\phi )\wedge \psi )$, from which it follows that
$E_{(\phi\vee\neg \phi )\wedge \psi }=E_\psi$. Therefore, $1\cap
E_\psi=E_\psi$.

Consider now the function $i_T:\Sigma\to L(T)$ given by $i_T(\phi
)=E_\phi$, and its natural extension to $\sg$.  A quick inductive
argument, using the definition of the Boolean operations on $L(T)$,
shows that $i_T (\phi )=E_\phi$ for all $\phi\in \mathsf{Sent}(\Sigma
)$.  The following shows that $i_T$ is a conservative interpretation
of $T$ in $L(T)$.

\begin{prop} $T\vdash\phi$ if and only if $i_T (\phi )=1$.  \end{prop}


\begin{proof} $T\vdash \phi$ iff $T\vdash (\psi\vee\neg\psi )\lra
  \phi$ iff $i_T (\phi )=E_\phi=E_{\psi\vee\neg \psi}=1$. \end{proof}

Since $i_T(\phi )=E_\phi$, the interpretation $i_T$ is also
surjective.

\begin{prop} Let $B$ be a Boolean algebra, and let $f:T\to B$ be an
  interpretation.  Then there is a unique homomorphism
  $\overline{f}:L(T)\to B$ such that $\overline{f}i_T=f$.  \end{prop}

\begin{proof} If $E_\phi=E_\psi$, then $T\vdash \phi\lra\psi$, and so
  $f(\phi )=f(\psi )$.  Thus, we may define $\hat{f}(E_\phi )=f(\phi
  )$.  It is straightforward to verify that $\hat{f}$ is a Boolean
  homomorphism, and it is clearly unique.
\end{proof}


\begin{defn} The Boolean algebra $L(T)$ is called the \emph{Lindenbaum
    algebra} of $T$. \end{defn}




\begin{prop} Let $B$ be a Boolean algebra.  There is a theory $T_B$
  and a conservative, surjective interpretation $e_B:T_B\to B$ such
  that for any theory $T$, and interpretation $f:T\to B$, there is a
  unique interpretation $\overline{f}:T\to T_B$ such that
  $e_B\overline{f}=f$.  \label{internal} \end{prop}

\[ \begin{tikzcd}
T_B \arrow{r}{e_B} & B  \\
T \arrow{ur}[below]{f} \arrow[dashed]{u}{\overline{f}} \end{tikzcd} \]

\begin{proof} Let $\Sigma _B=B$ be a signature.  (Recall that a
  propositional signature is just a set, where each element represents
  an elementary proposition.)  We define $e_B:\Sigma _B\to B$ as the
  identity, and use the symbol $e_B$ also for its extension to
  $\mathsf{Sent}(\Sigma _B)$.  We define a theory $T_B$ on $\Sigma _B$
  by: $T_B\vdash\phi$ if and only if $e_B(\phi )=1$.  Thus,
  $e_B:T_B\to B$ is automatically a conservative interpretation of
  $T_B$ in $B$.

  Now let $T$ be some theory in signature $\Sigma$, and let $f:T\to B$
  be an interpretation.  Since $\Sigma _B=B$, $f$ automatically gives
  rise to a reconstrual $f:\Sigma\to\Sigma _B$, which we will rename
  $\overline{f}$ for clarity.  And since $e_B$ is just the identity on
  $B=\Sigma _B$, we have $f=e_B\overline{f}$.

  Finally, to see that $\overline{f}:T\to T_B$ is a translation,
  suppose that $T\vdash \phi$.  Since $f$ is an interpretation of
  $T_B$, $f(\phi )=1$, which means that $e_B(\overline{f}(\phi ))=1$.
  Since $e_B$ is conservative, $T_B\vdash \overline{f}(\phi )$.
  Therefore, $\overline{f}$ is a translation.
\end{proof}



We have shown that each propositional theory $T$ corresponds to a
Boolean algebra $L(T)$, and each Boolean algebra $B$ corresponds to a
propositional theory $T_B$.  We will now show that these
correspondences are functorial.  First we show that a morphism $f:B\to
A$ in $\cat{Bool}$ naturally gives rise to a morphism $T(f):T_B\to
T_A$ in $\cat{Th}$.  Indeed, consider the following diagram:
\[ \begin{tikzcd}
  T_B \arrow{d}{e_B} \arrow[dashed]{r}{T(f)} & T_A \arrow{d}{e_A} \\
  B \arrow{r}{f} & A \end{tikzcd} \] Since $fe_B$ is an interpretation
of $T_B$ in $A$, Prop.\ \ref{internal} entails that there is a unique
translation $T(f):T_B\to T_A$ such that $e_AT(f)=fe_B$.  The
uniqueness clause also entails that $T$ commutes with composition of
morphisms, and maps identity morphisms to identity morphisms.  Thus,
$T:\cat{Bool}\to\cat{Th}$ is a functor.

Let's consider this translation $T(f):T_B\to T_A$ more concretely.
First of all, recall that translations from $T_B$ to $T_A$ are
actually equivalence classes of maps from $\Sigma _B$ to
$\mathsf{Sent}(\Sigma _A)$.  Thus, there's no sense to the question,
``which function is $T(f)$?''  However, there's a natural choice of a
representative function.  Indeed, consider $f$ itself as a function
from $\Sigma _B=B$ to $\Sigma _A=A$.  Then, for $x\in\Sigma _B=B$, we
have
\[ (e_A\circ T(f))(x) = e_A(f(x)) = f(x) = f(e_B(x)) ,\] since $e_A$
is the identity on $\Sigma _A$, and $e_B$ is the identity on $\Sigma
_B$.  In other words, $T(f)$ is the equivalence class of $f$ itself.
[But recall that translations, while initially defined on the
signature $\Sigma _B$, extend naturally to all elements of
$\mathsf{Sent}(\Sigma _B)$.  From this point of view, $T(f)$ has a
larger domain than $f$.]

A similar construction can be used to define the functor
$L:\cat{Th}\to \cat{Bool}$.  In particular, let $f:T\to T'$ be a
morphism in $\cat{Th}$, and consider the following diagram:
\[ \begin{tikzcd}
  T \arrow{d}{i_T} \arrow{r}{f} & T' \arrow{d}{i_{T'}} \\
  L(T) \arrow[dashed]{r}{L(f)} & L(T') \end{tikzcd} \] Since $i_{T'}f$
is an interpretation of $T$ in $L(T')$, Prop.\ \ref{lindenbaum}
entails that there is a unique homomorphism $L(f):L(T)\to L(T')$ such
that $L(f)i_T = i_{T'}f$.  

More explicitly,
\[ L(f)(E_\phi ) \: = \: L(f)(i_T(\phi)) \: = \: i_{T'}f(\phi ) \: =
\: E_{f(\phi )} .\] Recall, however, that identity of arrows in
$\cat{Th}$ is \textit{not} identity of the corresponding functions, in
the set-theoretic sense.  Rather, $f\simeq g$ just in case $T'\vdash
f(\phi )\lra g(\phi )$, for all $\phi\in\mathsf{Sent}(\Sigma )$.
Thus, we must verify that if $f\simeq g$ in $\cat{Th}$, then
$L(f)=L(g)$.  Indeed, since $i_{T'}$ is an interpretation of $T'$, we
have $i_{T'}(f(\phi ))=i_{T'}(g(\phi ))$; and since the diagram above
commutes, $L(f)\circ i_T=L(g)\circ i_T$.  Since $i_T$ is surjective,
$L(f)=L(g)$.  Thus, $f\simeq g$ only if $L(f)=L(g)$.  Finally, the
uniqueness clause in Prop.\ \ref{lindenbaum} entails that $L$ commutes
with composition, and maps identities to identities.  Therefore,
$L:\cat{Th}\to\cat{Bool}$ is a functor.

We will soon show that the functor $L:\cat{Th}\to\cat{Bool}$ is an
equivalence of categories, from which it follows that $L$ preserves
all categorically-definable properties.  For example, a translation
$f:T\to T'$ is monic if and only if $L(f):L(T)\to L(T')$ is monic,
etc..  However, it may be illuminating to prove some such facts
directly.

\begin{prop} Let $f:T\to T'$ be a translation. Then $f$ is
  conservative if and only if $L(f)$ is injective. \end{prop}

\begin{proof} Suppose first that $f$ is conservative.  Let $E_\phi
  ,E_\psi\in L(T)$ such that $L(f)(E_\phi )=L(f)(E_\psi )$.  Using the
  definition of $L(f)$, we have $E_{f(\phi )}=E_{f(\psi )}$, which
  means that $T'\vdash f(\phi )\lra f(\psi )$.  Since $f$ is
  conservative, $T\vdash \phi\lra \psi$, from which $E_\phi=E_\psi$.
  Therefore, $L(f)$ is injective.

  Suppose now that $L(f)$ is injective.  Let $\phi$ be a $\Sigma$
  sentence such that $T'\vdash f(\phi )$.  Since $f(\top )=\top$, we
  have $T'\vdash f(\top )\lra f(\phi )$, which means that
  $L(f)(E_{\top})=L(f)(E_\phi )$.  Since $L(f)$ is injective,
  $E_\top=E_\phi$, from which $T\vdash \phi$.  Therefore, $f$ is
  conservative. \end{proof}


\begin{prop} For any Boolean algebra $B$, there is a natural
  isomorphism $\eta _B:B\to L(T_B)$.  \end{prop}

\begin{proof} Let $e_B:T_B\to B$ be the interpretation from Prop.\
  \ref{internal}, and let $i_{T_B}:T_B\to L(T_B)$ be the
  interpretation from Prop.\ \ref{lindenbaum}.  Consider the following
  diagram:
  \[ \begin{tikzcd} T_B \arrow[r,"i_{T_B}"] \arrow[dr,"e_B"'] &
    L(T_B) \arrow[dashed,d,"\eta _B"] \\
    & B \end{tikzcd} \] By Prop.\ \ref{lindenbaum}, there is a unique
  homomorphism $\eta _B:L(T_B)\to B$ such that $e_B=\eta _Bi_{T_B}$.
  Since $e_B$ is the identity on $\Sigma _B$,
  \[ \eta _B(E_x)=\eta _Bi_{T_B}(x)=e_B(x)=x , \] for any $x\in
  B$. Thus, if $\eta _B$ has an inverse, it must be given by the map
  $x\mapsto E_x$.  We claim that this map is a Boolean homomorphism.
  To see this, recall that $\Sigma _B=B$.  Moreover, for $x,y\in B$,
  the Boolean meet $x\cap y$ is again an element of $B$, hence an
  element of the signature $\Sigma _B$.  By the defintion of $T_B$, we
  have $T_B\vdash (x\cap y)\leftrightarrow (x\wedge y)$, where the
  $\wedge$ symbol on the right is conjunction in $\mathsf{Sent}(\Sigma
  _B)$.  Thus,
  \[ E_{x\cap y}=E_{x\wedge y}=E_x\cap E_y .\] A similar argument
  shows that $E_{-x}=-E_x$.  Therefore, $x\mapsto E_x$ is a Boolean
  homomorphism, and $\eta _B$ is an isomorphism.

  It remains to show that $\eta _B$ is natural in $B$.  Consider the
  following diagram:
\[ \begin{tikzcd} T_B \arrow{rr}{T_f} \arrow{dr}{e_B}
  \arrow{ddr}[below]{i_{T_B}} & & T_A
  \arrow{dr}{e_A} \arrow{ddr}[below]{i_{T_A}} \\
  & B  \arrow{rr}{f} & & A  \\
  & L(T_B) \arrow[u,"\eta _B"'] \arrow{rr}{LT(f)} & & L(T_A) \arrow[u,"\eta _A"'] \end{tikzcd} \] The top square
commutes by the definition of the functor $T$.  The triangles on the
left and right commute by the definition of $\eta$.  And the outmost
square commutes by the definition of the functor $L$.  Thus we have
\[ \begin{array}{l l l} 
f\circ \eta _B\circ i_{T_B} & = & f\circ e_B \\
& = & e_A\circ T_f \\
& = & \eta _A\circ i_{T_A}\circ T_f \\
& = & \eta _A\circ LT(f)\circ i_{T_B} .
\end{array} \] Since $i_{T_B}$ is
surjective, it follows that $f\circ \eta _B=\eta _A\circ LT(f)$, and
therefore $\eta$ is a natural transformation.
\end{proof}

\begin{disc} Consider the algebra $L(T_B)$, which we have just proved
  is isomorphic to $B$.  This result is hardly surprising.  For any
  $x,y\in \Sigma _B$, we have $T_B\vdash x\lra y$ if and only if
  $x=e_B(x)=e_B(y)=y$.  Thus, the equivalence class $E_x$ contains $x$
  and no other element from $\Sigma _B$.  [That's why $\eta _B(E_x)=x$
  makes sense.]  We also know that for every
  $\phi\in\mathsf{Sent}(\Sigma _B)$, there is an $x\in \Sigma _B=B$
  such that $T_B\vdash x\lra \phi$.  In particular, $T_B\vdash
  e_B(\phi)\lra \phi$.  Thus, $E_\phi=E_x$, and there is a natural
  bijection between elements of $L(T_B)$ and elements of $B$.
\end{disc}


\begin{prop} For any theory $T$, there is a natural isomorphism
  $\varepsilon _T:T\to T_{L(T)}$. \end{prop}

\begin{proof} Consider the following diagram:
\[ \begin{tikzcd}
T_{L(T)} \arrow{r}{e_{L(T)}} & L(T) \\
T \arrow[dashed]{u}{\varepsilon _T} \arrow{ur}[below]{i_T}  
\end{tikzcd} \] By Prop.\ \ref{internal}, there is a unique
interpretation $\varepsilon _T:T\to T_{L(T)}$ such that
$e_{L(T)}\varepsilon _T=i_T$.  We claim that $\varepsilon _T$ is an
isomorphism.  To see that $\ve _T$ is conservative, suppose that
$T_{L(T)}\vdash \ve _T(\phi )$.  Since $e_{L(T)}$ is an
interpretation, $e_{L(T)}\ve _T (\phi )=1$ and hence $i_T(\phi )=1$.
Since $i_T$ is conservative, $T\vdash \phi$.  Therefore $\ve _T$ is
conservative.  

To see that $\ve _T$ is essentially surjective, suppose that $\psi\in
\mathsf{Sent}(\Sigma _{L(T)})$.  Since $i_T$ is surjective, there is a
$\phi\in \mathsf{Sent}(\Sigma )$ such that $i_T(\phi )=e_{L(T)}(\psi
)$.  Thus, $e_{L(T)}(\ve _T(\phi ))=e_{L(T)}(\psi )$.  Since
$e_{L(T)}$ is conservative, $T_{L(T)}\vdash \ve _T(\phi)\lra \psi$.
Therefore, $\ve _T$ is essentially surjective.

It remains to show that $\ve _T$ is natural in $T$.  Consider the
following diagram:
\[ \begin{tikzcd}
T \arrow{d}[left]{\ve _T} \arrow{rr}{f} \arrow{ddr}{i_T} & &  T'
\arrow{d}{\ve _{T'}} \arrow{ddr}{i_{T'}} \\
T_{L(T)} \arrow{rr}{TL(f)} \arrow{dr}[left]{e_{L(T)}} & & T_{L(T')}
  \arrow{dr}[left]{e_{L(T')}} \\
& L(T) \arrow{rr}{L(f)} & & L(T') \end{tikzcd} \] 
The triangles on the left and the right commute by the definition of
$\ve$.  The top square commutes by the definition of $L$, and the
bottom square commutes by the definition of $T$.  Thus, we have
\[ \begin{array}{l l l} e_{L(T')}\circ \ve _{T'}\circ f & = & i_{T'}\circ f \\
& = & L(f)\circ i_T \\
& = & L(f)\circ e_{L(T)}\circ \ve _T \\
& = & e_{L(T')}\circ TL(f)\circ \ve _T .\end{array} \]
Since $e_{L(T')}$ is conservative, $\ve _{T'}\circ f=TL(f)\circ \ve
_T$.  Therefore $\ve _T$ is natural in $T$.  
\end{proof}

\begin{disc} Recall that $\ve _T$ doesn't denote a unique function; it
  denotes an equivalence class of functions.  One representative of
  this equivalence class is the function $\ve _T:\Sigma\to \Sigma
  _{L(T)}$ given by $\ve _T(p)=E_p$.  In this case, a straightforward
  inductive argument shows that $T_{L(T)}\vdash E_\phi \lra \ve
  _T(\phi )$, for all $\phi\in\mathsf{Sent}(\Sigma )$.

  We know that $\ve _T$ has an inverse, which itself is an equivalence
  class of functions from $\Sigma _{L(T)}$ to $\mathsf{Sent}(\Sigma
  )$.  We can define a representative $f$ of this equivalence class by
  choosing, for each $E\in \Sigma _{L(T)}=L(T)$, some $\phi\in E$, and
  setting $f(E)=\phi$.  Another straightforward argument shows that if
  we made a different set of choices, the resulting function $f'$
  would be equivalent to $f$, i.e.\ it would correspond to the same
  translation from $T_{L(T)}$ to $T$.

  Based on these definitions, $f\ve _T(p)=f(E_p)$ is some $\phi \in
  E_p$, i.e.\ some $\phi$ such that $T\vdash p\lra \phi$.  Thus, $f\ve
  _T\cong 1_T$.  Similarly, $f(E_\phi )=\psi$, for some $\psi \in
  E_\phi$, and hence $\ve _T$ ...    
\end{disc}

\bigskip Since there are natural isomorphisms $\ve
:1_{\cat{Th}}\Rightarrow TL$ and $\eta :1_{\cat{Bool}}\Rightarrow LT$,
we have the following result:

\begin{box-thm}[Lindenbaum Theorem] The categories $\cat{Th}$ and
  $\cat{Bool}$ are equivalent.  \end{box-thm}


\section{Boolean algebras again}

The Lindenbaum Theorem would deliver everything we wanted --- if we
had a perfectly clear understanding of the category $\cat{Bool}$.
However, there remain questions about $\cat{Bool}$.  For example, are
all epimorphisms in $\cat{Bool}$ surjections?  In order to shed even
more light on $\cat{Bool}$, and hence on $\cat{Th}$, we will show that
$\cat{Bool}$ is dual to a certain category of topological spaces.
This famous result is called the \emph{Stone Duality Theorem}.  But in
order to prove it, we need to collect a few more facts about Boolean
algebras.

\begin{defn} Let $B$ be a Boolean algebra.  A subset $F\subseteq B$ is
  said to be a \emph{filter} just in case:
\begin{enumerate}
\item If $a,b\in F$ then $a\wedge b\in F$;
\item If $a\in F$ and $a\leq b$ then $b\in F$.
\end{enumerate}
If, in addition, $F\neq B$, then we say that $F$ is a \emph{proper
  filter}.  We say that $F$ is an \emph{ultrafilter} just in case $F$
is maximal among proper filters, i.e.\ if $F\subseteq F'$ where $F'$
is a proper filter, then $F=F'$.
\end{defn}

\begin{disc} Consider the Boolean algebra $B$ as a theory.  Then a
  filter $F\subseteq B$ can be thought of as supplying an update of
  information.  The first condition says that if we learn $a$ and $b$,
  then we've learned $a\wedge b$.  The second condition says that if
  we learn $a$, and $a\leq b$, then we've learned $b$.  In particular,
  an ultrafilter supplies maximal information.  \end{disc}

\begin{exercise} Let $F$ be a filter.  Show that $F$ is proper if and
  only if $0\not\in F$. \end{exercise}



%% properties of ultrafilters / maximal ideals, see Schechter p 337

% Claim: if $y\wedge a=0$ then $y\leq \neg a$.  that is, $y\wedge \neg
% a= y$.  Indeed,
% $$ y=y\wedge 1 = y\wedge (a\vee \neg a) = (y\wedge a)\vee (y\wedge
% \neg a) = y\wedge \neg a .$$


\begin{defn} Let $F\subseteq B$ be a filter, and $a\in B$.  We say
  that $a$ is \textbf{compatible} with $F$ just in case $a\wedge x\neq
  0$ for all $x\in F$. \end{defn}

\begin{lemma} Let $F\subseteq B$ be a proper filter, and let $a\in B$.
  Then either $a$ or $\neg a$ is compatible with
  $F$. \label{compatible} \end{lemma}

\begin{proof} Suppose for reductio ad absurdum that neither $a$ nor
  $\neg a$ is compatible with $F$.  That is, there is an $x\in F$ such
  that $x\wedge a=0$, and there is a $y\in F$ such that $y\wedge \neg
  a=0$.  Then
$$ x\wedge y = (x\wedge y)\wedge (a\vee \neg a) =  (x\wedge y\wedge
a)\vee (x\wedge y\wedge \neg a) = 0 .$$ Since $x,y\in F$, it follows
that $0=x\wedge y\in F$, contradicting the assumption that $F$ is
proper.  Therefore either $a$ or $\neg a$ is compatible with $F$.
\end{proof}

\begin{prop} Let $F$ be a proper filter on $B$.  Then the following
  are equivalent:
\begin{enumerate}
\item $F$ is an ultrafilter.
\item For all $a\in B$, either $a\in F$ or $\neg a\in F$.
\item For all $a,b\in B$, if $a\vee b\in F$ then either $a\in F$ or
  $b\in F$. \end{enumerate} \end{prop}

\begin{proof} ($1\Rightarrow 2$) Suppose that $F$ is an ultrafilter.
  By Lemma \ref{compatible}, either $a$ or $\neg a$ is compatible with
  $F$.  Suppose first that $a$ is compatible with $F$.  Then the set  
$$ F'=\{ y : x\wedge a\leq y ,\;\text{some}\; x\in F \} ,$$
is a proper filter that contains $F$ and $a$.  Since $F$ is an
ultrafilter, $F'=F$, and hence $a\in F$.  By symmetry, if $\neg a$ is
compatible with $F$, then $\neg a\in F$.

\bigskip ($2\Rightarrow 3$) Suppose that $a\vee b\in F$.  By 2, either
$a\in F$ or $\neg a\in F$.  If $\neg a\in F$, then $\neg a\wedge
(a\vee b)\in F$.  But $\neg a\wedge (a\vee b) \leq b$, and so $b\in
F$.

\bigskip ($3\Rightarrow 1$) Suppose that $F'$ is a filter that
contains $F$, and let $a\in F'-F$.  Since $a\vee \neg a=1\in F$, it
follows from (3) that $\neg a\in F$.  But then $0=a\wedge \neg a\in
F'$, that is $F'=B$.  Therefore $F$ is an ultrafilter.
\end{proof}





\begin{prop} There is a bijective correspondence between ultrafilters
  in $B$ and homomorphisms from $B$ into $2$.  In particular, for any
  homomorphism $f:B\to 2$, the subset $f^{-1}(1)$ is an ultrafilter in
  $B$.  \end{prop}

\begin{proof} Let $U$ be an ultrafilter on $B$.  Define $f:B\to 2$ by
  setting $f(a)=1$ iff $a\in U$.  Then
  \[ \begin{array}{l l l}
    f(a\wedge b)=1 & \text{iff} & a\wedge b\in U \\
    & \text{iff} & a\in U \,\text{and}\,b\in U \\
    & \text{iff} & f(a)=1\; \text{and}\; f(b)=1 .\end{array}
  \] Furthermore,
  \[ \begin{array}{l l l}
    f(\neg a)=1 & \text{iff} & \neg a\in U \\
    & \text{iff} & a\not\in U  \\
    & \text{iff} & f(a)=0 .\end{array} \] Therefore $f$ is a
  homomorphism.

  Now suppose that $f:B\to 2$ is a homomorphism, and let
  $U=f^{-1}(1)$.  Since $f(a)=1$ and $f(b)=1$ only if $f(a\wedge
  b)=1$, it follows that $U$ is closed under conjunction.  Since
  $a\leq b$ only if $f(a)\leq f(b)$, it follows that $U$ is closed
  under implication.  Finally, since $f(a)=0$ iff $f(\neg a)=1$, it
  follows that $a\not\in U$ iff $\neg a\in U$. \end{proof}

\begin{defn} For $a,b\in B$, define
\[ a\to b \: := \: \neg a\vee b ,\]
and define 
\[ a\lra b \: := \: (a\to b)\wedge (b\to a) .\] \end{defn}

It's straightforward to check that $\to$ behaves like the conditional
from propositional logic.  The next lemma gives a Boolean algebra
version of modus ponens.

\begin{lemma} Let $F$ be a filter.  If $a\to b\in F$ and $a\in F$ then
  $b\in F$.  \end{lemma}

\begin{proof} Suppose that $\neg a\vee b=a\to b\in F$ and $a\in F$.
  We then compute:
$$ b=b\vee 0=b\vee (a\wedge \neg a) = (a\vee b)\wedge (\neg a \vee b)
.$$ Since $a\in F$ and $a\leq a\vee b$, we have $a\vee b\in F$.  Since
$F$ is a filter, $b\in F$. \end{proof}


\begin{exercise} \mbox{}
  
\begin{enumerate}
\item Let $B$ be a Boolean algebra, and let $a,b,c\in B$.  Show that
  the following hold:
\begin{enumerate}
\item $(a\to b)=1$ iff $a\leq b$
\item $(a\wedge b)\leq c$ iff $a\leq (b\to c)$
\item $a\wedge (a\to b)\leq b$
\item $(a\lra b) = (b\lra a)$
\item $(a\lra a)=1$
\item $(a\lra 1)=a$
\end{enumerate}
% \item Let $f :B\to A$ be a homomorphism, and let $H_f
%   =f^{-1}(1)$. \begin{enumerate}
%   \item Show that $H_f$ is a filter in $B$.  \item Show that $a\lra
%     b\in H_f$ if and only if $f(a)=f(b)$.  \item Show that if $A=2$,
%     then $H_f$ is an ultrafilter in $B$. \end{enumerate}
\item Let $\2P N$ be the powerset of the natural numbers, and let
  $\2U$ be an ultrafilter on $\2P N$.  Show that if $\2U$ contains a
  finite set $F$, then $\2U$ contains a singleton set.
\end{enumerate}
\end{exercise}

\begin{defn} Let $B$ be a Boolean algebra, and let $R$ be an
  equivalence relation on the underlying set of $B$.  We say that $R$
  is a \emph{congruence} just in case $R$ is compatible with the
  operations on $B$ in the following sense: if $aRa'$ and $bRb'$ then
  $(a\wedge b)R(a'\wedge b')$, and $(a\vee b)R(a'\vee b')$, and $(\neg
  a)R(\neg a')$. \end{defn}

In a category $\cat{C}$ with limits (products, equalizers, pullbacks,
etc.), it's possible to formula the notion of an equivalence relation
in $\cat{C}$.  Thus, in $\cat{Bool}$, an equivalence relation $R$ on
$B$ is a subalgebra $R$ of $B\times B$ that satisfies the appropriate
analogues of reflexivity, symmetry, and transitivity.  Since $R$ is a
subalgebra of $B\times B$, it follows in particular that if $\langle
a,b\rangle \in R$, and $\langle a',b'\rangle\in R$, then $\langle
a\wedge a',b\wedge b'\rangle \in R$.  Continuing this reasoning, it's
not difficult to see that congruences, as defined above, are precisely
the equivalence relations in the category $\cat{Bool}$ of Boolean
algebras.  Thus, in the remainder of this chapter, when we speak of an
equivalence relation on a Boolean algebra $B$, we mean an equivalence
relation in $\cat{Bool}$, in other words, a congruence.  (To be clear,
not every equivalence relation on the set $B$ is an equivalence
relation on the Boolean algebra $B$.)

Now suppose that $\cat{C}$ is a category in which equivalence
relations are definable, and let $p_0,p_1:R\rightrightarrows B$ be an
equivalence relation.  [Here $p_0$ and $p_1$ are the projections of
$R$, considered as a subobject of $B\times B$.]  Then we can ask: do
these two maps $p_0$ and $p_1$ have a coequalizer?  That is, is there
an object $B/R$, and a map $q:B\to B/R$, with the relevant universal
property?  In the case of $\cat{Bool}$, a coequalizer can be
constructed directly.  We merely note that the Boolean operations on
$B$ can be used to induce Boolean operations on the set $B/R$ of
equivalence classes.

\begin{defn}[Quotient algebra] Suppose that $R$ is an equivalence
  relation on $B$.  For each $a\in B$, let $E_a$ denote its
  equivalence class, and let $B/R = \{ E_a\mid a\in B\}$.  We then
  define $E_a\wedge E_b=E_{a\wedge b}$, and similarly for $E_a\vee
  E_b$ and $\neg E_a$.  Since $R$ is a congruence (i.e.\ an
  equivalence relation on $\cat{Bool}$), these operations are
  well-defined.  It then follows immediately that $B/R$ is a Boolean
  algebra, and the quotient map $q:B\to B/R$ is a surjective Boolean
  homomorphism. \end{defn}

\begin{lemma} Let $R\subseteq B\times B$ be an equivalence relation.
  Then $q:B\to B/R$ is the coequalizer of the projection maps
  $p_0:R\to B$ and $p_1:R\to B$.  In particular, $q$ is a regular
  epimorphism. \end{lemma}

\begin{proof} It is obvious that $qp_0=qp_1$.  Now suppose that $A$ is
  another Boolean algebra and $f:B\to A$ such that $fp_0=fp_1$.
  Define $g:B/R\to A$ by setting $g(E_x)=f(x)$.  Since $fp_0=fp_1$,
  $g$ is well-defined.  Furthermore, 
  \[ g(E_x\wedge E_y)=g(E_{x\wedge y}) = f(x\wedge y)=f(x)\wedge
  f(y)=g(E_x)\wedge g(E_y) .\] Similarly, $g(\neg E_x)=\neg g(E_x)$.
  Therefore $g$ is a Boolean homomorphism.  Since $q$ is an
  epimorphism, $g$ is the unique homomorphism such that $gq=f$.
  Therefore, $q:B\to B/R$ is the coequalizer of $p_0$ and
  $p_1$. \end{proof}

The category $\cat{Bool}$ has further useful structure: there is a
one-to-one correspondence between equivalence relations and filters.


\begin{lemma} Suppose that $R\subseteq B\times B$ is an equivalence
  relation.  Let $F=\{ a\in B \mid aR1 \}$.  Then $F$ is a filter, and
  $R=\{ \langle a,b\rangle \in B\times B \mid a\lra b\in
  F\}$. \end{lemma}

\begin{proof} Suppose that $a,b\in F$.  That is, $aR1$ and $bR1$.
  Since $R$ is a congruence, $(a\wedge b)R(1\wedge 1)$ and therefore
  $(a\wedge b)R1$.  That is, $a\wedge b\in F$.  Now suppose that $x$
  is an arbitrary element of $B$ such that $a\leq x$.  That is, $x\vee
  a=x$.  Since $R$ is a congruence, $(x\vee a)R(x\vee 1)$ and so
  $(x\vee a)R1$, from which it follows that $xR1$.  Therefore $x\in
  F$, and $F$ is a filter.

  Now suppose that $aRb$.  Since $R$ is reflexive, $(a\vee \neg a)R1$,
  and thus $(b\vee \neg a)R1$.  Similarly $(a\vee \neg b)R1$, and
  therefore $(a\lra b)R1$.  That is, $a\lra b\in F$.
\end{proof}

\begin{lemma} Suppose that $F$ is a filter on $B$.  Let $R=\{ \langle
  a,b\rangle \in B\times B \mid a\lra b\in F\}$.  Then $R$ is an
  equivalence relation, and $F=\{ a\in B\mid aR1\}$.  \end{lemma}

\begin{proof} Showing that $R$ is an equivalence relation requires
  several straightforward verifications.  For example, $a\lra a=1$,
  and $1\in F$, therefore $aRa$.  We leave the remaining verifications
  to the reader.

  Now suppose that $a\in F$.  Since $a=(a\lra 1)$, it follows that
  $a\lra 1\in F$, which means that $aR1$.
\end{proof}


\begin{defn}[Quotient algebra] Let $F$ be a filter on $B$.  Given the
  correspondence between filters and equivalence relations, we write
  $B/F$ for the corresponding algebra of equivalence
  classes. \end{defn}

\begin{prop} Let $F$ be a proper filter on $B$.  Then $B/F$ is a
  two-element Boolean algebra if and only if $F$ is an
  ultrafilter. \end{prop}

\begin{proof} Suppose first that $B/F\cong 2$.  That is, for any $a\in
  B$, either $a\lra 1\in F$ or $a\lra 0\in F$.  But $a\lra 1 =a$ and
  $a\lra 0=\neg a$.  Therefore, either $a\in F$ or $\neg a\in F$, and
  $F$ is an ultrafilter.

  Suppose now that $F$ is an ultrafilter.  Then for any $a\in B$,
  either $a\in F$ or $\neg a\in F$.  In the former case, $a\lra 1\in
  F$.  In the latter case, $a\lra 0\in F$.  Therefore, $B/F\cong 2$.
\end{proof}


% Suppose that $f:B\to A$ is a homomorphism of Boolean algebras, and
% consider the subset 
% \[ E_f \: = \: \{ \langle a,b\rangle \in B\times B \mid f(a)=f(b) \}
% .\] It is easy to check that $E_f$ is an equivalence relation on $A$.
% The following proposition is then a straightforward verification.

% \begin{prop} Let $R$ be an equivalence relation on $B$ in
%   $\cat{Bool}$.  Then $R\cong E_f$, where $f:B\to B/R$ is the
%   canonical surjection. \end{prop}


% \begin{prop} Let $f:B\to A$ be a homomorphism of Boolean algebras.
%   Then $f=me$, where $e:B\to B/E_f$ is a regular epimorphism, and
%   $m:B/E_f\to A$ is a monomorphism. \end{prop}

% \[ \begin{tikzcd}
%   B \arrow{rr}{f} \arrow[->>]{dr}{e} & & A  \\
%   & B/E_f \arrow[>->]{ur}{m} \end{tikzcd} \]


% \begin{prop} Let $f:B\to A$ be a homomorphism, let $F\subseteq B$ be a
%   filter, and let $p:B\epi B/F$ be the corresponding epimorphism.
%   Then $f$ factorizes through $p$ if and only if $F\subseteq E_f=\{
%   a\mid f(a)=1 \}$. \end{prop}

% \begin{proof} Suppose first that $F\subseteq E_f$.  Let $e:B\epi
%   B/E_f$ and $m:B/E_f\monic A$ be the canonical factorization of $f$.
%   Let $E = \{ \langle a,b\rangle \mid a\lra b\in F \}$ be the
%   equivalence relation corresponding to $F$, and let $E'$ be the
%   equivalence relation corresponding to $E_f$.  Then $E\subseteq E'$,
%   and there is a unique surjection $g$ of the set $B/E$ onto the set
%   $B/E'$ such that $gp=e$.  It is straightforward to verify that
%   $g:B/F\to B/E_f$ is a Boolean homomorphism.
% \[ \begin{tikzcd} 
% B \arrow{r}{f} \arrow{d}{p} \arrow{dr}{e} & A  \\
% B/F \arrow{r}{g} & B/E_f \arrow{u}{m} \end{tikzcd} \]
% Finally, $f=me=mgp$, and $f$ factorizes through $p$. 

% Suppose now that $f$ factorizes through $p$, in particular, $f=hp$.
% Then for any $a\in B$, if $a\in F$, then $p(a)=1$, from which
% $f(a)=h(p(a))=h(1)=1$.  Thus, $a\in E_f$, and $F\subseteq E_f$.
% \end{proof}

\begin{exercise} (This exercise presupposes knowledge of measure
  theory.)  Let $\Sigma$ be the Boolean algebra of Borel subsets of
  $[0,1]$, and let $\mu$ be Lebesgue measure on $[0,1]$.  Let $\2F=\{
  S\in \Sigma \mid \mu (S) = 1\}$.
\begin{enumerate}
\item Show that $\2F$ is a filter.
\item Describe the equivalence relation on $\Sigma$ corresponding to
  $\2F$.
\end{enumerate}
\end{exercise}



According to our motivating analogy, a Boolean algebra $B$ is like a
theory, and a homorphism $\phi :B\to 2$ is like a model of this
theory.  We say that the algebra $B$ is \emph{syntactically
  consistent} just in case $0\neq 1$.  (In fact, we defined Boolean
algebras so as to require syntactic consistency.)  We say that the
algebra $B$ is \emph{semantically consistent} just in case there is a
homomorphism $\phi :B\to 2$.  Then semantic consistency clearly
implies syntactic consistency.  But does syntactic consistency imply
semantic consistency?

It's at this point that we have to invoke a powerful theorem --- or,
perhaps more accurately, a powerful set-theoretic axiom.  In short, if
we use the axiom of choice, or some equivalent such as Zorn's lemma,
then we can prove that every syntactically consistent Boolean algebra
is semantically consistent.  However, we do not actually need the full
power of the Axiom of Choice.  As set-theorists know, the Boolean
ultrafilter axiom (``UF'' for short) is strictly weaker than the axiom
of choice.\footnote{Algebraists often invoke an equivalent axiom
  called the Boolean Prime Ideal Theorem (BPI).  It's often called a
  ``Theorem'' because it can be proven from the Axiom of Choice, or
  equivalently, Zorn's Lemma.  But BPI can also be taken as an axiom,
  in which case it's strictly weaker than AC.}

\begin{prop} The following are equivalent:
  \begin{enumerate}
  \item \textbf{Boolean ultrafilter axiom (UF)} For any Boolean
    algebra $B$, there is a homomorphism $f:B\to 2$.
  \item For any Boolean algebra $B$, and proper filter $F\subseteq B$,
    there is a homomorphism $f:B\to 2$ such that $f(a)=1$ when $a\in
    F$.
  \item For any Boolean algebra $B$, if $a,b\in B$ such that $a\neq
    b$, then there is a homomorphism $f:B\to 2$ such that $f(a)\neq
    f(b)$.
\item For any Boolean algebra $B$, if $\phi (a)=1$ for all $\phi :B\to
  2$, then $a=1$.
\item For any two Boolean algebras $A,B$, and homomorphisms
  $f,g:A\rightrightarrows B$, if $\phi f=\phi g$ for all $\phi :B\to
  2$, then $f=g$. 
\end{enumerate} \end{prop}

\begin{proof} ($1\Rightarrow 2$) Suppose that $F$ is a proper filter
  in $B$.  Then there is a homomorphism $q:B\to B/F$ such that
  $q(a)=1$ for all $a\in F$.  By UF, there is a homomorphism $\phi
  :B/F\to 2$.  Therefore, $\phi \circ q:B\to 2$ is a homomorphism such
  that $(\phi \circ q)(a)=1$ for all $a\in F$.

  \bigskip ($1\Rightarrow 3$) Suppose that $a,b\in B$ with $a\neq b$.
  Then either $\neg a\wedge b\neq 0$ or $a\wedge \neg b\neq 0$.
  Without loss of generality, we assume that $\neg a\wedge b\neq 0$.
  In this case, the filter $F$ generated by $\neg a\wedge b$ is
  proper.  By UF, there is a homomorphism $\phi :B\to 2$ such that
  $\phi (x)=1$ when $x\in F$.  In particular, $\phi (\neg a\wedge
  b)=1$.  But then $\phi (a)=0$ and $\phi (b)=1$.

  \bigskip ($2\Rightarrow 4$) Suppose that $\phi (a)=1$ for all $\phi
  :B\to 2$.  Now let $F$ be the filter generated by $\neg a$.  If $F$
  is proper, then by (2), there is a $\phi :B\to 2$ such that $\phi
  (\neg a)=1$, a contradiction.  Thus, $F=B$, which implies that $\neg
  a=0$ and $a=1$.

  \bigskip ($4\Rightarrow 5$) Let $f,g:A\to B$ be homomorphisms, and
  suppose that for all $\phi :B\to 2$, $\phi f=\phi g$.  That is, for
  each $a\in A$, $\phi (f(a))=\phi (g(a))$.  But then $\phi (f(a)\lra
  g(a))=1$ for all $\phi :B\to 2$.  By (4), $f(a)\lra g(a)=1$, and
  therefore $f(a)=g(a)$.

  \bigskip ($5\Rightarrow 3$) Let $B$ be a Boolean algebra, and
  $a,b\in B$.  Suppose that $\phi (a)=\phi (b)$ for all $\phi :B\to
  2$.  Let $F$ be the four element Boolean algebra, with generator
  $p$.  Then there is a homomorphism $\hat{a}:F\to B$ such that
  $\hat{a}(p)=a$, and a homomorphism $\hat{b}:F\to B$ such that
  $\hat{b}(p)=b$.  Thus, $\phi \hat{a}=\phi \hat{b}$ for all $\phi
  :B\to 2$.  By (5), $\hat{a}=\hat{b}$, and therefore $a=b$.

  \bigskip ($3\Rightarrow 1$) Let $B$ be an arbitrary Boolean algebra.
  Since $0\neq 1$, (3) implies that there is a homomorphism $\phi
  :B\to 2$.
\end{proof}

We are finally in a position to prove the completeness of the
propositional calculus.  The following result assumes the Boolean
ultrafilter axiom (UF).

\begin{box-thm}[Completeness Theorem] If $T\vDash \phi$ then $T\vdash
  \phi$. \end{box-thm} \label{page:complete}

\begin{proof} Suppose that $T\not\vdash\phi$.  Then in the Lindenbaum
  algebra $L(T)$, we have $E_\phi\neq 1$.  In this case, there is a
  homomorphism $h:L(T)\to 2$ such that $h(E_\phi )=0$.  Hence, $h\circ
  i_T$ is a model of $T$ such that $(h\circ i_T)(\phi )=h(E_\phi )=0$.
  Therefore, $T\not\vDash\phi$.  \end{proof}

\begin{exercise} Let $\2P N$ be the powerset of the natural numbers.
  We say that a subset $E$ of $N$ is \emph{cofinite} just in case
  $N\backslash E$ is finite.  Let $\2F\subseteq \2P N$ be the set of
  cofinite subsets of $N$.
\begin{enumerate}
\item Show that $\2F$ is a filter.
\item Show that there are infinitely many ultrafilters containing
  $\2F$.
\end{enumerate}
\end{exercise}


\section{Stone spaces}

Philosophers talk about the set of all possible worlds.  Logicians and
mathematicians talk about ``sets of models''. Physicists talk about
``models of a theory''.  It's all pretty much the same thing, at least
from an abstract point of view.  But if we're going to undertake an
exact study of the set of possible worlds, then we need to make a
proposal about what structure this space carries.  But what do I mean
her by ``structure''?  Isn't the space of possible worlds just a bare
set?  Let me explain a couple of reasons why we might want to think of
the space of possible worlds as a structured set, and in particular,
as a topological space.

Suppose that there are infinitely many possible worlds, which we
represent by elements of a set $X$.  As philosophers are wont to do,
we then represent \emph{propositions} by subsets of $X$.  But should
we think that all $2^{|X|}$ subsets of $X$ correspond to genuine
propositions?  What would warrant such a claim?

There is another reason to worry about this approach.  For a person
with training in set theory, it is not difficult to build a collection
$C_1,C_2,\dots $ of subsets of $X$ with the following features: (i)
each $C_i$ is non-empty, (ii) $C_{i+1}\subseteq C_i$ for all $i$, and
(iii) $\bigcap _iC_i$ is empty.  Intuitively speaking, $\{ C_i\mid
i\in \7N \}$ is a family of propositions that are individually
consistent (since non-empty), that are becoming more and more
specific, and yet there is no world in $X$ that makes all $C_i$ true.
Why not?  It seems that $X$ is missing some worlds!  Indeed, here's a
description of a new world $w$ that does not belong to $X$: for each
proposition $\phi$, let $\phi$ be true in $w$ if and only if $\phi\cap
C_i$ is nonempty for all $i$.  It's not difficult to see that $w$ is
in fact a truth valuation on the set of all propositions, i.e.\ it is
a possible world.  But $w$ is not represented by a point in $X$.  What
we have here is a mismatch between the set $X$ of worlds, and the set
of propositions describing these worlds.

The idea behind logical topology is that not all subsets of $X$
correspond to propositions.  A designation of a topology on $X$ is
tantamount to saying which subsets of $X$ correspond to propositions.
However, the original motivation for the study of topology comes from
geometry (and analysis), not from logic.  Recall high school
mathematics, where you learned that a continuous function is one where
you don't have to lift your pencil from the paper in order to draw the
graph.  If your high school class was really good, or if you studied
calculus in college, then you will have learned that there is a more
rigorous definition of a continuous function --- a definition
involving epsilons and deltas.  In the early 20th century, it was
realized that the essence of continuity is even more abstract than
epsilons and deltas would suggest: all we need is a notion of nearness
of points, which we can capture in terms of a notion of a neighborhood
of a point.  The idea then is that a function $f:X\to Y$ is continuous
at a point $x$ just in case for any neighborhood $V$ of $f(x)$, there
is some neighborhood $U$ of $x$ such that $f(U)\subseteq V$.
Intuitively speaking, $f$ preserves closeness of points.

Notice, however, that if $X$ is an arbitrary set, then it's not
obvious what ``closeness'' means.  To be able to talk about closeness
of points in $X$, we need specify which subsets of $X$ count as the
neighborhoods of points.  Thus, a \emph{topology} on $X$ is a set of
subsets of $X$ that satisfies certain conditions.

\begin{defn} A \emph{topological space} is a set $X$ and a family
  $\2F$ of subsets of $X$ satisfying the following conditions:
\begin{enumerate}
\item $\emptyset\in \2F$ and $X\in \2F$;
\item If $U,V\in \2F$ then $U\cap V\in \2F$;
\item If $\2F _0$ is a subfamily of $\2F$, then $\bigcup _{U\in \2F
    _0}U\in \2F$.
\end{enumerate}
The sets in $\2F$ are called \emph{open subsets} of the space $(X,\2F
)$.  If $p\in U$ with $U$ an open subset, we say that $U$ is a
\emph{neighborhood} of $p$.
\end{defn}


%% refer to Steen and Seebach

There are many familiar examples of topological spaces.  In many
cases, however, we only know the open sets indirectly, by means of
certain nice open sets.  For example, in the case of the real numbers,
not every open subset is an interval.  However, every open subset is a
union of intervals.  In that case, we call the open intervals in
$\mathbb{R}$ a \emph{basis} for the topology.

\begin{prop} Let $\2B$ be a family of subsets of $X$ with the property
  that if $U,V\in \2B$ then $U\cap V\in \2B$.  Then there is a unique
  smallest topology $\2F$ on $X$ containing $\2B$.  \end{prop}

\begin{proof} Let $\2F$ be the collection obtained by taking all
  unions of sets in $\2B$, and then taking finite intersections of the
  resulting collection.  Clearly $\2F$ is a topology on $X$, and any
  topology on $X$ containing $\2B$ also contains $\2F$. \end{proof}

\begin{defn} If $\2B$ is a family of subsets of $X$ that is closed
  under intersection, and if $\2F$ is the topology generated by $\2B$,
  then we say that $\2B$ is a \emph{basis} for $\2F$. \end{defn}

\begin{prop} Let $(X,\2F )$ be a topological space.  Let $\2F_0$ be a
  subfamily of $\2F$ with the following properties: (1) $\2F_0$ is
  closed under finite intersections, and (2) for each $x\in X$ and
  $U\in \2F_0$ with $x\in U$, there is a $V\in\2F _0$ such that $x\in
  V\subseteq U$.  Then $\2F _0$ is a basis for the topology
  $\2F$. \end{prop}

\begin{proof} We need only show that each $U\in\2F$ is a union of
  elements in $\2F _0$.  And that follows immediately from the fact
  that if $x\in U$, then there is $V\in \2F _0$ with $x\in V\subseteq
  U$. \end{proof}

\newcommand{\cl}[1]{\overline{#1}}

\begin{defn} Let $X$ be a topological space.  A subset $C$ of $X$ is
  called \emph{closed} just in case $C=X\backslash U$ for some open
  subset $U$ of $X$.  The intersection of closed sets is closed.
  Hence, for each subset $E$ of $X$, there is a unique smallest closed
  set $\cl{E}$ containing $E$, namely the intersection of all closed
  supersets of $E$.  We call $\overline{E}$ the \emph{closure} of
  $E$.  \end{defn}


\begin{prop} Let $p\in X$ and let $S\subseteq X$.  Then $p\in
  \overline{S}$ if and only if every open neighborhood $U$ of $p$ has
  nonempty intersection with $S$.  \end{prop}

\begin{proof} Exercise. \end{proof}

\begin{defn} Let $S$ be a subset of $X$.  We say that $S$ is
  \emph{dense} in $X$ just in case $\overline{S}=X$. \end{defn}

\begin{defn} Let $E\subseteq X$.  We say that $p$ is a \textbf{limit
    point} of $E$ just in case for each open neighborhood $U$ of $p$,
  $U\cap E$ contains some point besides $p$.  We let $E'$ denote the
  set of all limit points of $E$.  \end{defn}

\begin{lemma} $E'\subseteq \overline{E}$.  \end{lemma}

\begin{proof} Let $p\in E'$, and let $C$ be a closed set containing
  $E$.  If $p\in X\backslash C$, then $p$ is contained in an open set
  that has empty intersection with $E$.  Thus, $p\in C$.  Since $C$
  was an arbitrary closed superset of $E$, it follows that
  $p\in\overline{E}$.
\end{proof}

\begin{prop} $\overline{E}=E\cup E'$ \end{prop}

\begin{proof} The previous lemma gives $E'\subseteq \overline{E}$.
  Thus, $E\cup E'\subseteq \overline{E}$.  

  Suppose now that $p\not\in E$ and $p\not\in E'$.  Then there is an
  open neighborhood $U$ of $p$ such that $U\cap E$ is empty.  Then
  $E\subseteq X\backslash U$, and since $X\backslash U$ is closed,
  $\overline{E}\subseteq X\backslash U$.  Therefore $p\not\in
  \overline{E}$.




\end{proof}

\begin{defn} A topological space $X$ is said to be:
\begin{itemize}
\item $T_1$ or \emph{Frechet} just in case all singleton
  subsets are closed.
\item $T_2$ or \emph{Hausdorff} just in case for any $x,y\in X$
  if $x\neq y$ then there are disjoint open neighborhoods of $x$ and
  $y$.
\item $T_3$ or \emph{regular} just in case for each $x\in X$, and for
  each closed $C\subseteq X$ such that $x\not\in C$, there are open
  neighborhoods $U$ of $x$, and $V$ of $C$, such that $U\cap
  V=\emptyset$.
\item $T_4$ or \emph{normal} just in case any two disjoint closed
  subsets of $X$ can be separated by disjoint open
  sets.  \end{itemize} \end{defn}

Clearly we have the implications 
\[ (T_1+T_4)\Rightarrow (T_1+T_3)\Rightarrow T_2 \Rightarrow T_1\] A
discrete space satisfies all of the separation axioms.  A non-trivial
indiscrete space satisfies none of the separation axioms.  A useful
heuristic here is that the stronger the separation axiom, the closer
the space is to discrete.  In this book, most of the spaces we
consider are very close to discrete in a precise sense we will
describe below.


\begin{exercise} \mbox{ }
\begin{enumerate}
\item Show that $X$ is regular iff for each $x\in X$ and open
  neighborhood $U$ of $x$, there is an open neighborhood $V$ of $x$
  such that $\cl{V}\subseteq U$.
\item Show that if $E\subseteq F$ then $\overline{E}\subseteq
  \overline{F}$.
\item Show that $\cl{\cl{E}}=\cl{E}$.
% \item Show that in the space $\mathbb{R}$, the only clopen subsets are
%   $\emptyset$ and $\mathbb{R}$.
\item Show that the intersection of two topologies is a topology.
\item Show that the infinite distributive law holds:
\[ U\cap \left( \bigcup _{i\in I}V_i \right) \: = \: \bigcup  _{i\in
  I}(U\cap V_i ) .\]
% \item Show that a space $X$ is Hausdorff if and only if the diagonal
%   $\Delta = \{ \langle x,x\rangle :x\in X\}$ is closed in the product
%   topology on $X\times X$.  [Oops: you cannot solve this exercise
%   until you know what the product topology is!]
\end{enumerate} \end{exercise}


\begin{defn} Let $S\subseteq X$.  A family $\2C$ of open subsets of
  $X$ is said to \emph{cover} $S$ just in case $S\subseteq \bigcup
  _{U\in \2C}U$.  We say that $S$ is \emph{compact} just in case for
  every open cover $\2C$ of $S$, there is a finite subcollection $\2C
  _0$ of $\2C$ that also covers $S$.  We say that the space $X$ is
  compact just in case it's compact as a subset of itself.  \end{defn}

\begin{defn} A collection $\2C$ of subsets of $X$ is said to satisfy
  the \emph{finite intersection property} if for every finite
  subcollection $C_1,\dots ,C_n$ of $\2C$, the intersection $C_1\cap
  \cdots \cap C_n$ is nonempty. \end{defn}

\begin{disc} Suppose that $X$ is the space of possible worlds, so that
  we can think of subsets of $X$ as propositions.  If $A\cap B$ is
  nonempty, then the propositions $A$ and $B$ are consistent, i.e.\
  there is a world in which they are both true.  Thus, a collection
  $\2C$ of propositions has the finite intersection property just in
  case it is finitely consistent.
\end{disc}

Recall that compactness of propositional logic states that if a set
$\2C$ of propositions is finitely consistent, then $\2C$ is
consistent.  The terminology here is no accident; a topological space
is compact just in case finite consistency entails consistency.

\begin{prop} A space $X$ is compact if and only if for every
  collection $\2C$ of closed subsets of $X$, if $\2C$ satisfies the
  finite intersection property, then $\bigcap \2C$ is
  nonempty. \end{prop}

\begin{proof} ($\Rightarrow$) Assume first that $X$ is compact, and
  let $\2C$ be a family of closed subsets of $X$.  We will show that
  if $\2C$ satisfies the finite intersection property, then the
  intersection of all sets in $\2C$ is nonempty.  Assume the negation
  of the consequent, i.e.\ that $\bigcap _{C\in\2C}C$ is empty.  Let
  $\2C '=\{ C' :C\in \2C \}$, where $C'=X\backslash C$ is the
  complement of $C$ in $X$.  [Warning: this notation can be confusing.
  Previously I used $E'$ to denote the set of limit points of $E$.
  This $C'$ has nothing to do with limit points.]  Each $C'$ is open,
  and
$$ \left( \bigcup _{C\in \2C}C' \right) ' \: = \: \bigcap _{C\in\2C}C ,$$
which is empty.  It follows then that $\2C '$ is an open cover of $X$.
Since $X$ is compact, there is a finite subcover $\2C '_0$ of $\2C '$.
If we let $\2C _0$ be the complements of sets in $\2C '_0$, then $\2C
_0$ is a finite collection of sets in $\2C$ whose intersection is
empty.  Therefore $\2C$ does not satisfy the finite intersection
property.

($\Leftarrow$) Assume now that $X$ is not compact.  In particular,
suppose that $\2U$ is an open cover with no finite
subcover.  Let $\2C = \{ X\backslash U\mid U\in \2U \}$.  For any
finite subcollection $X\backslash U_1,\dots ,X\backslash U_n$ of
$\2C$, we have 
\[ U_1\cup\cdots\cup U_n \neq X ,\]
and hence 
\[ (X\backslash U_1)\cap \cdots \cap (X\backslash U_n)\neq \emptyset
.\] Thus, $\2C$ has the fip.  Nonetheless, since $\2U$ covers $X$, the
intersection of all sets in $\2C$ is empty.  \end{proof}

\begin{prop} In a compact space, closed subsets are
  compact.  \label{closed-compact} \end{prop}

\begin{proof} Let $\2C$ be an open cover of $S$, and consider the
  cover $\2C' = \2C \cup \{ X\backslash S \}$ of $X$.  Since $X$ is
  compact, there is a finite subcover $\2C _0$ of $\2C '$.  Removing
  $X\backslash S$ from $\2C _0$ gives a finite subcover of the
  original cover $\2C$ of $S$. \end{proof}


\begin{prop} Suppose that $X$ is compact, and let $U$ be an open set
  in $X$.  Let $\{ F_i\}_{i\in I}$ be a family of closed subsets of
  $X$ such that $\bigcap _{i\in I}F_i\subseteq U$.  Then there is a
  finite subset $J$ of $I$ such that $\bigcap _{i\in J}F_i\subseteq
  U$. \label{fin-down} \end{prop}

\begin{proof} Let $C=X\backslash U$, which is closed.  Thus, the
  hypotheses of the proposition say that the family $\2C :=\{ C\}\cup
  \{ F_i:i\in I\}$ has empty intersection.  Since $X$ is compact,
  $\2C$ also fails to have the finite intersection property.  That is,
  there are $i_1,\dots ,i_k\in I$ such that $C\cap
  F_{i_1}\cap\cdots\cap F_{i_k}=\emptyset$.  Therefore
  $F_{i_1}\cap\cdots\cap F_{i_k}\subseteq U$. \end{proof}


\begin{prop} If $X$ is compact Hausdorff, then $X$ is
  regular. \label{chaus-regular} \end{prop}

\begin{proof} Let $x\in X$, and let $C\subseteq X$ be closed.  For
  each $y\in C$, let $U_y$ be an open neighborhood of $x$, and $V_y$
  an open neighborhood of $y$ such that $U_y\cap V_y=\emptyset$.  The
  $V_y$ form an open cover of $C$.  Since $C$ is closed and $X$ is
  compact, $C$ is compact.  Hence there is a finite subcollection
  $V_{y_1},\dots ,V_{y_n}$ that cover $C$.  But then $U=\cap
  _{i=1}^nU_{y_i}$ is an open neighborhood of $x$, and $V=\cup
  _{i=1}^nV_{y_i}$ is an open neighborhood of $C$, such that $U\cap
  V=\emptyset$.  Therefore $X$ is regular. \end{proof}



\begin{prop} In Hausdorff spaces, compact subsets are
  closed. \label{haus-cc} \end{prop}

\begin{proof} Let $p$ be a point of $X$ that is not in $K$.  Since $X$
  is Hausdorff, for each $x\in K$, there are open neighborhoods $U_x$
  of $x$ and $V_x$ of $p$ such that $U_x\cap V_x=\emptyset$.  The
  family $\{ U_x:x\in K\}$ covers $K$.  Since $K$ is compact, it is
  covered by a finite subcollection $U_{x_1},\dots ,U_{x_n}$.  But
  then $\cap _{i=1}^nV_{x_i}$ is an open neighborhood of $p$ that is
  disjoint from $K$.  It follows that $X-K$ is open, and $K$ is
  closed. \end{proof}


\begin{defn} Let $X,Y$ be topological spaces.  A function $f:X\to Y$
  is said to be \emph{continuous} just in case for each open subset
  $U$ of $Y$, $f^{-1}(U)$ is an open subset of $X$.  \end{defn}

\begin{example} Let $f:\mathbb{R}\to \mathbb{R}$ be the function that
  is constantly zero on $(-\infty ,0)$, and $1$ on $[0,\infty )$.
  Then $f$ is not continuous:
  $f^{-1}(\frac{1}{2},\frac{3}{2})=[0,\infty )$, which is not
  open. \end{example}

In the exercises, you will show that a function $f$ is continuous if
and only if $f^{-1}(C)$ is closed whenever $C$ is closed.  Thus, in
particular, if $C$ is a clopen subset of $Y$, then $f^{-1}(C)$ is a
clopen subset of $X$.


\begin{prop} Let $\mathbf{Top}$ consist of the class of topological
  spaces and continuous maps between them.  For $X\stackrel{f}{\to}
  Y\stackrel{g}{\to }Z$, define $g\circ f$ to be the composition of
  $g$ and $f$.  Then $\mathbf{Top}$ is a category. \end{prop}

\begin{proof} It needs to be confirmed that if $f$ and $g$ are
  continuous, then $g\circ f$ is continuous.  We leave this to the
  exercises.  Since composition is associative, $\cat{Top}$ is a
  category. \end{proof}


\begin{prop} Suppose that $f:X\to Y$ is continuous.  If $K$ is compact
  in $X$, then $f(K)$ is compact in
  $Y$. \label{image-compact} \end{prop}

\begin{proof} Let $\2G$ be a collection of open subsets of $Y$ that
  covers $f(K)$.  Let
  \[ \2G ' = \{ f^{-1}(U) :U\in \2G \} .\] When $\2G '$ is an open
  cover of $K$.  Since $K$ is compact, $\2G '$ has a finite subcover
  $f^{-1}(U_1),\dots ,f^{-1}(U_n)$.  But then $U_1,\dots ,U_n$ is a
  finite subcover of $\2G$. \end{proof}

We remind the reader of the category theoretic definitions:
\begin{itemize}
\item $f$ is a \emph{monomorphism} just in case $fh=fk$ implies $h=k$.
\item $f$ is an \emph{epimorphism} just in case $hf=kf$ implies $h=k$.
\item $f$ is an \emph{isomorphism} just in case there is a $g:Y\to X$
  such that $gf=1_X$ and $fg=1_Y$.
\end{itemize} For historical reasons, isomorphisms in $\mathbf{Top}$
are usually called \emph{homeomorphisms}.  It is easy to show that a
continuous map $f:X\to Y$ is monic if and only if $f$ is injective.
It is also true that $f:X\to Y$ is epi if and only if $f$ is
surjective (but the proof is somewhat subtle).  In contrast, a
continuous bijection is not necessarily an isomorphism in $\cat{Top}$.
For example, if we let $X$ be a two element set with the discrete
topology, and $Y$ be a two element set with the indiscrete topology,
then any bijection $f:X\to Y$ is continuous, but is not an
isomorphism.

% \begin{prop} In $\mathbf{Top}$, epimorphisms are
%   surjective. \end{prop}

% \begin{proof} Let $f:X\to Y$ be epi.  Let $Z=\{ 0,1\}$ with the
%   indiscrete topology, let $h:Y\to Z$ be constantly $1$, and let
%   $k:Y\to Z$ be the characteristic function of $f(Y)$.  Then $hf=kf$,
%   and since $f$ is epi, $h=k$.  Therefore, $f(X)=Y$.  \end{proof}

%% following example: $f$ is epi in $\mathbf{Haus}$, but not in
%% $\mathbf{Top}$.




\begin{exercise} \mbox{ } 
\begin{enumerate}
\item Show that if $f$ and $g$ are continuous, then $g\circ f$ is
continuous.
\item Suppose that $f:X\to Y$ is a surjection.  Show that if $E$ is
  dense in $X$, then $f(E)$ is dense in $Y$.
\item Show that $f:X\to Y$ is continuous if and only if $f^{-1}(C)$ is
  closed whenever $C$ is closed.
\item Let $Y$ be a Hausdorff space, and let $f,g:X\to Y$ be
  continuous.  Show that if $f$ and $g$ agree on a dense subset $S$ of
  $X$, then $f=g$.
\end{enumerate} \end{exercise}

\begin{exercise} Show that $f^{-1}(V)\subseteq U$ if and only if $V\subseteq
Y\backslash f(X\backslash U)$. \end{exercise}

\begin{defn} A continuous mapping $f:X\to Y$ is said to be
  \emph{closed} just in case for every closed set $C\subseteq X$ the
  image $f(C)$ is closed in $Y$.  Similarly, $f:X\to Y$ is said to be
  \emph{open} just in case for every open set $U\subseteq X$, the
  image $f(U)$ is open in $Y$. \end{defn}

\begin{prop} Let $f:X\to Y$ be continuous.  Then the following are
  equivalent.
\begin{enumerate}
\item $f$ is closed.
\item For every open set $U\subseteq X$, the set $\{ y\in Y\mid
  f^{-1}\{ y\}\subseteq U\}$ is open.
\item For every $y\in Y$, and every neighborhood $U$ of $f^{-1}\{
  y\}$, there is a neighborhood $V$ of $y$ such that
  $f^{-1}(V)\subseteq U$.
\end{enumerate}
\label{closed-map} \end{prop}


\begin{proof} ($2\Leftrightarrow 3$) The equivalence of (2) and (3) is
  straightforward, and we leave its proof as an exercise.

  \bigskip ($3\Rightarrow 1$) Suppose that $f$ satisfies condition
  (3), and let $C$ be a closed subset of $X$.  To show that $f(C)$
  closed, assume that $y\in Y\backslash f(C)$.  Then $f^{-1}\{ y\}
  \subseteq X\backslash C$.  Since $X\backslash C$ is open, there is a
  neighborhood $V$ of $y$ such that $f^{-1}(V)\subseteq U$.  Then
  \[ V \: \subseteq \: Y\backslash f(X\backslash U) \: = \:
  Y\backslash f(C) .\] Since $y$ was an arbitrary element of
  $Y\backslash f(C)$, it follows that $Y\backslash f(C)$ is open, and
  $f(C)$ is closed.

  \bigskip ($1\Rightarrow 3$) Suppose that $f$ is closed.  Let $y\in
  Y$, and let $U$ be a neighborhood of $f^{-1}\{ y\}$.  Then
  $X\backslash U$ is closed, and $f(X\backslash U)$ is also closed.
  Let $V=Y\backslash f(X\backslash U)$.  Then $V$ is an open
  neighborhood of $y$ and $f^{-1}(V)\subseteq U$. \end{proof}

\begin{prop} Suppose that $X$ and $Y$ are compact Hausdorff.  If
  $f:X\to Y$ is continuous, then $f$ is a closed
  map. \label{auto-closed} \end{prop}

\begin{proof} Let $B$ be a closed subset of $X$.  By Proposition
  \ref{closed-compact}, $B$ is compact.  By Proposition
  \ref{image-compact}, $f(B)$ is compact.  And by Proposition
  \ref{haus-cc}, $f(B)$ is closed.  Therefore, $f$ is a closed map.
\end{proof}



\begin{prop} Suppose that $X$ and $Y$ are compact Hausdorff.  If
  $f:X\to Y$ is a continuous bijection, then $f$ is an
  isomorphism. \label{chaus-bi} \end{prop}

\begin{proof} Let $f:X\to Y$ be a continuous bijection.  Thus, there
  is function $g:Y\to X$ such that $gf=1_X$ and $fg=1_Y$.  We will
  show that $g$ is continuous.  By Proposition \ref{auto-closed}, $f$
  is closed.  Moreover, for any closed subset $B$ of $X$, we have
  $g^{-1}(B)=f(B)$.  Thus, $g^{-1}$ preserves closed subsets, and
  hence $g$ is continuous.
\end{proof}



% \begin{defn} We say that a space $X$ is \emph{totally disconnected} if
%   the only connected subsets are singletons (equivalently, all
%   components are singletons).  We say that $X$ is a \emph{Stone space}
%   if $X$ is compact, Hausdorff, and totally disconnected. \end{defn}

\begin{defn} A topological space $X$ is said to be \emph{totally
    separated} if for any $x,y\in X$, if $x\neq y$ then there is a
  closed and open (clopen) subset of $X$ containing $x$ but not
  $y$. \end{defn}

\begin{defn} We say that $X$ is a \emph{Stone space} if $X$ is compact
  and totally separated.  We let $\cat{Stone}$ denote the full
  subcategory of $\cat{Top}$ consisting of Stone spaces.  To say that
  $\cat{Stone}$ is a full subcategory means that the arrows between
  two Stone spaces $X$ and $Y$ are just the arrows between $X$ and $Y$
  considered as topological spaces, i.e.\ continuous
  functions.  \end{defn}

\begin{note} Let $E$ be a clopen subset of $X$.  Then there is a
  continuous function $f:X\to \{ 0,1\}$ such that $f(x)=1$ for $x\in
  E$, and $f(x)=0$ for $x\in X\backslash E$.  Here we are considering
  $\{ 0,1\}$ with the discrete topology. \end{note}

%% TO DO

\begin{prop} Let $X$ and $Y$ be Stone spaces.  If $f:X\to Y$ is an
  epimorphism, then $f$ is surjective. \label{stones} \end{prop}

\begin{proof} Suppose that $f$ is not surjective.  Since $X$ is
  compact, the image $f(X)$ is compact in $Y$, hence closed.  Since
  $f$ is not surjective, there is a $y\in Y\backslash f(X)$.  Since
  $Y$ is a regular space, there is a clopen neighborhood $U$ of $y$
  such that $U\cap f(X)=\emptyset$.  Define $g:Y\to \{ 0,1\}$ to be
  constantly $0$.  Define $h:Y\to \{ 0,1\}$ to be $1$ on $U$, and $0$
  on $Y\backslash U$.  Then $g\circ f=h\circ f$, but $g\neq h$.
  Therefore $f$ is not an epimorphism.
\end{proof}

\begin{prop} Let $X$ and $Y$ be Stone spaces.  If $f:X\to Y$ is both a
  monomorphism and an epimorphism, then $f$ is an
  isomorphism. \label{stone-balanced} \end{prop}

\begin{proof} By Proposition \ref{stones}, $f$ is surjective.
  Therefore, $f$ is a continuous bijection.  By Proposition
  \ref{chaus-bi}, $f$ is an isomorphism.
\end{proof}





\section{Stone duality} \label{sec:stone}

In this section we show that the category $\cat{Bool}$ is dual to a
certain category of topological spaces, namely the category
$\cat{Stone}$ of Stone spaces.  To say that categories are ``dual''
means that the first is equivalent to the mirror image of the second.

\begin{defn} We say that categories $\cat{C}$ and $\cat{D}$ are
  \emph{dual} just in case there are contravariant functors
  $F:\cat{C}\to\cat{D}$ and $G:\cat{D}\to\cat{C}$ such that $GF\cong
  1_{\cat{C}}$ and $FG\cong 1_{\cat{D}}$.  To see that this definition
  makes sense, note that if $F$ and $G$ are contravariant functors,
  then $GF$ and $FG$ are covariant functors.  If $\cat{C}$ and
  $\cat{D}$ are dual, we write $\cat{C}\cong \cat{D}^{op}$, to
  indicate that $\cat{C}$ is equivalent to the opposite category of
  $\cat{D}$, i.e.\ the category that has the same objects as
  $\cat{D}$, but arrows running in the opposite direction. \end{defn}


\subsection*{The functor from Bool to Stone}

We now define a contravariant functor $S:\cat{Bool}\to\cat{Stone}$.
For reasons that will become clear later, the functor $S$ is sometimes
called the \emph{semantic functor}.

Consider the set $\hom (B,2)$ of $2$-valued homomorphisms of the
Boolean algebra $B$.  For each $a\in B$, define
\[ C_a \: = \: \{ \phi \in \hom (B,2) \mid \phi (a)=1 \} .\] Clearly,
the family $\{ C_a\mid a\in B\}$ forms a basis for a topology on $\hom
(B,2)$.  We let $S(B)$ denote the resulting topological space.  Note
that $S(B)$ has a basis of clopen sets.  Thus, if $S(B)$ is compact,
then $S(B)$ is a Stone space.

\begin{lemma} If $B$ is a Boolean algebra, then $S(B)$ is a Stone
  space. \end{lemma}

\begin{proof} Let $\2B = \{ C_a\mid a\in B\}$ denote the chosen basis
  for the topology on $S(B)$.  To show that $S(B)$ is compact, it will
  suffice to show that for any subfamily $\2C$ of $\2B$, if $\2C$ has
  the finite intersection property, then $\bigcap \2C$ is nonempty.
  Now let $F$ be the set of $b\in B$ such that
  \[ C_{a_1}\cap \cdots \cap C_{a_n}\subseteq C_b ,\] for some
  $C_{a_1},\dots ,C_{a_n}\in \2C$.  Since $\2C$ has the finite
  intersection property, $F$ is a filter in $B$.  Thus, UF entails
  that $F$ is contained in an ultrafilter $U$.  This ultrafilter $U$
  corresponds to a $\phi :B\to 2$, and we have $\phi (a)=1$ whenever
  $C_a\in \2C$.  In other words, $\phi \in C_a$, whenever $C_a\in
  \2C$.  Therefore, $\bigcap \2C$ is nonempty, and $S(B)$ is compact.
\end{proof}

Let $f:A\to B$ be a homomorphism, and let $S(f):S(B)\to S(A)$ be given
by $S(f)=\hom (f,2)$; that is,
\[ S(f)(\phi ) \: = \: \phi \circ f ,\qquad \forall \phi \in S(B) .\]
We claim now that $S(f)$ is a continuous map.  Indeed, for any basic
open subset $C_a$ of $S(A)$, we have
\begin{equation} S(f)^{-1}(C_a) \: = \: \{ \phi \in S(B) \mid \phi
  (f(a)) = 1 \} \: = \: C_{f(a)} .\label{slick} \end{equation} It is
straightforward to verify that $S(1_A)=1_{S(A)}$, and that $S(g\circ
f)=S(f)\circ S(f)$.  Therefore, $S:\cat{Bool}\to\cat{Stone}$ is a
contravariant functor.



\subsection*{The functor from Stone to Bool}

Let $X$ be a Stone space.  Then the set $K(X)$ of clopen subsets of
$X$ is a Boolean algebra, and is a basis for the topology on $X$.  We
now show that $K$ is the object part of a contravariant functor
$K:\cat{Stone}\to\cat{Bool}$.  For reasons that will become clear
later, $K$ is sometimes called the \emph{syntactic functor}.

Indeed, if $X,Y$ are Stone spaces, and $f:X\to Y$ is continuous, then
for each clopen subset $U$ of $Y$, $f^{-1}(U)$ is a clopen subset of
$X$.  Moreover, $f^{-1}$ preserves union, intersection, and complement
of subsets; thus $f^{-1}:K(Y)\to K(X)$ is a Boolean homomorphism.  We
define the mapping $K$ on arrows by $K(f)=f^{-1}$.  Obviously,
$K(1_X)=1_{K(X)}$, and $K(g\circ f)=K(f)\circ K(g)$.  Therefore $K$ is
a contravariant functor.

Now we will show that $KS$ is naturally isomorphic to the identity on
$\cat{Bool}$, and $SK$ is naturally isomorphic to the identity on
$\cat{Stone}$.  For each Boolean algebra $B$, define $\eta _B:B\to
KS(B)$ by
\[ \eta _B(a) \: = \: C_a \: = \: \{ \phi \in S(B) \mid \phi (a)=1 \}
.\]

\begin{lemma} The map $\eta _B:B\to KS(B)$ is an isomorphism of
  Boolean algebras.  \end{lemma}

\begin{proof} We first verify that $a\mapsto C_a$ is a Boolean
  homomorphism.  For $a,b\in B$, we have
  \[ \begin{array}{l l l}  C_{a\wedge b} & = & \{ \phi \mid \phi (a\wedge b)=1 \} \\
    & = & \{ \phi \mid \phi(a)=1\;\text{and}\; \phi(b) =1
    \} \\
    & = & C_a\wedge C_b . \end{array} \] A similar calculation shows
  that $C_{\neg a}=X\backslash C_a$.  Therefore, $a\mapsto C_a$ is a
  Boolean homomorphism.

  To show that $a\mapsto C_a$ is injective, it will suffice to show
  that $C_a=\emptyset$ only if $a=0$.  In other words, it will suffice
  to show that for each $a\in B$, if $a\neq 0$ then there is some
  $\phi :B\to 2$ such that $\phi (a)=1$.  Thus, the result follows
  from UF.

  Finally, to see that $\eta _B$ is surjective, let $U$ be a clopen
  subset of $S(B)$.  Since $U$ is open, $U=\bigcup _{a\in I}C_a$, for
  some subset $I$ of $B$.  Since $U$ is closed in the compact space
  $G(B)$, it follows that $U$ is compact.  Thus, there is a finite
  subset $F$ of $B$ such that $U=\bigcup _{a\in F}C_a$.  And since
  $a\mapsto C_a$ is a Boolean homomorphism, $\bigcup _{a\in
    F}C_a=C_b$, where $b=\bigvee _{a\in F}a$.  Therefore, $\eta _B$ is
  surjective.
\end{proof}

\begin{lemma} The family of maps $\{ \eta _A:A\to KS(A) \}$ is natural
  in $A$.  \end{lemma}

\begin{proof} Suppose that $A$ and $B$ are Boolean algebras, and that
  $f:A\to B$ is a Boolean homomorphism.  Consider the following
  diagram:
\[ \begin{tikzcd}
  A \arrow{d}{\eta _A} \arrow{r}{f} & B \arrow{d}{\eta _B} \\
  KS(A) \arrow{r}{KS(f)} & KS(B) \end{tikzcd} \] For $a\in A$, we have
$\eta _B(f(a))=C_{f(a)}$, and $\eta _A(a)=C_a$.  Furthermore,
\[ KS(f)(C_a) \: = \: S(f)^{-1}(C_a) \: = \: C_{f(a)} ,\] by Eqn.\
\ref{slick}.  Therefore, the diagram commutes, and $\eta$ is a natural
transformation. \end{proof}

Now we define a natural isomorphism $\theta :1_{\mathbf{S}}\Rightarrow
SK$.  For a Stone space $X$, $K(X)$ is the Boolean algebra of clopen
subsets of $X$, and $SK(X)$ is the Stone space of $K(X)$.  For each
point $\phi \in X$, let $\hat{\phi}:K(X)\to 2$ be defined by
\[ \hat{\phi }(C) = \left\{ \begin{array}{l l} 1 & \phi\in C , \\ 0 &
    \phi\not\in C .\end{array}\right. \] It's straightforward to
verify that $\hat{\phi}$ is a Boolean homomorphism.  We define $\theta
_X:X\to SK(X)$ by $\theta _X(\phi )=\hat{\phi}$.  

\begin{lemma} The map $\theta _X:X\to SK(X)$ is a homeomorphism of
  Stone spaces. \end{lemma} 

\begin{proof} It will suffice to show that $\theta _X$ is bijective
  and continuous.  (Do you remember why?  Hint: Stone spaces are
  compact Hausdorff.)  To see that $\theta _X$ is injective, suppose
  that $\phi$ and $\psi$ are distinct elements of $X$.  Since $X$ is a
  Stone space, there is a clopen set $U$ of $X$ such that $\phi\in U$
  and $\psi\not\in U$.  But then $\hat{\phi}\neq \hat{\psi}$.  Thus,
  $\theta _X$ is injective.

To see that $\theta _X$ is surjective, let $h:K(X)\to 2$ be a Boolean
homomorphism.  Let
\[ \2C \: = \: \{ C\in K(X) \mid h(C) = 1 \} .\] In particular $X\in
\2C$; and since $h$ is a homomorphism, $\2C$ has the finite
intersection property.  Since $X$ is compact, $\bigcap \2C$ is
nonempty.  Let $\phi$ be a point in $\bigcap\2C$.  Then for any $C\in
K(X)$, if $h(C)=1$, then $C\in \2C$ and $\phi \in C$, from which it
follows that $\hat{\phi}(C)=1$.  Similarly, if $h(C)=0$ then
$X\backslash C\in \2C$, and $\hat{\phi }(C)=0$.  Thus, $\theta _X(\phi
)=\hat{\phi}=h$, and $\theta _X$ is surjective.

To see that $\theta _X$ is continuous, note that each basic open
subset of $SK(X)$ is of the form 
\[ \hat{C} \: = \: \{ h:K(X)\to 2 \mid h(C)=1 \} ,\] for some $C\in
K(X)$.  Moreover, for any $\phi \in X$, we have $\hat{\phi}\in
\hat{C}$ iff $\hat{\phi}(C)=1$ iff $\phi\in C$.  Therefore,
\[ \theta _X^{-1}(\hat{C}) \: = \: \{ \phi\in X \mid \hat{\phi}(C)=1
\} \: = \: C .\] Therefore, $\theta _X$ is continuous. \end{proof}

\begin{lemma} The family of maps $\{ \theta _X:X\to SK(X) \}$ is
  natural in $X$. \end{lemma}

\begin{proof} Let $X,Y$ be Stone spaces, and let $f:X\to Y$ be
  continuous.  Consider the diagram:
\[ \begin{tikzcd}
  X \arrow{r}{f} \arrow{d}{\theta _X} & Y \arrow{d}{\theta _Y} \\
  SK(X) \arrow{r}{SK(f)} & SK(Y)
\end{tikzcd} \] For arbitrary $\phi\in X$, we have $(\theta _Y\circ
f)(\phi )=\widehat{f(\phi )}$.  Furthermore,
\[ SK(f) \: = \: \hom (K(f),2) \: = \: \hom (f^{-1},2) \] In other
words, for a homomorphism $h:K(X)\to 2$, we have
\[ SK(f)(h) \: = \: h\circ f^{-1} .\] In particular, $SK(f)(\hat{\phi
})=\hat{\phi}\circ f^{-1}$.  For any $C\in K(Y)$, we have
\[ (\hat{\phi}\circ f^{-1})(C) \: = \: \left\{ \begin{array}{l l l}
    1 & f(\phi ) \in C ,\\
    0 & f(\phi ) \not\in C. \end{array} \right. \] That is,
$\hat{\phi}\circ f^{-1}=\widehat{f(\phi )}$.  Therefore, the diagram
commutes, and $\theta$ is a natural isomorphism.  \end{proof}

This completes the proof that $K$ and $S$ are quasi-inverse, and
yields the famous theorem:

\begin{box-thm}[Stone Duality Theorem] The categories $\mathbf{Stone}$
  and $\mathbf{Bool}$ are dual to each other.  In particular, any
  Boolean algebra $B$ is isomorphic to the field of clopen subsets of
  its state space $S(B)$.  \end{box-thm}


% \begin{prop} Let $B$ be a finite Boolean algebra.  Then $B\cong \2P
%   S(B)$, where $S(B)$ is the finite set of states of $B$. \end{prop}

% \begin{proof} As a set, $S(B)$ consists of homomorphisms from $B$ to
%   $2$.  If $B$ is finite, then $S(B)$ is finite, and every subset of
%   $S(B)$ is clopen.  Therefore, $B\cong KS(B)\cong \2PS(B)$.
% \end{proof}

% In fact, Stone duality provides a complete classification of finite
% Boolean algebras.  Each finite set $X$ (with discrete topology) is a
% Stone space, and $K(X)\cong \2P X$ is a Boolean algebra.  Moreover,
% the previous proposition shows that if $B$ is a finite Boolean
% algebra, then $B\cong \2P X$, for some finite set $X$.  Using the
% duality between $\cat{Stone}$ and $\cat{Bool}$, it follows directly
% that:

% \begin{thm} The powerset functor $\2P$ is one half of a duality
%   between the category $\cat{Sets}_f$ of finite sets and the category
%   $\cat{Bool}_f$ of finite Boolean algebras.  \end{thm}




% \begin{prop} Let $B$ be a Boolean algebra, and $A$ a proper subalgebra
%   of $B$.  Then there are states $f$ and $g$ of $B$ such that
%   $f|_A=g|_A$ by $f\neq g$. \label{baby-beth} \end{prop}

% \begin{proof} Let $m:A\to B$ be the inclusion of $A$ in $B$.  Then
%   $S(m):S(B)\to S(A)$ is an epimorphism.  If $S(m)$ were also
%   injective, then it would be an isomorphism (Proposition
%   \ref{stone-balanced}).  In this case, $m$ is also an isomorphism,
%   contradicting the assumption that $A$ is a proper subalgebra of $B$.
%   Thus, $S(m)$ is not injective.  That is, there are $f,g\in S(B)$
%   such that $f\neq g$ but $S(m)(f)=S(m)(g)$.  Finally, note that
%   $S(m)(f)=f\circ m$ is the restriction of $f$ to $A$, and
%   similarly for $S(m)(g)$.  Therefore, $f |_A=g |_A$.  \end{proof}




%% this is Beth's theorem

\begin{prop} Let $A\subseteq B$, and $a\in B$.  Then the following are
  equivalent:
\begin{enumerate}
\item For any states $f$ and $g$ of $B$, if $f|_A=g|_A$
  then $f(a)=g(a)$.
\item If $h$ is a state of $A$, then any two extensions of $h$ to $B$
  agree on $a$.
\item $a\in A$. \end{enumerate} \label{bool-beth} \end{prop}

\begin{proof} Since every state of $A$ can be extended to a state of
  $B$, (1) and (2) are obviously equivalent.  Furthermore, (3)
  obviously implies (1).  Thus, we only need to show that (1) implies
  (3).

  Let $m:A\to B$ be the inclusion of $A$ in $B$, and let $s:S(B)\to
  S(A)$ be the corresponding surjection of states.  We need to show
  that $C_a=s^{-1}(U)$ for some clopen subset $U$ of $S(A)$.

  By (1), for any $x\in S(A)$, either $s^{-1}\{ x\}\subseteq C_a$ or
  $s^{-1}\{ x\}\subseteq C_{\neg a}$.  By Proposition
  \ref{auto-closed}, $s$ is a closed map.  Since $C_a$ is open,
  Proposition \ref{closed-map} entails that the sets
  \[ U = \{ x\in S(B) \mid s^{-1}\{ x\} \subseteq C_a \} ,\quad
  \text{and} \quad V = \{ x\in S(B) \mid s^{-1}\{ x\} \subseteq
  C_{\neg a} \} ,\] are open.  Since $U=S(A)\backslash V$, it follows
  that $U$ is clopen.  Finally, it's clear that $s^{-1}(U)=C_a$.
\end{proof}

\begin{prop} In $\cat{Bool}$, epimorphisms are surjective. \end{prop}

\begin{proof} Suppose that $f:A\to B$ is not surjective.  Then $f(A)$
  is a proper subalgebra of $B$.  By Proposition \ref{bool-beth},
  there are states $g,h$ of $B$ such that $g\neq h$, but
  $g|_{f(A)}=h|_{f(A)}$.  In other words, $g\circ f=h\circ f$, and $f$
  is not an epimorphism.
\end{proof}


Combining the previous two theorems, we have the following
equivalences:
\[ \cat{Th} \: \cong \: \cat{Bool} \: \cong \: \cat{Stone}^{op} .\] We
will now exploit these equivalences to explore the structure of the
category of theories.

% \begin{prop} For each $n=0,1,\dots$, there is a unique propositional
%   theory with $n$ models. \end{prop}

% \begin{proof} Fix $n\in \7N$.  By Prop.\ \ref{}, there is a unique
%   Boolean algebra $B$ with Stone space $\{ 1,\dots ,n\}$.  Thus, $T_B$
%   is a theory with $n$ models.  Conversely, if $T$ is a theory with
%   $n$ models, then $|\hom (L(T),2)|=n$, from which it follows that
%   $L(T)\cong B$.  Thus, $T\cong T_{L(T)}\cong T_B$.  \end{proof}


% \begin{defn} If $T$ is a propositional theory, we let $M(T)$ denote
%   the set of all models of $T$. \end{defn}

% \begin{prop} If $T$ is a theory, then $M(T)\cong\hom
%   (L(T),2)$. \end{prop}

% \begin{proof} A model of $T$ is none other than an interpretation of
%   $T$ in $2$.  By Prop.\ \ref{lindenbaum}, interpretations of $T$ in
%   $2$ are in bijective correspondence with homomorphisms of $L(T)$ to
%   $2$. \end{proof}

% \begin{prop} If $B$ is a Boolean algebra, then $M(T_B)\cong \hom
%   (B,2)$.  \end{prop}

% \begin{proof} Since $B\cong L(T_B)$, we have $\hom (B,2)\cong \hom
%   (L(T_B),2)$.  By the previous Proposition, $\hom (L(T_B),2)\cong
%   M(T_B)$. \end{proof}

% \begin{prop} Let $T_0$ be the unique propositional theory whose
%   Lindenbaum algebra is the free algebra on countably many generators.
%   Let $\Sigma$ be a countably infinite signature, and let $T$ be an
%   arbitrary theory in $\Sigma$.  Then there is a surjective
%   interpretation $e:T\to T_0$, and there is a conservative
%   interpretation $m:T\to T_0$.  However, $T_0$ and $T$ are not
%   necessarily equivalent.  \end{prop}

% This shows, by the way, that metatheory of propositional logic is a
% first-order theory!  More is true, it's an algebraic theory --- and
% that tells us some things, e.g.\ Birkhoff's theorem.

\begin{prop} Let $T$ be a propositional theory in a countable
  signature.  Then there is a conservative translation $f:T\to T_0$,
  where $T_0$ is an empty theory, i.e.\ a theory with no
  axioms. \label{charity} \end{prop}

\begin{proof} After proving the above equivalences, we have several
  ways of seeing why this result is true.  In terms of Boolean
  algebras, the proposition says that every countable Boolean algebra
  is embeddable into the free Boolean algebra on a countable number of
  generators (i.e.\ the Boolean algebra of clopen subsets of the
  Cantor space).  That well-known result follows from the fact that
  Boolean algebras are always generated by their finite subalgebras.
  (In categorical terms, every Boolean algebra is a filtered colimit
  of finite Boolean algebras.)

  In terms of Stone spaces, the proposition says that for every
  separable Stone space $Y$, there is a continuous surjection
  $p:X\to Y$, where $X$ is the Cantor space.  That fact is well-known
  to topologists.  One interesting proof uses the fact that a Stone
  space $Y$ is {\it profinite}, i.e.\ $Y$ is a limits of finite
  Hausdorff (hence discrete) spaces.  One then shows that the Cantor
  space $X$ has enough surjections onto discrete spaces, and lifts
  these up to a surjection $p:X\to Y$.  See, for example,
  \cite{ribes}.  Or, for a more direct argument: each clopen subset
  $U$ of $Y$ corresponds to a continuous map $p_U:Y\to \{ 0,1\}$.
  There are countably many such clopen subsets of $Y$.  Since
  $X\simeq \prod_{i\in\7N}\{ 0,1\}$, these $p_U$ induce a continuous
  function $p:X\to Y$.  Moreover, since every point $y\in Y$ has a
  neighborhood basis of clopen sets, $p$ is surjective.
\end{proof}

\begin{disc}[Quine on eliminating posulates] \label{qgood} It's no
  surprise that one can be charitable to a fault.  Suppose that I am a
  theist, and you are an extremely charitable atheist.  You are so
  charitable that you want to affirm the things I say.  Here's how you
  can do it: when I say ``God'', assume that I really mean
  ``kittens.''  Then when I say ``God exists,'' you can interpret me
  to be saying ``kittens exist.''  Then you can smile and say: ``I
  completely agree!''

  Proposition \ref{charity} provides a general recipe for charitable
  interpretation.  Imagine that I accept a theory $T$, which might be
  controversial.  Imagine that you, on the other hand, like to play it
  safe: you only accept tautologies, viz.\ empty theory $T_0$.  The
  previous proposition shows that there is a conservative translation
  $f:T\to T_0$.  In other words, you can reinterpret my sentences in
  such a way that everything I say comes out as true by your lights,
  i.e.\ true by logic alone.

  Since we're dealing merely with propositional logic, this result
  might not seem very provocative.  However, a directly analogous
  result --- proven by \cite{quine-goodman,quine-implicit} --- was
  thought to refute the analytic-synthetic distinction that was
  central to the logical positivist program.  Quine's argument runs as
  follows: suppose that $T$ is intended to represent a contingently
  true theory, such as (presumably) quantum mechanics or evolutionary
  biology.  By making a series of clever definitions, the sentences of
  $T$ can be reconstrued as tautologies.  That is, any contingently
  true theory $T$ can be reconstrued so that all of its claims come
  out as true by definition.

  What we see here is an early instance of a strategy that Quine was
  to use again and again throughout his philosophical career.  There
  is a supposedly important distinction in a theory $T$.  Quine shows
  that this distinction doesn't survive translation of $T$ into some
  other theory $T_0$.  This result, Quine claims, shows that the
  distinction must be rejected.

  Whether or not Quine's strategy is generally good, we should be a
  bit suspicious in the present case.  The translation $f:T\to T_0$ is
  not an {\it equivalence} of theories, i.e.\ it does not show that
  $T$ is equivalent to $T_0$.  Since $f$ is conservative, it does show
  a sense in which $T$ is {\it embeddable in} or {\it reducible to}
  $T_0$.  But we are left wondering: why should the existence of a
  formal relation $f:T\to T_0$ undercut the importance of the
  distinctions that are made within $T$?
\end{disc}

If Proposition \ref{charity} was surprising, then the following result
is even more surprising:

\begin{prop} Let $T$ be a consistent propositional theory in a
  countably infinite signature.  If $T$ has a finite number of axioms,
  then $T$ is equivalent to the empty theory $T_0$. \end{prop}

%% TO DO: finish this

\begin{proof}[Sketch of proof] Suppose that $T$ has a finite number of
  axioms.  Without loss of generality, we assume that $T$ has a single
  axiom $\phi$.  Let $X$ be the Cantor space, i.e.\ the Stone space of
  the empty theory $T_0$.  Let $U_\phi\subseteq X$ be the clopen
  subset of all models in which $\phi$ is true.  Then $U_\phi$ is
  homeomorphic to the Stone space of $T$.  Assume for the moment any
  nonempty clopen subset of the Cantor space is homeomorphic to the
  Cantor space.  In that case, $U_\phi$ is homeomorphic to the Cantor
  space $X$; and by Stone duality, $T$ is equivalent to $T_0$.

  We now argue that nonempty clopen subset of the Cantor space is
  homeomorphic to the Cantor space.  (This result admits of several
  proofs, some more topologically illuminating than the one we give
  here.)  We begin by arguing that if $\phi$ is a conjunction of
  literals (atomic or negated atomic sentences), then $U_\phi$ is
  homeomorphic to the Cantor space.  Indeed, there is a direct proof
  that the theory $\{ \phi \}$ is equivalent to the empty theory;
  hence by Stone duality, $U_\phi$ is homeomorphic to $X$.  Now, an
  arbitrary clopen subset $U$ of $X$ has the form $U_\phi$ for some
  sentence $\phi$.  We may rewrite $\phi$ in disjunctive normal form,
  i.e.\ as a finite disjunction of conjunctions of literals.  Thus,
  $U_\phi$ is a disjoint union of
  $U_{\phi _1},U_{\phi _2},\dots ,U_{\phi _n}$.  By the previous
  argument, each $U_{\phi _i}$ is homeomorphic to $X$, and a disjoint
  union of copies of $X$ is also homeomorphic to $X$.
\end{proof}

The previous proposition might suggest that the notion of equivalence
we have adopted (Definition \ref{df:hom}) is too {\it liberal}, i.e.\
that it counts too many theories as equivalent.  If you think that's
the case, we enjoin you to propose another criterion, and to explore
its consequences.

\begin{disc} The Stone duality theorem has proven to be extremely
  fruitful in pure mathematics; and it also seems to offer some
  interesting new vistas for understanding the structure of scientific
  theories.  In particular, the Stone duality theorem suggests that to
  accept a theory involves, or can involve, certain commitments to
  claims about nearness relations between models/worlds/states.

  One theory $T$ leads to a particular topological structure on the
  space of possibilities, and another theory $T'$ leads to a different
  topological structure on the space of possibilities.  This might
  seem like a novelty to some philosophers, but scientists have always
  known it.  For example, when one accepts the general theory of
  relativity (GTR), one doesn't simply believe that our universe is
  isomorphic to one of its models.  Rather, one believes that the
  situtation we find ourselves in is one among many other situations
  that obey the laws of this theory.  Moreover, some such situations
  are more similar than others.  See \cite{fletcher} for an extended
  discussion of this example.
\end{disc}


\section{Notes}

We have given only the most cursory introduction to the rich
mathematical fields of Boolean algebras, topology, and the
interactions between them.  There is much more to be learned, and many
good books on these topics.  We mention some of our favorites below.

\begin{itemize}
\item For more on Boolean algebras, see
  \cite{sikorski-boo,dwinger,koppelberg,halmos,monk},
\item There are many good books on topology.  We learned originally
  from \cite{munkres}, and our favorites include \cite{engelking} and
  \cite{willard}.  The latter is notable for its presentation of the
  ultrafilter approach to convergence.
\item Stone spaces, being a particular kind of topological space, are
  sometimes mentioned in books about topology.  But for a more
  systematic treatment of Stone spaces, you'll need to consult other
  resources.  For a fully general and categorical treatment of Stone
  duality, see \cite{john-stone}.  For briefer and more pedestriaan
  treatments, see \cite{bell-machover,halmos1998,cori}.  For a proof
  that Stone spaces are profinite, see \cite{ribes}.
\end{itemize}

%% Proper maps are treated in
%%  Bourbaki, \textit{General Topology}, and in Escard{\'o},
%%  ``Intersections of compactly many open sets are open.''  






%%% Local Variables:
%%% mode: latex
%%% TeX-master: "main"
%%% End:
